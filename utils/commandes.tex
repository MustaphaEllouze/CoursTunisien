\newcounter{countitems}
\newcounter{nextitemizecount}
\newcommand{\setupcountitems}{%
  \stepcounter{nextitemizecount}%
  \setcounter{countitems}{0}%
  \preto\item{\stepcounter{countitems}}%
}
\makeatletter
\newcommand{\computecountitems}{%
  \edef\@currentlabel{\number\c@countitems}%
  \label{countitems@\number\numexpr\value{nextitemizecount}-1\relax}%
}
\newcommand{\nextitemizecount}{%
  \getrefnumber{countitems@\number\c@nextitemizecount}%
}
\newcommand{\previtemizecount}{%
  \getrefnumber{countitems@\number\numexpr\value{nextitemizecount}-1\relax}%
}
\makeatother    
\newenvironment{AutoMultiColItemize}{%
\ifnumcomp{\nextitemizecount}{>}{3}{\begin{multicols}{2}}{}%
\setupcountitems\begin{itemize}}%
{\end{itemize}%
\unskip\computecountitems\ifnumcomp{\previtemizecount}{>}{3}{\end{multicols}}{}}

%%%%%%%%%%%%%%%%%%%%%%%%%%%%%%%%%%%%%%%%%%%%%%%%%%%%%%%%%%%%%%%%%%%%%%%%%%%%%%%%%%%%%%%%%%%%%%%%%%%%%%%%%%%%%%%%%%%%%%%%%%%%%%%%%%%%%%%%%%%%%%%%%%%%%%%%%%%%

% Titre des chapitres en français
\renewcommand{\chaptername}{Chapitre}

% Titre de la table des matières en français
\renewcommand{\contentsname}{Table des matières}

% Titre des annexes en français
\renewcommand{\appendixname}{Annexe}

% Titre des parties en français
\renewcommand{\partname}{Partie}

% Commande \chapterletter pour faire les premières lettres de chapitres en grand
\newcommand{\chapterletter}[1]{%
    {\Huge \textbf{#1}}
}

%%%%%%%%%%%%%%%%%%%%%%%%%%%%%%%%%%%%%%%%%%%%%%%%%%%%%%%%%%%%%%%%%%%%%%%%%%%%%%
% Des commandes pour faciliter les caractères spéciaux 
\newcommand{\hb}{\textcrh}
\newcommand{\HB}{\Hwithstroke}
\newcommand{\vs}{\v{s}}
\newcommand{\VS}{\v{S}}
\newcommand{\va}{\v{a}}
\newcommand{\vo}{\v{o}}
\newcommand{\vi}{\v{i}}
\newcommand{\cs}{\c{s}}
\newcommand{\cdh}{\c{\dh}}
\newcommand{\ct}{\c{t}}
\newcommand{\ca}{\c{a}}

%%%%%%%%%%%%%%%%%%%%%%%%%%%%%%%%%%%%%%%%%%%%%%%%%%%%%%%%%%%%%%%%%%%%%%%%%%%%%%%
% Factorisation des pronoms personnels
\newcommand{\je}{'Éna }
\newcommand{\nous}{'A\textcrh na }
\newcommand{\tu}{'Enti }
\newcommand{\vous}{'Entuuma }
\newcommand{\il}{Huwwa }
\newcommand{\elle}{Hiyya }
\newcommand{\ils}{Huuma }

\newcommand{\jegras}{\textbf{\je}}
\newcommand{\tugras}{\textbf{\tu}}
\newcommand{\ilgras}{\textbf{\il}}
\newcommand{\ellegras}{\textbf{\elle}}
\newcommand{\nousgras}{\textbf{\nous}}
\newcommand{\vousgras}{\textbf{\vous}}
\newcommand{\ilsgras}{\textbf{\ils}}


%%%%%%%%%%%%%%%%%%%%%%%%%%%%%%%%%%%%%%%%%%%%%%%%%%%%%%%%%%%%%%%%%%%%%%%%%%%%%%%%%%%%%%%%%%%%%%%%%%%%%%%%%%%%%%%%%%%%%%%%%%%%%%%%%%%%%%%%%%%%%%%%%%%%%%%%%%%%

% Tableaux de conjugaisons avec des titres 
\newcommand{\conjugaisonTitreTableau}[2]{%
  \vspace{3mm}

  \begin{center}
    \fbox{\hspace{5mm}\LARGE \textbf{#1} \hspace{3mm}}
    \vspace{2mm}

    #2

  \vspace{3mm}
  \end{center}
}


% Helper function to unpack values
\def\unpackperson(#1, #2){%
    #1 & #2
}

\newcommand{\conjugaison}[8]{
  \begin{minipage}{115mm}
    \conjugaisonTitreTableau{#1}{
    \begin{tabular}{||c | c | c||}
      \hline
      \textbf{Pronom} & \textbf{Passé} & \textbf{Présent} \\
      \hline\hline
      \je & \unpackperson(#2)\\ \hline
      \tu & \unpackperson(#3)\\ \hline
      \il & \unpackperson(#4)\\ \hline
      \elle & \unpackperson(#5)\\ \hline
      \nous & \unpackperson(#6)\\ \hline
      \vous & \unpackperson(#7)\\ \hline
      \ils & \unpackperson(#8)\\ \hline
     \end{tabular}
    }
  \end{minipage}
}

%%%%%%%%%%%%%%%%%%%%%%%%%%%%%%%%%%%%%%%%%%%%%%%%%%%%%%%%%%%%%%%%%%%%%%%%%%%%%%%%%