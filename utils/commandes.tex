\newcounter{countitems}
\newcounter{nextitemizecount}
\newcommand{\setupcountitems}{%
  \stepcounter{nextitemizecount}%
  \setcounter{countitems}{0}%
  \preto\item{\stepcounter{countitems}}%
}
\makeatletter
\newcommand{\computecountitems}{%
  \edef\@currentlabel{\number\c@countitems}%
  \label{countitems@\number\numexpr\value{nextitemizecount}-1\relax}%
}
\newcommand{\nextitemizecount}{%
  \getrefnumber{countitems@\number\c@nextitemizecount}%
}
\newcommand{\previtemizecount}{%
  \getrefnumber{countitems@\number\numexpr\value{nextitemizecount}-1\relax}%
}
\makeatother    
\newenvironment{AutoMultiColItemize}{%
\ifnumcomp{\nextitemizecount}{>}{3}{\begin{multicols}{2}}{}%
\setupcountitems\begin{itemize}}%
{\end{itemize}%
\unskip\computecountitems\ifnumcomp{\previtemizecount}{>}{3}{\end{multicols}}{}}

%%%%%%%%%%%%%%%%%%%%%%%%%%%%%%%%%%%%%%%%%%%%%%%%%%%%%%%%%%%%%%%%%%%%%%%%%%%%%%%%%%%%%%%%%%%%%%%%%%%%%%%%%%%%%%%%%%%%%%%%%%%%%%%%%%%%%%%%%%%%%%%%%%%%%%%%%%%%

% Titre des chapitres en français
\renewcommand{\chaptername}{Chapitre}

% Titre de la table des matières en français
\renewcommand{\contentsname}{Table des matières}

% Commande \chapterletter pour faire les premières lettres de chapitres en grand
\newcommand{\chapterletter}[1]{%
    {\Huge \textbf{#1}}
}


%%%%%%%%%%%%%%%%%%%%%%%%%%%%%%%%%%%%%%%%%%%%%%%%%%%%%%%%%%%%%%%%%%%%%%%%%%%%%%%%%%%%%%%%%%%%%%%%%%%%%%%%%%%%%%%%%%%%%%%%%%%%%%%%%%%%%%%%%%%%%%%%%%%%%%%%%%%%

% Tableaux de conjugaisons avec des titres 
\newcommand{\conjugaisonTitreTableau}[2]{%
  \begin{center}
    \fbox{\hspace{5mm}\LARGE \textbf{#1} \hspace{5mm}}
    \vspace{2mm}

    #2
  \end{center}
}


% Helper function to unpack values
\def\unpackperson(#1, #2){%
    #1 & #2
}

\newcommand{\conjugaison}[8]{
    \conjugaisonTitreTableau{#1}{
    \begin{tabular}{||c | c | c||}
      \hline
      \textbf{Pronom} & \textbf{Passé} & \textbf{Présent} \\
      \hline\hline
      'Éna & \unpackperson(#2)\\
      \hline
      'Enti & \unpackperson(#3)\\ 
      \hline
      Huwwa & \unpackperson(#4)\\ 
      \hline
      Hiyya & \unpackperson(#5)\\ 
      \hline
      'A\textcrh na & \unpackperson(#6)\\ 
      \hline
      'Entuuma & \unpackperson(#7)\\ 
      \hline
      Huuma & \unpackperson(#8)\\ 
      \hline
     \end{tabular}
    }
}

%%%%%%%%%%%%%%%%%%%%%%%%%%%%%%%%%%%%%%%%%%%%%%%%%%%%%%%%%%%%%%%%%%%%%%%%%%%%%%%%%