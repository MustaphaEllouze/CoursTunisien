\chapter*{Comment ce cours a été conçu ?}
\chapterletter{J'}ai construit ce cours en tant qu'introduction à ma langue maternelle, le tunisien. Il ne se targue pas d'être un cours de linguistique précis, ou cours académique en tunisien. Il vise plutôt un apprentissage simple et intuitif, destiné dans un premier temps aux francophones, et dans un deuxième temps aux tunisophones qui souhaiteraient en apprendre plus sur leur langue, et qui pourront faire le lien avec la seule langue arabe qu'ils aient généralement vue à l'école, c'est-à-dire l'arabe moderne standard. 

Je trouvais dommage que la langue tunisienne ne soit pas enseignée à l'école primaire, et je n'ai appris que récemment qu'il existait des ouvrages afin de l'apprendre. Cependant, je trouvais que ces ouvrages, ou en tout cas ceux que j'ai eu le loisir de lire, se plongeait beaucoup trop rapidement dans les phrases et expressions courantes, pour délaisser le côté grammatical, qui est pourtant enseigné lors de l'apprentissage de toute autre langue.

Rappelons que le tunisien n'a jamais été une langue standardisée. Malgré le fait qu'il soit dérivé de l'arabe (une langue écrite), que plusieurs articles, journaux de l'époque et publicités modernes aient essayé de l'écrire, il n'y a aucun effort de standardisation de l'écriture et de la langue orale qui n'ait réussi à s'imposer. Jusqu'à aujourd'hui, l'ensemble des personnes que je croise écrivent systématiquement le tunisien en phonétique, en se servant de l'alphabet latin ou de l'abjad arabe.

Avec ce cours, j'ai essayé de proposer une manière intuitive pour les francophones de lire le tunisien. Ainsi, j'ai fait le choix d'adopter un système d'écriture sur la base de l'alphabet latin, agrémenté de symboles empruntés ça et là afin de représenter les sons manquants. J'espère un jour pouvoir changer ceci, le jour où le tunisien sera standardisé avec un alphabet adapté à sa prononciation. 

Il faut également noter que le tunisien est par essence une langue qui a été fortement influencée par les langues voisines et les langues de ces anciens occupants : français, italien, berbère, turc, et de nos jours l'anglais. Dans la mesure du possible, j'essayerai tant que possible de préciser l'origine de certaines expressions, constructions grammaticales, ou mots de vocabulaire, afin de contenter les curieux (comme moi) et de permettre aux lecteurs, peu importe leur origine, de faire le pont avec les autres langues qu'ils connaissent.