\documentclass[draft]{book}
% \documentclass{book}

\usepackage{arabtex}
\usepackage{cjhebrew}
\usepackage{utf8}
\usepackage{etoolbox,refcount}
\usepackage{multicol}
\usepackage{tabularx}
\usepackage{tipa}
\usepackage[T1]{fontenc}
\usepackage{color}
\usepackage{xcolor}
\usepackage{makecell}
\setcode{utf8}

% Pour les liens hypertexte
\usepackage{hyperref}
\hypersetup{
    final,
    hidelinks,
}

% \lang      {french}
% \title     {Cours d'Arabe Tunisien}
% \subtitle  {}
% \authors   {Mustapha ELLOUZE}
% \cover     {ressources/partie1.jpg}{ressources/partie2.jpg}
% \edition   {1}{2023}

% \note{Ce cours s'adresse aux débutants en arabe tunisien, qu'ils aient des affinités ou non avec l'arabe moderne standard. Il s'appuie surtout sur mon expérience en tant que locuteur natif, en espérant que ni ma mémoire ni mes villes d'origine ne me trahissent.}

% \blurb{Ce cours s'adresse aux débutants en arabe tunisien, qu'ils aient des affinités ou non avec l'arabe moderne standard. Il s'appuie surtout sur mon expérience en tant que locuteur natif, en espérant que ni ma mémoire ni mes villes d'origine ne me trahissent.}

\begin{document}
\newcounter{countitems}
\newcounter{nextitemizecount}
\newcommand{\setupcountitems}{%
  \stepcounter{nextitemizecount}%
  \setcounter{countitems}{0}%
  \preto\item{\stepcounter{countitems}}%
}
\makeatletter
\newcommand{\computecountitems}{%
  \edef\@currentlabel{\number\c@countitems}%
  \label{countitems@\number\numexpr\value{nextitemizecount}-1\relax}%
}
\newcommand{\nextitemizecount}{%
  \getrefnumber{countitems@\number\c@nextitemizecount}%
}
\newcommand{\previtemizecount}{%
  \getrefnumber{countitems@\number\numexpr\value{nextitemizecount}-1\relax}%
}
\makeatother    
\newenvironment{AutoMultiColItemize}{%
\ifnumcomp{\nextitemizecount}{>}{3}{\begin{multicols}{2}}{}%
\setupcountitems\begin{itemize}}%
{\end{itemize}%
\unskip\computecountitems\ifnumcomp{\previtemizecount}{>}{3}{\end{multicols}}{}}

%%%%%%%%%%%%%%%%%%%%%%%%%%%%%%%%%%%%%%%%%%%%%%%%%%%%%%%%%%%%%%%%%%%%%%%%%%%%%%%%%%%%%%%%%%%%%%%%%%%%%%%%%%%%%%%%%%%%%%%%%%%%%%%%%%%%%%%%%%%%%%%%%%%%%%%%%%%%

% Titre des chapitres en français
\renewcommand{\chaptername}{Chapitre}

% Titre de la table des matières en français
\renewcommand{\contentsname}{Table des matières}

% Titre des annexes en français
\renewcommand{\appendixname}{Annexe}

% Titre des parties en français
\renewcommand{\partname}{Partie}

% Commande \chapterletter pour faire les premières lettres de chapitres en grand
\newcommand{\chapterletter}[1]{%
    {\Huge \textbf{#1}}
}

%%%%%%%%%%%%%%%%%%%%%%%%%%%%%%%%%%%%%%%%%%%%%%%%%%%%%%%%%%%%%%%%%%%%%%%%%%%%%%
% Des commandes pour faciliter les caractères spéciaux 
\newcommand{\hb}{\textcrh}
\newcommand{\HB}{\Hwithstroke}
\newcommand{\vs}{\v{s}}
\newcommand{\va}{\v{a}}
\newcommand{\vo}{\v{o}}
\newcommand{\vi}{\v{i}}
\newcommand{\cs}{\c{s}}
\newcommand{\cdh}{\c{\dh}}
\newcommand{\ct}{\c{t}}
\newcommand{\ca}{\c{a}}

%%%%%%%%%%%%%%%%%%%%%%%%%%%%%%%%%%%%%%%%%%%%%%%%%%%%%%%%%%%%%%%%%%%%%%%%%%%%%%%
% Factorisation des pronoms personnels
\newcommand{\je}{'Éna }
\newcommand{\nous}{'A\textcrh na }
\newcommand{\tu}{'Enti }
\newcommand{\vous}{'Entuuma }
\newcommand{\il}{Huwwa }
\newcommand{\elle}{Hiyya }
\newcommand{\ils}{Huuma }

\newcommand{\jegras}{\textbf{\je}}
\newcommand{\tugras}{\textbf{\tu}}
\newcommand{\ilgras}{\textbf{\il}}
\newcommand{\ellegras}{\textbf{\elle}}
\newcommand{\nousgras}{\textbf{\nous}}
\newcommand{\vousgras}{\textbf{\vous}}
\newcommand{\ilsgras}{\textbf{\ils}}


%%%%%%%%%%%%%%%%%%%%%%%%%%%%%%%%%%%%%%%%%%%%%%%%%%%%%%%%%%%%%%%%%%%%%%%%%%%%%%%%%%%%%%%%%%%%%%%%%%%%%%%%%%%%%%%%%%%%%%%%%%%%%%%%%%%%%%%%%%%%%%%%%%%%%%%%%%%%

% Tableaux de conjugaisons avec des titres 
\newcommand{\conjugaisonTitreTableau}[2]{%
  \vspace{3mm}

  \begin{center}
    \fbox{\hspace{5mm}\LARGE \textbf{#1} \hspace{3mm}}
    \vspace{2mm}

    #2

  \vspace{3mm}
  \end{center}
}


% Helper function to unpack values
\def\unpackperson(#1, #2){%
    #1 & #2
}

\newcommand{\conjugaison}[8]{
  \begin{minipage}{115mm}
    \conjugaisonTitreTableau{#1}{
    \begin{tabular}{||c | c | c||}
      \hline
      \textbf{Pronom} & \textbf{Passé} & \textbf{Présent} \\
      \hline\hline
      \je & \unpackperson(#2)\\ \hline
      \tu & \unpackperson(#3)\\ \hline
      \il & \unpackperson(#4)\\ \hline
      \elle & \unpackperson(#5)\\ \hline
      \nous & \unpackperson(#6)\\ \hline
      \vous & \unpackperson(#7)\\ \hline
      \ils & \unpackperson(#8)\\ \hline
     \end{tabular}
    }
  \end{minipage}
}

%%%%%%%%%%%%%%%%%%%%%%%%%%%%%%%%%%%%%%%%%%%%%%%%%%%%%%%%%%%%%%%%%%%%%%%%%%%%%%%%%

\h*{Comment ce cours a été conçu ?}
\l{J}'ai construit ce cours en tant qu'introduction à ma langue maternelle, le tunisien. Il ne se targue pas d'être un cours de linguistique précis, ou cours académique en tunisien. Il vise plutôt un apprentissage simple et intuitif, destiné dans un premier temps aux francophones, et dans un deuxième temps aux tunisophones qui souhaiteraient en apprendre plus sur leur langue, et qui pourront faire le lien avec la seule langue arabe qu'ils aient généralement vue à l'école, c'est-à-dire l'arabe moderne standard. 

Je trouvais dommage que la langue tunisienne ne soit pas enseignée à l'école primaire, et je n'ai appris que récemment qu'il existait des ouvrages afin de l'apprendre. Cependant, je trouvais que ces ouvrages, ou en tout cas ceux que j'ai eu le loisir de lire, se plongeait beaucoup trop rapidement dans les phrases et expressions courantes, pour délaisser le côté grammatical, qui est pourtant enseigné lors de l'apprentissage de toute autre langue.

Rappelons que le tunisien n'a jamais été une langue standardisée. Malgré le fait qu'il soit dérivé de l'arabe (une langue écrite), que plusieurs articles, journaux de l'époque et publicités modernes aient essayé de l'écrire, il n'y a aucun effort de standardisation de l'écriture et de la langue orale qui n'ait réussi à s'imposer. Jusqu'à aujourd'hui, l'ensemble des personnes que je croise écrivent systématiquement le tunisien en phonétique, en se servant de l'alphabet latin ou de l'abjad arabe.

Avec ce cours, j'ai essayé de proposer une manière intuitive pour les francophones de lire le tunisien. Ainsi, j'ai fait le choix d'adopter un système d'écriture sur la base de l'alphabet latin, agrémenté de symboles empruntés ça et là afin de représenter les sons manquants. J'espère un jour pouvoir changer ceci, le jour où le tunisien sera standardisé avec un alphabet adapté à sa prononciation. 

Il faut également noter que le tunisien est par essence une langue qui a été fortement influencée par les langues voisines et les langues de ces anciens occupants : français, italien, berbère, turc, et de nos jours l'anglais. Dans la mesure du possible, j'essayerai tant que possible de préciser l'origine de certaines expressions, constructions grammaticales, ou mots de vocabulaire, afin de contenter les curieux (comme moi) et de permettre aux lecteurs, peu importe leur origine, de faire le pont avec les autres langues qu'ils connaissent.

\tableofcontents

\chapter{Transcription adoptée et prononciation}
\chapterletter{D}ans ce chapitre, je vous propose d'en apprendre plus sur la transcription que j'ai choisie afin d'écrire le tunisien dans ce cours, ainsi que les différents sons qui composent la langue.

\section{Sons que vous pouvez rencontrer en tunisien}
Nous parlerons dans cette section de l'ensemble des sons que nous pouvons rencontrer en tunisien. Lorsqu'on parle de sons, on ne parle pas des lettres qui composent l'alphabet, mais bien de l'ensemble des phonèmes qu'un locuteur peut prononcer et sait distinguer. Ainsi, si l'alphabet français comprend 5 voyelles, un locuteur français peut en réalité prononcer jusqu'à 20 phonèmes vocaliques différents (en fonction du dialecte et de l'accent) !

L'inventaire phonétique consonantique du tunisien est particulièrement riche. Ainsi, on dit souvent en Tunisie que lorsqu'on parle tunisien, ou une autre langue d'origine arabe en général, il est relativement aisé de réaliser la plupart des consonnes du français (et des langues latines en général) et de l'anglais. 

A l'inverse, l'inventaire phonétique vocalique de l'arabe étant relativement pauvre, celui du tunisien l'est également, notamment quand on le compare à celui du français. On comprend alors facilement pourquoi les locuteurs d'origine arabe ont souvent beaucoup de mal avec certaines voyelles dont ils n'ont pas l'habitude; l'exemple parfait étant le mot "électricité", où toutes les voyelles sont prononcées /i/.

\subsection{Consonnes d'origine arabe}
Les consonnes que vous pouvez retrouver en tunisien ont pour la plupart une origine arabe\footnote{Les autres ont fait leur apparition avec des mots importés du français, de l'anglais ou de l'italien, et n'apparaissent donc exclusivement que dans des mots originaires de ces langues.}. En voici un tableau récapitulatif :

\begin{center}
\begin{tabular}{||c c c||} 
 \hline
 Transcription arabe & Transcription phonétique & Transcriptions \\ [2.5ex] 
 \hline\hline
 \RL{ا} / \RL{ء} & ['] & ' \\ 
 \hline
 \RL{ب} & [b] & b \\ 
 \hline
 \RL{ت} & [t] & t \\ 
 \hline
 \RL{ث} & [\texttheta] & th \\ 
 \hline
 \RL{ج} & [\textyogh] & j \\ 
 \hline
 \RL{ح} & [h] & h / 7 \\ 
 \hline
 \RL{خ} & [\textchi] & kh / 5 \\ 
 \hline
 \RL{د} & [d] & d \\ 
 \hline
 \RL{ذ} & [\dh] & dh / 4 \\ 
 \hline
 \RL{ر} & [r] & r \\ 
 \hline
 \RL{ز} & [z] & z \\ 
 \hline
 \RL{س} & [s] & s \\ 
 \hline
 \RL{ش} & [\textesh] & ch / sh \\ 
 \hline
 \RL{ص} & [s\super \textrevglotstop] & s \\ 
 \hline
 \RL{ض}/\RL{ظ} & [\dh \super \textrevglotstop] & dh / 4 \\ 
 \hline
 \RL{ط} & [t\super \textrevglotstop] & t \\ 
 \hline
 \RL{ع} & [\textrevglotstop] & aa / \textroundcap{a} / 3 \\ 
 \hline
 \RL{غ} & [\textinvscr] & gh \\ 
 \hline
 \RL{ف} & [f] & f \\ 
 \hline
 \RL{ق} & [q] & q / 9 \\ 
 \hline
 \RL{ك} & [k] & c / k \\ 
 \hline
 \RL{ل} & [l] & l \\ 
 \hline
 \RL{م} & [m] & m \\ 
 \hline
 \RL{ن} & [n] & n \\ 
 \hline
 \RL{ه} & [\texthth] & h \\ 
 \hline
 \RL{و} & [w] & w \\ 
 \hline
 \RL{ي} & [y] & y \\  [2.5ex] 
 \hline
\end{tabular}
\end{center}

La plupart des consonnes de l'arabe standard ont été transmises au tunisien. Si vous vous documentiez sur l'alphabet arabe, vous trouveriez que j'ai inscrit au-dessus l'ensemble des lettres qui y sont présentes. La différence notable réside dans les deux consonnes \RL{ض} et \RL{ظ}. Au fur des années, la prononciation de ces deux consonnes s'est rapprochée, jusqu'à ce qu'elles se prononcent identiquement, du moins dans mon accent du tunisien, qui correspond au dialecte parlé à la radio et à la télé\footnote{Dans certains dialectes régionaux, une légère différence peut être trouvée entre ces deux lettres.}.

Ensuite, vous vous êtes sans doute interrogés, sauf si vous êtes déjà familier avec, sur la "transcription phonétique". Cette transcription correspond aux symboles définis dans l'alphabet phonétique international. Plusieurs symboles ([b] ou [t]) restent compréhensibles du grand public, mais certains sont obscurs pour un non-initié. Dans la suite de la section, je vous détaillerai la prononciation exacte de chacune de ces consonnes.

Finalement, la troisième colonne vous propose plusieurs transcriptions possibles pour un même son, telles qu'on peut les retrouver en langage "SMS" (sur les réseaux sociaux par exemple), ou en transcription n'utilisant que des lettres de l'alphabet latin (rare, mais particulièrement populaire chez les personnes nées avant le début des années 80).

Je vous propose maintenant de parcourir ensemble chaque consonne et de détailler sa prononciation.


\textbf{| Prononciation de \RL{ا} / \RL{ء} [']}

Ce phonème correspond à la lettre arabe \RL{ا}. C'est un phonème qui n'est d'habitude pas jugé par les non-linguistes comme une consonne à part entière, cependant il correspond au \textbf{/h/ aspiré} en français. On peut l'entendre par exemple entre les deux mots dans \textbf{les haricots} : la liaison n'est pas faite et une légère pause est marquée; on ne dira ni \textbf{/lézarico/} ni \textbf{/léaricot/}, mais bien \textbf{/lé arico/} On également entendre ce son entre les deux \textbf{/o/} du mot \textbf{zoo}.


\textbf{| Prononciation de \RL{ب} [b]}

Ce phonème correspond à la lettre arabe \RL{ب}. Il est l'un des phonèmes les plus répandus dans les langues internationales, et se prononce comme le \textbf{/b/} en français, comme dans les mots \textbf{bébé} ou \textbf{bateau}.



\textbf{| Prononciation de \RL{ت} [t]}

Ce phonème correspond à la lettre arabe \RL{ت}. Il est l'un des phonèmes les plus répandus dans les langues internationales, et se prononce comme le \textbf{/t/} en français, comme dans les mots \textbf{tuyau} ou \textbf{table}.



\textbf{| Prononciation de \RL{ث} [\texttheta]}

Ce phonème correspond à la lettre arabe \RL{ث}. Ce phonème se retrouve en anglais avec la retranscription \textbf{/th/}, comme dans \textbf{thorn} ou \textbf{thin}. Pour prononcer ce son, il suffit de prononcer un \textbf{/s/} avec la langue coincée entre les dents. Autre méthode, si vous êtes familier avec le son \textbf{/th/} dans le mot \textbf{the} \textbf{[\dh]} : \textbf{[\dh]} est à \textbf{[\texttheta]} ce que un \textbf{[d]} est à \textbf{[t]} : pour passer d'un son à l'autre, il suffit de faire vibrer ou non ses cordes vocales (on dit que ces consonnes sont respectivement sonores et sourdes).

\textbf{| Prononciation de \RL{ج} [\textyogh]}

Ce phonème correspond à la lettre arabe \RL{ج}. Derrière le symbole phonétique se cache simplement le son \textbf{/j/} comme on peut le retrouver en français dans les mots \textbf{jeu} et \textbf{girouette}.


\textbf{| Prononciation de \RL{ح} [h]}

Ce phonème correspond à la lettre arabe \RL{ح}. L'exemple le plus parlant de l'utilisation de ce phonème à l'international est la prononciation de \textbf{México / Méjico} en espagnol, prononcée à la mexicaine et non à la castillane. Une façon d'arriver à prononcer ce son revient à prononcer la \textbf{version sourde de /h/}, /h/ se retrouvant dans les mots anglais \textbf{hungry} ou \textbf{high}. Une autre façon de réaliser cette consonne est de s'imaginer expirer de l'air sur sa main, comme si on voulait sentir son haleine.

\textbf{| Prononciation de \RL{خ} [\textchi]}

Ce phonème correspond à la lettre arabe \RL{خ}. C'est un son qui peut sembler difficile à prononcer pour un francophone, car il n'est prononcé en français que dans un contexte consonantique assez particulier. On peut le retrouver lorsqu'on prononce le son \textbf{/r/} lorsqu'il suit les phonèmes \textbf{/k/} et \textbf{/t/}, comme par exemple dans les mots \textbf{crin} et \textbf{train}. Contrairement à la croyance générale, les réalisations de ces \textbf{/r/} diffèrent du "r classique" : essayez de prononcer le mot \textbf{rein} par exemple.


\textbf{| Prononciation de \RL{د} [d]}

Ce phonème correspond à la lettre arabe \RL{د}. Il est l'un des phonèmes les plus répandus dans les langues internationales, et se prononce comme le \textbf{/d/} en français, comme dans les mots \textbf{décoration} ou \textbf{diminuer}.


\textbf{| Prononciation de \RL{ذ} [\dh]}

Ce phonème correspond à la lettre arabe \RL{ذ}. Il est la version \textbf{sonore} du phonème \RL{ث} [\texttheta]. A ce titre, on le retrouve dans les mots anglais \textbf{the} ou \textbf{then}. Pour prononcer ce son, il suffit de prononcer un \textbf{/z/} avec la langue coincée entre les dents.


\textbf{| Prononciation de \RL{ر} [r]}

Ce phonème correspond à la lettre arabe \RL{ر}. Il est le son qui correspond à ce qu'on désigne par \textbf{r roulé} en français. On le retrouve en espagnol, dans le mot \textbf{perro (chien)}\footnote{Dans certains dialectes du tunisien (typiquement le sfaxien), la réalisation est un tout petit peu différente, et ressemble plus la prononciation du mot espagnol \textbf{pero (mais)}. Techniquement, on ne parle alors pas d'un \textbf{r roulé} mais plutôt d'un \textbf{r battu}. La différence est assez subtile aux oreilles d'un tunisien, vous pouvez donc tout à fait ignorer cette note.}. Si vous avez des difficultés pour le prononcer, essayez tout d'abord de positionner votre langue dans la zone alvéolaire de la bouche (c'est la zone située juste avant les dents, celle où vous prononcez les \textbf{/n/}, \textbf{/t/} et \textbf{d}), sans que votre langue ne rentre en contact avec votre palais. Il s'agit ensuite d'insuffler juste assez peu d'air pour que votre langue se mette à entrer en vibration.


\textbf{| Prononciation de \RL{ز} [z]}

Ce phonème correspond à la lettre arabe \RL{ز}. Il se prononce comme le \textbf{/z/} en français, comme dans les mots \textbf{zèbre} ou \textbf{zoo}.


\textbf{| Prononciation de \RL{س} [s]}

Ce phonème correspond à la lettre arabe \RL{س}. Il est l'un des phonèmes les plus répandus dans les langues internationales, et se prononce comme le \textbf{/s/} en français, comme dans les mots \textbf{sauter} ou \textbf{salade}.



\textbf{| Prononciation de \RL{ش} [\textesh]}

Ce phonème correspond à la lettre arabe \RL{ش}. Il se prononce comme le \textbf{/ch/} en français, comme dans les mots \textbf{cheval} ou \textbf{chute}.


\textbf{| Prononciation de \RL{ص} [s\super \textrevglotstop]}

Ce phonème correspond à la lettre arabe \RL{ص}. Ce type de consonnes, appelées consonnes pharyngalisées, n'est quasiment rencontré qu'en arabe et en hébreu. Dans les manuels d'arabe, on parlera de \textbf{consonnes emphatiques}. Afin de prononcer correctement ce son, il convient de prononcer \textbf{/s/} tout en \textbf{contractant le pharynx} et en \textbf{rapprochant l'arrière de votre langue de votre luette} (ce n'est toutefois pas un exercice très simple)\footnote{Essayez de penser au mot \textbf{ça} prononcé à la québécoise.}. 



\textbf{| Prononciation de \RL{ض}/\RL{ظ} [\dh \super \textrevglotstop]}

Ce phonème correspond aux lettres arabes \RL{ض}/\RL{ظ}. C'est également une \textbf{consonne emphatique}. Afin de prononcer correctement ce son, il convient de prononcer \textbf{/dh/ [\dh]} tout en \textbf{contractant le pharynx} et en \textbf{rapprochant l'arrière de votre langue de votre luette}\footnote{De la même manière, vous pouvez penser au mot \textbf{ça} prononcé à la québécoise, en remplaçant à la consonne par celle qui va bien.}. 


\textbf{| Prononciation de \RL{ط} [t \super \textrevglotstop]}

Ce phonème correspond à la lettre arabe \RL{ط}. C'est également une \textbf{consonne emphatique}. Afin de prononcer correctement ce son, il convient de prononcer \textbf{/t/} tout en \textbf{contractant le pharynx} et en \textbf{rapprochant l'arrière de votre langue de votre luette}. 


\textbf{| Prononciation de \RL{ع} [\textrevglotstop]}

Ce phonème correspond à la lettre arabe \RL{ع}. Le son n'est pas systématiquement reconnu comme étant une \textbf{consonne emphatique} mais c'est bien le son désigné par \RL{ع} [\textrevglotstop] qui sert de base à la prononciation des autres consonnes emphatiques. Par ailleurs, en linguistique, on parlera de \RL{ع} comme d'une consonne spirante ou d'une approximante ;  alors qu'en dehors de la sphère linguistique, on parlera de semi-voyelle comme \textbf{/w/} ou \textbf{/y/}, même si d'un point de vue grammatical l'arabe ne considère pas \RL{ع} comme telle. Afin de prononcer correctement ce son, il convient de \textbf{contracter le pharynx} tout en \textbf{rapprochant l'arrière de votre langue de votre luette}. Vu la complexité de la tâche et le fait que le son ne soit accompagné d'aucun autre (voir la prononciation de \RL{ص} [s\super \textrevglotstop] par exemple), il fait parti des deux phonèmes avec lequel les francophones ont le plus de mal\footnote{Une manière pas très jolie de vous aider à prononcer cette consonne est de vous souvenir du son que vous avez produit avec votre gorge la dernière fois que vous avez abusé de l'alcool, ou manger des fruits de mer pas frais.}.


\textbf{| Prononciation de \RL{غ} [\textinvscr]}

Ce phonème correspond à la lettre arabe \RL{غ}. C'est un son qui peut sembler très familier pour un francophone, car il ressemble très fortement au \textbf{/r/} du français. Cependant, la prononciation exacte est légèrement différente, car il n'est prononcé en français que dans un contexte consonantique assez particulier. On peut le retrouver lorsqu'on prononce le son \textbf{/r/} lorsqu'il suit les phonèmes \textbf{/g/} et \textbf{/d/}, comme par exemple dans les mots \textbf{grain} et \textbf{drain}. Contrairement à la croyance générale, les réalisations de ces \textbf{/r/} diffèrent du "r classique" : essayez de prononcer le mot \textbf{rein} par exemple\footnote{Ce son n'est en réalité que la version sonore de \RL{خ} [\textchi].}\footnote{En pratique, vous pouvez vous contenter de prononcer ce son comme le \textbf{/r/} de \textbf{rein}, un tunisophone ne fera pas nécessairement la différence, d'autant plus que le son \textbf{/r/} du français n'existe pas en tunisien.}.

\textbf{| Prononciation de  \RL{ف} [f]}

Ce phonème correspond à la lettre arabe \RL{ف}. Il est l'un des phonèmes les plus répandus dans les langues internationales, et se prononce comme le \textbf{/f/} en français, comme dans les mots \textbf{faire} ou \textbf{foin}.


\textbf{| Prononciation de  \RL{ق} [q]}

Ce phonème correspond à la lettre arabe \RL{ق}. Avec \RL{ع}, il constitue facilement le phonème avec lequel les non-arabophones ont le plus de mal. Sa prononciation se fait similairement au son \textbf{/k/}, sauf que les deux parties de votre bouche qui doivent être en contact sont \textbf{l'arrière de votre langue et votre luette}\footnote{Afin de vous aider à prononcer ce son, vous pouvez vous allonger tête vers le haut, et essayer de reproduire le son \textbf{/k/} : la gravité fera naturellement en sorte que votre langue se rapproche de votre luette. En cas de difficulté, essayez d'imaginer le son que quelqu'un produit lorsqu'il ronfle, qu'il se bloque la respiration vers l'arrière de la bouche, et qu'il essaie quand même d'expirer.}.


\textbf{| Prononciation de  \RL{ك} [k]}

Ce phonème correspond à la lettre arabe \RL{ك}. Il est l'un des phonèmes les plus répandus dans les langues internationales, et se prononce comme le \textbf{/k/} en français, comme dans les mots \textbf{camion} ou \textbf{kiwi}\footnote{Au sens strict du terme, \RL{ك} [k] en tunisien (et en arabe en général) se prononce de façon plus douce que le \textbf{/k/} en français ou en anglais : un arabophone expulsera moins d'air de ses poumons à sa prononciation. Si vous ne cherchez pas à avoir une prononciation identique à celle d'un tunisophone, vous pouvez ignorer cette note.}.


\textbf{| Prononciation de  \RL{ل} [l]}

Ce phonème correspond à la lettre arabe \RL{ل}. Il est l'un des phonèmes les plus répandus dans les langues internationales, et se prononce comme le \textbf{/l/} en français, comme dans les mots \textbf{lumière} ou \textbf{livre}.


\textbf{| Prononciation de  \RL{م} [m]}

Ce phonème correspond à la lettre arabe \RL{م}. Il est l'un des phonèmes les plus répandus dans les langues internationales, et se prononce comme le \textbf{/m/} en français, comme dans les mots \textbf{montre} ou \textbf{manteau}.


\textbf{| Prononciation de  \RL{ن} [n]}

Ce phonème correspond à la lettre arabe \RL{ن}. Il est l'un des phonèmes les plus répandus dans les langues internationales, et se prononce comme le \textbf{/n/} en français, comme dans les mots \textbf{notre} ou \textbf{niveau}.


\textbf{| Prononciation de \RL{ه} [\texthth]}

Ce phonème correspond à la lettre arabe \RL{ه}. Il correspond à ce qui est appelé \textbf{h expiré}, et apparaît dans des onomatopées et interjection en français, comme dans \textbf{hé !}. On rencontre ce son en anglais également, dans des mots comme \textbf{heavy} ou \textbf{hallelujah}.


\textbf{| Prononciation de \RL{و} [w]}

Ce phonème correspond à la lettre arabe \RL{و}. Il correspond au son \textbf{/w/} en français, comme dans les mots \textbf{wasabi} ou \textbf{web}.


\textbf{| Prononciation de \RL{ي} [y]}

Ce phonème correspond à la lettre arabe \RL{ي}. Il correspond au son \textbf{/y/} en français, comme dans les mots \textbf{yaourt} ou \textbf{youpi}.


\subsection{Consonne ayant évolué depuis l'arabe}

Il existe également \textbf{une} consonne qui a évolué depuis l'arabe depuis la  consonne \textbf{\RL{ق} [q]}.

Le phonème \textbf{[g]} apparaît dans plusieurs mots d'origine arabe, comme \textbf{digla (datte)} par exemple. Sa prononciation est similaire au son \textbf{/g/} qu'on peut retrouver en français dans les mots \textbf{garage} ou \textbf{gueule}. Dans plusieurs transcriptions, on pourra retrouver l'écriture \textbf{\RL{ڨ}}, mais elle ne fait pas l'unanimité puisqu'elle n'est pas originaire.

On retrouvera surtout cette consonne dans des dialectes tunisiens qui remplacent le son \textbf{[q]} par le son \textbf{[g]} (on parle de dialectes hilaliens). Dans la version "standardisée" du tunisien, cette consonne est utilisée pour prononcer les mots importés, même si la proportion de mots non importés l'utilisant est non négligeable. On retrouvera donc des mots assez modernes qui changent de sens en fonction de l'emploi de \textbf{[q]} ou \textbf{[g]}, la substitution de l'un par l'autre n'est donc pas nécessairement anodine. Par exemple, on pourra retrouver les mots \textbf{/qammer/ (parier)} et \textbf{/gammer/ (viser)}.

\subsection{Consonnes d'origine étrangère}

A l'ensemble des consonnes qui vous ont été présentées plus haut s'ajoutent deux autres consonnes, provenant toutes les deux de langues étrangères, probablement du \textbf{français} et de \textbf{l'italien}.

Ces deux consonnes servent uniquement à prononcer des mots importés : 
\begin{itemize}
    \item Le son \textbf{[p]}, comme dans les mots \textbf{port} ou \textbf{papa}.
    \item Le son \textbf{[v]}, comme dans les mots \textbf{valise} ou \textbf{voiture}.
\end{itemize}

On notera que deux écritures "arabisantes" existent, mais ne sont pas systématiquement reconnues comme orthographes officielles. Ainsi, pour \textbf{[p]}, on pourra voir \RL{پ}, alors que pour \textbf{[v]}, on pourra trouver \RL{ڥ}.

Il est intéressant de noter que ces deux consonnes peuvent être remplacées par le son le plus proche existant nativement en arabe, généralement lorsqu'il y a une difficulté de prononciation, ou par habitude. Ainsi, le \textbf{[p]} pourra se transformer en \textbf{[b]}, alors que \textbf{[v]} pourra se transformer en \textbf{[f]}.

\subsection{Système vocalique}\label{Système vocalique}
Le tunisien étant dérivé de l'arabe, le système vocalique, en tout cas dans la façon qu'on les tunisiens de se l'imaginer, tourne autour de \textbf{trois} voyelles uniquement. Passons les d'abord en revue :
\begin{itemize}
    \item La \textbf{fatha}, correspondant au son \textbf{[a-e]};
    \item La \textbf{dhamma}, correspondant au son \textbf{[u]};
    \item La \textbf{kasra}, correspondant au son \textbf{[i]}.
\end{itemize}

Cependant, les années passant, la réalisation de certaines voyelles a évolué. Les linguistes modernes ont du mal à s'accorder sur le nombre de voyelles que distingue le tunisien. C'est en réalité un exercice assez difficile dans la mesure où les voyelles sont réalisées très différemment en fonction de la région du locuteur, à l'instar du français\footnote{Le mot \textit{rose} ne se prononce pas pareil à Paris qu'à Toulouse.} et de l'anglais\footnote{Le RP English distingue beaucoup plus de diphtongues que le General American par exemple.}. 

Je vous propose ci-dessous d'en parcourir la quasi-totalité afin de vous rendre compte de l'étendue de l'inventaire phonétique tunisien. Certaines ne présentent que des différences mineures entre elles, auquel cas il n'est pas nécessaire de se forcer à les prononcer de manière exacte (un tunisophone ne fera lui-même que peu la distinction). Gardez à l'esprit que \textbf{grammaticalement}, le tunisien ne fait bien la différence qu'entre trois voyelles.
\begin{center}
\begin{tabular}{||c | c | c | c||} 
 \hline
 \textbf{\makecell{Transcription\\phonétique}} & \textbf{Transcriptions} & \textbf{Équivalent FR/EN} & \textbf{Tunisien}\\ [2.5ex] 
 \hline\hline
 [a]  & a & \underline{a}ller / g\underline{u}t & \RL{قَرْنْ}\\ 
 \hline
 [\ae]\texttildelow[\textepsilon]  & è & \underline{é}couter / b\underline{e}d & \RL{عْلاَشْ}\\
 \hline
 [\textsc{i}]  & é & m\underline{é}chant / b\underline{i}t & \RL{مَاتْ}\\  
 \hline
 [ \textschwa]  & e & kill\underline{e}r & \RL{ظَاهِرْ}\\ 
 \hline
 [i]  & i & r\underline{i}vière / m\underline{ee}t & \RL{فِيسَعْ}\\ 
 \hline
 [\textopeno]\texttildelow[\textupsilon]  & o & s\underline{o}rtir / c\underline{o}re & \RL{مُخْ}\\ 
 \hline
 [u]  & u & m\underline{ou}ton / d\underline{oo}m & \RL{مَاهُوشْ}\\ 
 \hline
\end{tabular}
\end{center}

Dans le tableau ci-dessous, vous aurez pu remarquer que certains sons sont relativement proches. Je me suis également permis de regrouper des sons qui, même si tous réalisés par des tunisophones, ne sont pas conscientisés comme étant des voyelles différentes. 

Historiquement, les voyelles ont évolué comme suit :
\begin{itemize}
    \item La \textbf{fatha} a évolué pour donner les voyelles [a], [\ae]\texttildelow[\textepsilon] et [\textsc{i}] ;
    \item La \textbf{dhamma} a évolué pour donner les voyelles [\textopeno]\texttildelow[\textupsilon] et [u] ;
    \item La \textbf{kasra} a évolué pour donner les voyelles [ \textschwa] et [i].
\end{itemize}

En pratique, l'utilisation de certaines voyelles dépend de l'environnement consonantique, et plus particulièrement de la consonne qui précède la voyelle d'intérêt. Ainsi, 

\begin{itemize}
    \item Les voyelles dérivées d'une \textbf{fatha} se prononcent [a] quand elles se situent juste avant ou juste après les phonèmes suivants : \RL{ز} [z], \RL{ر} [r], \RL{ق} [q], \RL{خ} [\textchi], \RL{غ} [\textinvscr], \RL{ه} [\texthth], \RL{ع} [\textrevglotstop], \RL{ح} [h], \RL{ض} [\dh \super\textrevglotstop], \RL{ط} [t\super\textrevglotstop] et \RL{ص} [s\super\textrevglotstop] ;
    \item Les voyelles dérivées d'une \textbf{fatha} se prononcent [\ae]\texttildelow[\textepsilon] autour du  phonème \RL{ل} [l];
    \item Dans les autres cas, ces voyelles là se prononcent [\textsc{i}].
\end{itemize}

\textbf{\textsc{Note} :} On pourra parler en quelque sorte d'une harmonie vocalique.

Il faut également noter que le tunisien fait la distinction entre les voyelles courtes et les voyelles longues, comme en arabe standard. Voyelles courtes et voyelles longues, en fonction de leur environnement consonantique, ont une sémantique différente. Nous aborderons dans la suite du cours ces différences-là.

Il faudra également parler des \textbf{voyelles nasales}. La Tunisie et les tunisiens sont restés en contact assez longtemps avec la langue française pour que plusieurs mots passent d'une langue à l'autre, le sens nous intéressant en l'occurrence étant du français vers le tunisien. Ces mots importés ont réussi à imposer avec eux l'import de trois voyelles nasales : 

\begin{center}
\begin{tabular}{||c c c||} 
 \hline
 Transcription en français & Alphabet phonétique & Exemple en français\\ [2.5ex] 
 \hline\hline
 \textbf{an}  & \~\textscripta & Pl\underline{an} \\
 \textbf{on}  & \~\textopeno & C\underline{om}pas \\
 \textbf{in}  & \~\textepsilon & \underline{In}ternet\\ 
 \hline
\end{tabular}
\end{center}

Vous avez dû remarquer qu'on prononce une quatrième voyelle nasale en français, couramment notée \textbf{/en/} [\~\oe] comme dans \textbf{r\underline{en}dez-vous}. Cette voyelle nasale est remplacée dans toutes les occurrences des mots importés par le \textbf{/an/}, en tout cas chez la majorité des tunisiens. Cependant, il vous est tout à fait envisageable de la prononcer comme bon vous semble, \textbf{/an/} ou \textbf{/en/}.

Il faut également savoir que la plupart de ces voyelles nasales ne sont pas complètement nasalisées, c'est-à-dire que vous pourrez souvent entendre sur la fin de la voyelle quelque chose qui ressemble à un \textbf{/n/}\footnote{C'est le même phénomène qu'on peut retrouver dans l'accent du sud-ouest de la France, où on assiste à la dénasalisation des voyelles nasales}. 

\section{Transcription utilisée dans ce cours}

Maintenant que nous avons vu l'ensemble des sons qui sont réalisables en  tunisien, je vous propose d'établir ensemble une transcription, c'est-à-dire le système de substitution des sons par des lettres (autrement dit, le système d'écriture).

Je nous fixe quelques règles pour cette transcription, en espérant que nous puissions les respecter le plus possible :
\begin{itemize}
    \item Chaque son devra être représenté par un seul et unique symbole ou combinaison de symboles ;
    \item On doit limiter le nombre de son produits par une combinaison de plusieurs symboles (comme en français où \textbf{/o/} et \textbf{/i/} s'associent pour faire le son \textbf{[wa]}) ;
    \item On s'autorise à regrouper des sons qui seront représentés par le même symboles, du moment que le contexte phonétique permette de déduire le son que représente le symbole sans ambiguïté.
\end{itemize}

\subsection{Transcription des consonnes et des voyelles}
Après quelques itérations, j'en arrive au système suivant, qui a ses défauts et ses avantages, comme tout système. 

\begin{center}
\begin{tabular}{||c c c||} 
 \hline
 \textbf{\makecell{Transcription\\langue d'origine}} & \textbf{\makecell{Transcription\\phonétique}} & \textbf{\makecell{Transcription\\retenue}}\\ [2.5ex] 
 \hline\hline
 \RL{ا} / \RL{ء} & ['] & ' / $\emptyset$ \\ 
 \hline
 \RL{ب} & [b] & b \\ 
 \hline
 \RL{ت} & [t] & t \\ 
 \hline
 \RL{ث} & [\texttheta] & \th \\ 
 \hline
 \RL{ج} & [\textyogh] & j \\ 
 \hline
 \RL{ح} & [h] & \textcrh \\ 
 \hline
 \RL{خ} & [\textchi] & x \\ 
 \hline
 \RL{د} & [d] & d \\ 
 \hline
 \RL{ذ} & [\dh] & \dh \\ 
 \hline
 \RL{ر} & [r] & r \\ 
 \hline
 \RL{ز} & [z] & z \\ 
 \hline
 \RL{س} & [s] & s \\ 
 \hline
 \RL{ش} & [\textesh] & \v{s} \\ 
 \hline
 \RL{ص} & [s\super \textrevglotstop] & \c{s} \\ 
 \hline
 \RL{ض}/\RL{ظ} & [\dh \super \textrevglotstop] & \c{\dh} \\ 
 \hline
 \RL{ط} & [t\super \textrevglotstop] & \c{t} \\ 
 \hline
 \RL{ع} & [\textrevglotstop] & \c{a} \\ 
 \hline
 \RL{غ} & [\textinvscr] & \v{r} \\ 
 \hline
 \RL{ف} & [f] & f \\ 
 \hline
 \RL{ق} & [q] & q \\ 
 \hline
 \RL{ك} & [k] & k \\ 
 \hline
 \RL{ل} & [l] & l \\ 
 \hline
 \RL{م} & [m] & m \\ 
 \hline
 \RL{ن} & [n] & n \\ 
 \hline
 \RL{ه} & [\texthth] & h \\ 
 \hline
\end{tabular}
\end{center}


\begin{center}
 \begin{tabular}{||c c c||} 
 \hline
 \textbf{\makecell{Transcription\\langue d'origine}} & \textbf{\makecell{Transcription\\phonétique}} & \textbf{\makecell{Transcription\\retenue}}\\ [2.5ex] 
 \hline\hline
 \RL{و} & [w] & w \\ 
 \hline
 \RL{ي} & [y] & y \\ [2.5ex] 
 \hline
 \RL{ڨ} & [g] & g \\  
 \hline
 \RL{پ} & [p] & p \\  
 \hline
 \RL{ڥ} & [v] & v \\ 
 \hline
 $\emptyset$ & [a]  & a \\ 
 \hline
 $\emptyset$ & [\ae]\texttildelow[\textepsilon]  & è \\
 \hline
 $\emptyset$ & [\textsc{i}]  & é \\  
 \hline
 $\emptyset$ &[ \textschwa]  & e \\ 
 \hline
 $\emptyset$ &[i]  & i \\ 
 \hline
 $\emptyset$ &[\textopeno]\texttildelow[\textupsilon]  & o \\ 
 \hline
 $\emptyset$ &[u]  & u \\
 \hline
 \textbf{an} &[\~\textscripta]  & \v{a} \\
 \hline
 \textbf{on} &[\~\textopeno]  & \v{o} \\
 \hline
 \textbf{in} &[\~\textepsilon]  & \v{i} \\ [2.5ex] 
 \hline
\end{tabular}
\end{center}

Quelques commentaires sur cette transcription.

\begin{itemize}
    \item J'ai décidé de laisser la possibilité d'omettre le symbole pour le \textbf{coup de glotte [']}. Ce choix est motivé par le fait que les tunisophones ont de plus en plus tendance à l'omettre, que ce soit au début ou à la fin des mots. Je laisse donc la possibilité de le rajouter au milieu des mots, en lui affectant un symbole \footnote{En pratique, je vais essayer de le noter tant que faire se peut au cours de ce cours, notamment en début de mot, pour des raisons \textbf{grammaticale} et \textbf{étymologique}.};
    \item Tous les symboles des \textbf{consonnes emphatiques} portent une \textbf{cédille}. J'ai choisi ce système afin d'aider à la prononciation, et aider si besoin pour l'harmonie consonantique;
    \item La consonne [\textrevglotstop] se note aussi avec une \textbf{cédille}, en se servant d'un \textbf{/a/} comme support. J'ai préféré cette notation pour ne pas induire en erreur en proposant une lettre support qui était sans rapport. L'alphabet maltais a par exemple fait le choix de se servir d'un /g/ comme support;
    \item J'ai emprunté deux lettres à l'alphabet du moyen anglais : \textsc{thorn} "\textbf{\th}" et \textsc{eth} "\textbf{\dh}". Elles servaient à l'époque à marquer les mêmes sons, mais les symboles ont disparu avec l'importation de l'imprimerie depuis la France;
    \item J'ai marqué d'un diacritique les symboles pour \textbf{[\textesh]} et \textbf{[\textinvscr]}. On pourrait envisager deux symboles pour marquer ces sons-là, comme par exemple /sh/ et /gh/, mais il me semblait que ça porterait à confusion avec le son \textbf{[h]};
    \item J'ai mis la même diacritique pour les \textbf{voyelles nasales} : ce qui motive ce choix est que cette diacritique sert à noter une prononciation différente mais proche de la lettre support;
    \item Pour les autres \textbf{voyelles}, j'ai choisi un système qui soit intuitif pour les locuteurs francophones : la transcription correspond à peu de choses près à la prononciation qu'on aurait en français. La seule exception que je me suis autorisée est pour le son \textbf{[u]}, qui n'est marqué que d'un symbole plutôt que de deux en français (/ou/).
\end{itemize}

\subsection{Voyelles longues et consonnes géminées}
En plus de la retranscription des sons, il faut parler du cas des voyelles longues et des consonnes géminées (les consonnes doublées).

Le tunisien, comme l'arabe, fait une distinction sémantique entre :
\begin{itemize}
    \item \textbf{Voyelles courtes et longues} : La longueur d'une voyelle change le sens d'un mot, par exemple sa fonction grammaticale comme dans \textbf{[mut] (meurs, verbe à l'impératif)} et \textbf{[mu:t] (la mort)}.
    \item \textbf{Consonnes simples et consonnes géminées (doublées)} : Le doublage des consonnes en tunisien change également le sens d'un mot, par exemple \textbf{[ba\textrevglotstop \textschwa d] (après)} et \textbf{[ba\textrevglotstop\textrevglotstop \textschwa d] (éloigne, verbe à l'impératif)}.
\end{itemize}

Pour continuer de faire cette distinction à l'écrit, je propose dans la suite de \textbf{doubler} les symboles qui représentent les voyelles ou les consonnes longues. Ainsi, en reprenant les exemples précédents : 

\begin{itemize}
    \item \textbf{[mut]} $\rightarrow$ \textbf{mut}
    \item \textbf{[mu:t]} $\rightarrow$ \textbf{muut}
    \item \textbf{[ba\textrevglotstop \textschwa d]} $\rightarrow$ \textbf{ba\c{a}ed}
    \item \textbf{[ba\textrevglotstop\textrevglotstop \textschwa d]} $\rightarrow$ \textbf{ba\c{a}\c{a}ed}
\end{itemize}

\section{Et maintenant ? }
Avec tout ceci en poche, je crois que nous avons l'ensemble des outils nécessaires pour commencer à apprendre le tunisien dans de bonnes conditions. C'est parti !

\chapter{Phénomènes phonétiques en tunisien}
\chapterletter{A}vant d'aborder votre premier point de grammaire, il me paraît essentiel de parler de certains phénomènes phonétiques que vous pourrez rencontrer, qui affecteront a minima la prononciation, et jusqu'à la justification d'une retranscription différente de certaines formes grammaticales.

\section{Décomposition des mots en syllabes}
Pour la prononciation des mots que vous allez rencontrer, il est important de savoir les découper en syllabes. Pour cela, la linguistique s'intéresse souvent à la \textbf{structure syllabique} d'une langue, c'est-à-dire quels sont les successions autorisées (et interdites donc) de sons. Par la suite, c'est bien la structure des syllabes et leur agencement entre elles qui définissent comme se prononce un mot. Au-delà des sons présents dans une langue, c'est bien la structure syllabique et la structure des mots qui donne son \textit{feeling} à une langue.

Ainsi, en \textbf{français} par exemple, on pourra retrouver les syllabes \textbf{/ba/}, \textbf{/bar/} et \textbf{/bwar/}, mais on ne pourra jamais retrouver une syllabe du style \textbf{/mlorj/}, même si on n'aurait pas spécialement de difficulté à la prononcer.

Rassurez-vous, il ne s'agit pas dans cette section de disséquer totalement les structures des syllabes en tunisien, simplement d'aborder des points importants qui vous permettront de lire un mot \textit{de la bonne façon}. 

Pour ce faire, voici les points importants à retenir : 
\begin{itemize}
    \item Le tunisien, comme l'arabe, fait commencer \textbf{toutes} ses syllabes par \textbf{une consonne}\footnote{Dans le cas des mots qui \textit{semblent} commencer par une voyelle, un \textbf{coup de glotte /'/} est prononcé en début de syllabe.}.
    \item Les \textbf{consonnes doubles} ne sont \textbf{jamais} dans la même syllabe, sauf si c'est la \underline{dernière} syllabe du mot.
    \item De façon plus générale, le tunisien hait les clusters de consonnes, et aura tendance tant que faire se peut de \textbf{séparer} les consonnes proches pour les mettre dans des syllabes différentes.
    \item Vous ne trouverez jamais \textbf{trois} consonnes successives, ni \textbf{deux} voyelles successives.
\end{itemize}

De ces quelques règles-ci, voici un petit algorithme qui vous aidera dans la majorité des cas :
\begin{itemize}
    \item Si une consonne est encadrée par deux voyelles, alors cette consonne marque le début d'une nouvelle syllabe ;
    \item Si une consonne est répétée dans une syllabe non-terminale, alors la consonne répétée marque le début d'une nouvelle syllabe ;
    \item Si deux consonnes sont encadrées par deux voyelles, alors la deuxième consonne marque le début d'une nouvelle syllabe ;
    \item Si trois consonnes se suivent, alors les deuxième et troisième consonnes démarrent une nouvelle syllabe avec la voyelle qui les suit.
\end{itemize}

Voici quelques exemples pour vous faire la main :

\begin{center}
    \begin{tabular}{||c | c | c||} 
    \hline
    \textbf{Tunisien} & \textbf{Syllabes} & \textbf{Trad.}\\
    \hline\hline
    net\c{a}allmuu & net | \c{a}al | lmuu & \textit{nous apprenons}\\ 
    \hline
    tetekteb & te | tek | teb & \textit{elle est écrite}\\ 
    \hline
    ordinater & or | di | na | ter & \textit{ordinateur}\\ 
    \hline
    xzééna & xzéé | na & \textit{armoire}\\ 
    \hline
    xobz & xobz & \textit{du pain}\\ 
    \hline
    xobza & xob | za & \textit{une baguette}\\ 
    \hline
   \end{tabular}
\end{center}

\section{Position de l'accent tonique}
XXX

\section{Assimilation des consonnes}\label{Assimilation}
Parler de \textbf{\c{a}h > \textcrh\textcrh} et les autres assimilations qu'ils peut y avoir.

\section{Métathèse et simplification vocalique}
XXX

\section{Harmonie consonantique}
Pour être totalement franc, vous pouvez ignorer ce passage, sauf si vous êtes curieux ou que vous voulez apprendre à avoir un accent parfait en tunisien. 

Le tunisien présente une caractéristique assez particulière, qu'il partage avec d'autres langues d'origine arabe, qui est \textbf{l'harmonie consonantique}, plus particulièrement une \textbf{harmonie d'articulation secondaire}. Concrètement, cela veut dire que les \textbf{consonnes emphatiques}, qui sont pharyngalisées, ont tendance à "pharyngaliser" les consonnes qui sont autour. 

Par exemple, un mot comme \textbf{[mba\textrevglotstop \textschwa d] (après)} sera plutôt prononcé \textbf{[mb\super\textrevglotstop a\textrevglotstop \textschwa d\super\textrevglotstop]}. L'harmonie ne s'étend généralement qu'aux deux consonnes les plus proches d'un même mot, et s'étendent en quelques occasions à deux consonnes.

Ce détail n'a pas d'incidence sur la sémantique, puisque, comme l'exemple ci-dessus le montre, les sons qui sont produits (\textbf{[b\super\textrevglotstop]} et \textbf{[d\super\textrevglotstop]} ici) ne sont pas des phonèmes qui existent isolés de tout autre contexte consonantique. Il permettra par contre à votre accent d'être plus naturel.

\chapter{Introduction aux phrases nominales}
\chapterletter{D}ans ce chapitre, nous allons parler de la construction des phrases nominales en tunisien, qui occupe une plus grande place dans la grammaire que les phrases nominales dans les langues latines et germaniques. 

\section{Un peu d'histoire}
La structure des phrases en tunisien est directement héritée de l'arabe standard. Chez la langue-mère du tunisien, \textbf{deux types} de structures sont possibles :
\begin{itemize}
    \item Les phrases \textbf{verbales} : elles comprennent sujet, verbe et compléments éventuels;
    \item Les phrases \textbf{nominales} : elles se caractérisent par l'absence de verbe, mais cela ne veut pas dire qu'elles sont pauvre en sens.
\end{itemize}

Une des caractéristiques qui marquent souvent le plus chez les nouveaux apprenant de l'arabe est l'absence de verbes qui peuvent paraître essentiels dans d'autres langues, notamment les verbes \textbf{être} et \textbf{avoir}. L'arabe, et le tunisien au même titre, se permettent d'échapper à cette nécessité en se reposant sur une structure grammaticale plus rigide. 

Ainsi, vous l'aurez compris, \textbf{les phrases nominales} servent dans tous les langues arabes à former des phrases \textbf{descriptives} et des \textbf{constats}, puisqu'il manquera systématiquement un verbe décrivant l'action. Quelques adverbes permettent néanmoins de véhiculer un sens plus nuancé, mais cela ne relève pas de ce cours. 

\section{Structure des phrases nominales}
Une phrase nominale en tunisien est constituée de \textbf{deux} constituants principaux :
\begin{itemize}
    \item \textbf{Le sujet} : C'est l'élément sur lequel une information sera donnée;
    \item \textbf{Un complément} : C'est l'information qui est donnée sur le sujet. En arabe, on l(appelle "l'information".
\end{itemize}

Pour former la phrase, \textbf{il suffit de juxtaposer le sujet et le complément}.

\textbf{Exemples :}
\begin{center}
 \begin{tabular}{||c | c | c||} 
 \hline
 Tunisien & Français & Traduction littérale \\ [2.5ex] 
 \hline\hline
 Esmi Mo\c{s}\c{t}fa &\textit{Mon prénom est Mostafa}  & \textit{Prénom-mon Mostafa} \\ 
 \hline
 Es-smèè' zarqa &\textit{Le ciel est bleu}  & \textit{Le ciel bleu} \\ 
 \hline
 E\c{t}-\c{t}aq\c{s} sxuun &\textit{Il fait chaud}  & \textit{Le temps chaud} \\ 
 \hline
 E\c{t}-\c{t}aawla \c{a}aalya &\textit{La table est haute}  & \textit{La table haute} \\ 
 \hline
 Enti bnayya &\textit{Tu es une fille}  & \textit{Toi fille} \\ 
 \hline
 Huwwa raajel &\textit{Il est un homme}  & \textit{Lui homme} \\ 
 \hline
\end{tabular}
\end{center}

Si vous avez l'\oe il affûté, vous aurez déjà remarqué un point commun dans toutes ses phrases : \textbf{dans toutes les phrases nominales, le sujet est nécessairement sous une forme définie}. On dira bien : \underline{le} temps, \underline{la} table, \underline{mon} prénom, \underline{toi}, etc. Et à l'inverse, \textbf{le complément sera nécessairement sous la forme indéfinie}. On peut donc facilement faire la distinction entre chaque groupe grammatical.

Vous aurez aussi remarqué que le complément peut être de différentes natures : nom commun, adjectif ou encore nom propre.

\section{Pronoms personnels}
Puisqu'ils peuvent vous aider à former vos propres phrases, parlons rapidement des \textbf{pronoms personnels}. Ils prennent tous racine dans l'arabe standard.

\begin{center}
    \begin{tabular}{||c | c | c||}
    \hline
        \textbf{Tunisien} & \textbf{Français} & \textbf{Arabe standard} \\ [2.5ex] 
        \hline\hline
        \je & Je / Moi & \RL{انا}\\ \hline
        \tu & Tu / Toi & \RL{انت}\\ \hline
        \il & Il / Lui & \RL{هو}\\ \hline
        \elle & Elle & \RL{هي}\\ \hline
        \nous & Nous & \RL{نحن}\\ \hline
        \vous& Vous & \RL{انتم}\\ \hline
        \ils & Ils / Elles / Eux & \RL{هم}\\ \hline
    \end{tabular}
\end{center}

Ils ont beaucoup évolué depuis l'arabe standard, le tunisien perdant en tout plus de \textbf{6} pronoms personnels :
\begin{itemize}
    \item Toutes les \textbf{formes duelles} (3 au total) :  ce sont les pronoms qui qualifient exactement deux personnes (il y en avait une pour la deuxième personne, et deux pour la troisième personne).
    \item Tous les \textbf{pluriels féminins} (2 au total) : ces pronoms correspondent aux groupes constitués entièrement de sujets féminins, à la deuxième et troisième personne.
    \item Le pronom \textbf{féminin singulier de la deuxième personne} (1 au total) : cela correspond au pronom "tu" accordé au féminin. Il est quand même intéressant de noter que certains dialectes du tunisien ont conservé ce pronom (ce cours ne couvrira pas ces formes-là).
\end{itemize}

\section{Quelques variations}
Il est tout à fait possible d'agrémenter le sujet avec d'autres adjectifs ou des démonstratifs par exemple. Dans ce cas-là, le complément sera toujours composé d'un \textbf{mot unique}, et tout le reste des mots formeront le \textbf{groupe sujet}.

\textbf{Exemples :}


\begin{center}
    \begin{tabular}{||c | c | c||}
        \hline
        \textbf{Tunisien} & \textbf{Français} & \textbf{\makecell{Traduction \\littérale}} \\ [2.5ex] 
        \hline\hline
        \makecell{Héé\dh i el-karehba\\ \textcrh amra.} &\textit{\makecell{Cette voiture-ci \\est rouge.}}  & \textit{\makecell{Cette la voiture \\rouge.}} \\ 
        \hline
        \makecell{Héé\dh a er-raajel\\ \c{t}wiil.} &\textit{\makecell{Cet homme \\est grand.}}  & \textit{\makecell{Ce le homme\\ grand.}} \\ 
        \hline
        \makecell{El-kosksi et-tuunsi\\ bniin.} &\textit{\makecell{Le couscous tunisien\\ est bon.}}  & \textit{\makecell{Le couscous le \\tunisien bon.}} \\ 
        \hline
        \makecell{Héé\dh a l-ordinater e\c{s}-\c{s}\v{r}iir \\w el-\v{r}aali ak\textcrh al.} &\textit{\makecell{Ce petit ordinateur \\cher est noir.}}  & \textit{\makecell{Ce l'ordinateur petit \\et cher noir.}} \\ 
        \hline
    \end{tabular}
\end{center}

Dans les exemples ci-dessus, le complément est systématiquement le dernier mot de la phrase et \textbf{sous forme indéfinie}. Tout le reste des mots forme le sujet. Sémantiquement, cela suppose que toutes les informations \textbf{en dehors du complément} sont soit déjà connues, soit moins importantes.

\begin{minipage}{10cm}

\section*{Vocabulaire}
\begin{center}
    \begin{tabular}{||c | c | c||}
        \hline
        Vocabulaire & Traduction & Origine \\\hline\hline
        'esm (masc.) & prénom / nom & (\textsc{ar}) \RL{اسم} \\\hline
        smèè' (fem.) & ciel & (\textsc{ar}) \RL{سماء} \\\hline
        azraq / zarqa (adj.) & bleu & (\textsc{ar}) \RL{أزرق} \\\hline
        \c{t}aq\c{s} (masc.) & temps/météo & (\textsc{ar}) \RL{طقس} \\\hline
        sxuun / sxuuna (adj.) & chaud & (\textsc{ar}) \RL{ساخن} \\\hline
        \c{t}aawla (fem.) & table & (\textsc{ar}) \RL{طاولة} \\\hline
        \c{a}aali / \c{a}aalya (adj.) & haut & (\textsc{ar}) \RL{عالي} \\\hline
        bnayya (fem.) & fille & (\textsc{ar}) \RL{ابن} (fils) \\\hline
        raajel (masc.) & homme & (\textsc{ar}) \RL{راجل} \\\hline
        héé\dh a / héé\dh i (démons.) & celui-ci / celle-ci & (\textsc{ar}) \RL{هذا / هذه} \\\hline
        a\textcrh mar / \textcrh amra (adj.) & rouge & (\textsc{ar}) \RL{أحمر} \\\hline
        karehba (fem.) & voiture & (\textsc{ar}) \RL{كهرباء} (électricité) \\\hline
        \c{t}wiil / \c{t}wiila (adj.) & grand (en taille) & (\textsc{ar}) \RL{طويل / طويلة} \\\hline
        tuunsii / tuunsiyya (adj.) & tunisien/tunisienne & (\textsc{ar}) \RL{تونسي / تونسية} \\\hline
        bniin / bniina (adj.) & bon/bonne (en goût) & --- \\\hline
        ordinater (masc.) & ordinateur & (\textsc{fr}) ordinateur \\\hline
        \c{s}\v{r}iir / \c{s}\v{r}iira (adj.) & petit / petite & \textsc{ar} \RL{صغير / صغيرة} \\\hline
        \v{r}aali / \v{r}aalya (adj.) & cher / chère (en prix) & (\textsc{ar}) \RL{غالي / غالية} \\\hline
        ak\textcrh al / ka\textcrh laa (adj.) & noire / noire & (\textsc{ar}) \R{اكحل} \\\hline
    \end{tabular}
\end{center}

\end{minipage}



\h{Introduction aux phrases verbales}
\l{I}ntéressons nous maintenant aux phrases verbales, qui constituent avec les phrases nominales, les deux structures possibles pour une phrase en tunisien.

\hh{Un peu d'histoire}
Tout comme les phrases nominales, les phrases verbales prennent leurs racines dans l'arabe standard. Cette structure correspond à la structure d'une phrase classique qu'on pourrait retrouver dans les autres langues internationales.

Comme son nom l'indique, cette structure comprend nécessairement un \textbf{verbe}, qui servira à décrire une action. Ainsi, cette structure s'oppose directement à la structure de la phrase nominale, qui est elle nécessairement descriptive.

En \textbf{arabe standard}, la structure d'une phrase verbale est \textbf{différent} de celle du \textbf{tunisien}. En effet, l'arabe standard, comme environ 10\% des langues dans le monde, est une \textbf{VSO}, c'est-à-dire qu'elle obéit à la structure \textbf{verbe-sujet-objet}.

\textbf{\textsc{Note} :} Contrairement à ce que pensent certains, ce n'est pas la structure \textbf{SVO (sujet-verbe-objet, 42\%)} qui est la plus répandue, mais bien la structure \textbf{SOV (sujet-objet-verbe, 45\%)} !

Vous verrez dans la suite de ce cours que cette structure historique \textbf{VSO} garde encore des traces jusqu'à maintenant en tunisien. Il est donc utile de la garder en tête.

\hh{Structure des phrases verbales}
Contrairement à l'arabe standard, le \textbf{tunisien} arbore une structure \textbf{SVO (sujet-verbe-objet)}, qui est donc la même structure que le langues latines, ou l'anglais.

\begin{table}[h]
\begin{tabularx}{\textwidth}{||X | X | X||}
 \hline
 Tunisien & Français & Traduction littérale \\ [2.5ex] 
 \hline\hline
 Lemra tékel xobz  & \textit{La femme mange du pain} & \textit{La femme mange pain}\\ 
 \hline
 Leq\c{t}aate\c{s} i\textcrh ebbuu el\textcrh uut  & \textit{Les chats aiment le poisson} & \textit{Les chats aiment le poisson}\\ 
 \hline
 Enti taqraa ktééb  & \textit{Tu lis un livre} & \textit{Tu lis livre}\\ 
 \hline
\end{tabularx}
\end{table}


\hh{Omission des pronoms personnels}
Il n'y pas grand chose à retenir de plus sur la structure des phrases verbales. Cependant, je dirais quand même quelques mots sur \textbf{l'omission des pronoms personnels}.

Comme c'est le cas de l'italien, l'espagnol et de l'arabe standard, le tunisien s'autorise à \textbf{omettre les pronoms personnels dans les phrases verbales}. Dans ce cas, c'est la \textbf{conjugaison du verbe} qui indique quelle personne est visée.

Cette habitude a directement été héritée de l'arabe. A l'instar de l'espagnol et de l'italien que j'ai cités plus haut, \textbf{l'arabe standard distingue précisément les conjugaisons} de chaque personne à tous les temps. \textbf{Ce n'est cependant pas le cas du tunisien}, ou le temps a suffisamment érodé la prononciation de certaines formes conjuguées pour que certaines d'entre elles soient maintenant indistinguable.

Fort heureusement, le contexte permet de retrouver quasiment systématiquement la personne à laquelle la phrase se réfère. En cas de doute, les tunisophones n'hésitent alors pas à préciser le pronom personnel en question au lieu de l'omettre et risquer la confusion.

\begin{table}[h]
\begin{tabularx}{\textwidth}{||X | X | X||}
 \hline
 Tunisien & Français & Traduction littérale \\ [2.5ex] 
 \hline\hline
 N\textcrh ebb elkuura & \textit{J'aime le foot} & \textit{(J') Aime le ballon}\\ 
 \hline
 To\v{s}orbuu & \textit{Vous buvez} & \textit{Buvez}\\ 
 \hline
 Taqraa majalla  & \textit{Tu lis / Elle lit un magazine} & \textit{lis / lit magazine}\\ 
 \hline
\end{tabularx}
\end{table}

\hh*{Vocabulaire}
\begin{table}[h]
\begin{tabularx}{\textwidth}{||X | X | X||}
 \hline
 Vocabulaire & Traduction & Origine \\
 \hline\hline
 mra (fem.) / nsé' (plu.) & femme & (\textsc{ar}) \RL{امراة / نساء} \\
 \hline
 yéékel (verbe) & manger & (\textsc{ar}) \RL{أكل} \\
 \hline
 xobz (masc.) & pain & (\textsc{ar}) \RL{خبز} \\
 \hline
 qa\c{t}\c{t}uu\c{s} (masc.) / q\c{t}aate\c{s} (plu.) & chat & (\textsc{ar}) \RL{قطّ} \\
 \hline
 i\textcrh ebb (verbe) & aimer/vouloir & (\textsc{ar}) \RL{حبّ} \\
 \hline
 \textcrh uut (masc.) & poisson & (\textsc{ar}) \RL{حوت} (baleine) \\
 \hline
 yaqraa (verbe) & lire / étudier & (\textsc{ar}) \RL{قرأ} \\
 \hline
\end{tabularx}
\end{table}

\begin{table}[h]
\begin{tabularx}{\textwidth}{||X | X | X||}
 \hline
 Vocabulaire & Traduction & Origine \\
 \hline\hline
 ktééb (masc.) & livre & (\textsc{ar}) \RL{كتاب} \\
 \hline
\end{tabularx}
\end{table}

\chapter{Conjugaison des verbes sains simples}
\label{ConjSS}
\chapterletter{P}assons maintenant à la conjugaison des verbes réguliers simples en tunisien, qui forment une bonne partie des verbes qui vous pourrez rencontrer.

Je vous propose de passer un peu de temps sur les verbes en arabe standard, et comment le tunisien a évolué à partir de cette langue. Cette partie n'est pas essentielle pour apprendre à parler tunisien, mais comprendre la façon dont tout s'articule en arabe standard et comment ces particularités se sont transmises au tunisien vous aidera à comprendre la logique sous-jacente de la conjugaison tunisienne.

\section{Un peu d'histoire} \label{ConjSS1}
Commençons simplement par remettre dans le contexte la conjugaison et les verbes du tunisien, et faire le pont avec ce qu'on peut retrouver dans l'arabe standard.

Ce qui me semble être un gros avantage dans l'apprentissage de l'arabe standard, une fois la barrière de l'alphabet passé, est que la conjugaison est relativement simple. On rencontre certes 13 personnes différentes, mais l'arabe comporte très peu de temps : \textbf{le passé}, \textbf{le présent} (à partir duquel on déduit d'autre temps en ajoutant des préfixes ou en changeant les voyelles finales), et \textbf{l'impératif}. En plus de cela, on ajoute une conjugaison extrêmement systématique et régulière, qui fait qu'on ne retrouvera pas de listes de verbes irréguliers à apprendre par c\oe ur. 

L'arabe standard se construit autour de ce qu'on appelle des \textbf{racines sémitiques}. Ces racines sont des \textbf{triplets de consonnes} (moins couramment des quadruplets), qui portent toute une famille de sens, et dont dérivent la quasi-totalité des mots de la langue par l'application de \textbf{schèmes} (des structures de voyelles et de consonnes qui peuvent être rajoutées pour changer le sens de la racine). Nous en reparlerons plus tard dans ce cours, car le tunisien se construit également autour de ces racines et de ces schèmes.

Je vous parle de cette particularité pour trois raisons principales :
\begin{itemize}
    \item Premièrement, pour qu'on puisse définir ce qu'est un \textbf{verbe simple}, qui se résume simplement en la phrase suivante : "\textbf{Un verbe est simple si toutes les consonnes qui le composent sont les consonnes de sa racine}". Ainsi, dès que vous verrez un verbe composé d'exactement trois consonnes, vous saurez qu'il est simple.
    \item Deuxièmement, pour vous sensibilisez sur le fait que \textbf{la conjugaison des verbe s'appuiera sur des schèmes}. Ainsi, même si des verbes peuvent se ressembler dans une certaine forme particulière (typiquement, conjugués au passé à la 3ème personne masculin du singulier), le reste de la conjugaison peut varier.
    \item Finalement, parce que les linguistes étudiant l'arabe standard font la distinction entre plusieurs \textbf{groupes de verbes}, et que l'appartenance à ces groupes se déduit par l'agencement des consonnes au sein de la racine. 
\end{itemize}


\section{Évolutions en tunisien}

Fort heureusement, l'arabe tunisien garde quasiment toute la régularité de sa langue mère. Certaines simplifications sont apparues avec le temps, notamment au niveau des pronoms comme nous l'avons déjà vu. Mais comme dans toutes les langues du monde, qui dit simplifications dit également apparition d'irrégularités.

Je vous propose de lister maintenant les points communs et les évolutions qui sont apparues dans le tunisien, tant au niveau de la grammaire qu'au niveau de la prononciation.

\begin{itemize}
    \item Comme dit plus haut, le tunisien ne comporte que \textbf{7 pronoms}, au lieu des 13 de l'arabe standard;
    \item Le tunisien \textbf{a perdu l'ensemble des temps dérivés} du présent en arabe standard, dont le futur, au profit de l'ajout de verbes semi-auxiliaires (nous en reparlerons);
    \item Beaucoup de voyelles se sont perdues dans le temps, ce qui donnera \linebreak l'impression au tunisien l'impression d'avoir une conjugaison beaucoup plus compacte.
    \item Le tunisien a gardé l'ensemble des \textbf{groupes des verbes} des l'arabe standard;
    \item L'emploi de certaines voyelles en tunisien doit être fait avec plus de précision qu'en arabe standard, même s'il existe beaucoup d'allophonie (des sons que les locuteurs assimilent au même phonème).
\end{itemize}

Étant donnée la nature vivante d'une langue, il est possible qu'en parlant à certains locuteurs tunisiens, vous vous rendiez compte que leur prononciation sera différente de ce que je vais vous présenter dans ce chapitre. Comme toutes les langues, cela ne voudra pas dire que ce que vous appris ou que ce locuteur a prononcé est faux, simplement qu'il a plusieurs accents et plusieurs manières de réaliser certains points de grammaire. 

Ce dernier point est particulièrement important : lors que j'ai fait mes premières recherches sur la conjugaison des verbes, j'ai remarqué qu'il était possible de substituer certaines voyelles à d'autres, voire même d'en omettre certaines et de fusionner des syllabes, et ce chez plusieurs locuteurs qui ont a priori le même accent. 

Ainsi, je vous propose dans la suite de ce chapitre de vous enseigner une conjugaison qui soit facile à retenir, même si la prononciation n'est pas la prononciation majoritaire, vu qu'il n'existe pas de version standardisée du tunisien. Cela facilitera votre apprentissage tout en vous permettant d'être intelligible.

\section{Qu'est-ce qu'un verbe sain simple ?}
On appelle verbe sain simple un verbe qui est à la fois : 
\begin{itemize}
    \item \textbf{Simple :} Comme évoqué plus haut, il s'agit de l'ensemble des verbes dont toutes les consonnes sont exactement celles de sa racine. À quelques exceptions près, vous pouvez admettre que un verbe est sain s'il comportement exactement trois consonnes.
    \item \textbf{Sain :} Il s'agit d'un verbe dont toutes les consonnes de sa racine sont saines.
\end{itemize}

Attardons-nous quelques instants sur la dénomination \textbf{consonne saine}. En arabe standard, et en tunisien, on fait la distinction, au niveau de la conjugaison, entre les lettres saines et les lettres défectueuses. Ces dernières forment l'ensemble des lettres dont la présence dans une racine change la façon de conjuguer le verbe. 

On compte au total quatre marques de la défectuosité d'une racine : \textbf{/w/}, \textbf{/y/} et \textbf{les voyelles longues}. 

Si vous avez l'\oe il, vous remarquerez : 

\begin{itemize}
    \item Pour \textbf{/w/} et \textbf{/y/}, il s'agit tout simpelement des deux \textbf{semi-voyelles} du tunisien.
    \item Pour les \textbf{voyelles longues}, elles marquent essentiellement l'absence d'une des trois consonnes de la base\footnote{En linguistique arabe, on dit que la racine est justement défectueuse car il lui manque une consonne.}.
\end{itemize}

En somme, pour repérer un verbe sain simple, faites rapidement les vérifications suivantes :
\begin{itemize}
    \item 4 lettres, dont \textbf{3 consonnes}
    \item Pas de \textbf{/w/},  ni de \textbf{/y/}
    \item \textbf{Pas de voyelle longue}

\end{itemize}

Quelques exemples de verbes sains simples : \textbf{xraj}, \textbf{mro\c{\dh}}, \textbf{mlek}, \textbf{f\v{s}el}, \textbf{l\c{a}ab}, \textbf{\textcrh fa\c{\dh}}, \textbf{s'el}.


\section{Les groupes et leur conjugaison}\label{GroupesVerbesSimples}
Ceci étant dit, il convient de séparer les verbes sains simples en plusieurs groupes, en fonction de la conjugaison qui leur convient. En réalité, la conjugaison de ces trois groupes reste très similaire, et on pourrait imaginer une analyse du tunisien qui ne fasse pas la distinction. Pour un apprentissage plus facile, je vous propose donc de faire cette distinction.

La distinction est en réalité déjà faite en arabe standard. Elle n'est pas forcément enseignée en cours, mais vous verrez naturellement les arabophones conjuguer les verbes selon ces groupes-là.

Je vais distinguer trois groupes au total, ils correspondent en réalité à chacune des trois voyelles de l'arabe standard. 

\textbf{Note :} Pour ceux qui ont l'habitude de l'arabe standard, ces trois groupes correspondent aux trois conjugaisons suivantes : 
\begin{itemize}
    \item \RL{لَعَبَ يَلْعَبُ} qui suit le schème \RL{فَعَلَ يَفْعَلُ}
    \item \RL{خَرَجَ يَخْرُجُ} qui suit le schème \RL{فَعَلَ يَفْعُلُ}
    \item \RL{مَلَكَ يَمْلِكُ} qui suit le schème \RL{فَعَلَ يَفْعِلُ}
\end{itemize}

\subsection{Verbes du premier groupe}
Ce groupe correspond en arabe aux verbes qui se conjuguent au présent avec une \textbf{fat\textcrh a}.

La conjugaison pour le verbe \textbf{l\c{a}ab} (jouer) est la suivante (\textbf{en gras} les préfixes et terminaisons liés à chaque pronom et temps) :

\begin{table}[ht]
\begin{tabularx}{\textwidth}{||X | X | X||}
 \hline
 Pronom & Passé & Présent \\
 \hline\hline
 Éna & l\c{a}ab\textbf{t} & \textbf{na}l\c{a}ab \\
 \hline
 Enti & l\c{a}ab\textbf{t} & \textbf{ta}l\c{a}ab\\ 
 \hline
 Huwwa & l\c{a}ab & \textbf{ya}l\c{a}ab\\ 
 \hline
 Hiyya & la\c{a}b\textbf{et} & \textbf{ta}l\c{a}ab\\ 
 \hline
 A\textcrh na & l\c{a}ab\textbf{naa} & \textbf{na}la\c{a}b\textbf{uu}\\ 
 \hline
 Entuuma & l\c{a}ab\textbf{tuu} & \textbf{ta}la\c{a}b\textbf{uu}\\ 
 \hline
 Huuma & la\c{a}b\textbf{uu} & \textbf{ya}la\c{a}b\textbf{uu}\\ 
 \hline
\end{tabularx}
\end{table}

Voici quelques clés de lecture du tableau ci-dessus :

\begin{itemize}
    \item Pour vous aider dans la prononciation, pensez bien à bien identifier où commence et s'arrête chaque syllabe : repérez d'abord les voyelles, et assemblez les consonnes autour pour vous aider à prononcer le mot;
    \item Faites attention à \textbf{hiyya} et \textbf{huuma} au passé, et aux \textbf{pronoms pluriel} au présent : la voyelle est inversée avec la consonne qui la précède;
    \item La conjugaison de ce groupe se trouve, au point précédent près, n'être qu'une conjugaison basée sur des \textbf{préfixes} et des \textbf{suffixes}.
\end{itemize}

En analysant de plus près ce tableau, vous verrez qu'il existe plusieurs points communs entre les différentes personnes, et donc il existe différents moyens mnémotechniques pour le retenir. Vous pourrez par exemple remarquer que les \textbf{voyelles longues} sont systématiquement associées aux \textbf{pronoms pluriel}, ou que le triplet (\textbf{n, t, y}) correspond dans l'ordre aux premières, deuxièmes, et troisièmes personnes.

\textbf{Note :} Vous pourrez entendre certains locuteurs prononcer la conjugaison au présent pour les personnes plurielles avec \textbf{deux} syllabes plutôt que \textbf{trois}, comme je l'ai marqué au tableau ci-dessus. Dans ce cas, c'est la voyelle centrale qui n'est pas prononcée, et les deux première syllabes qui sont fusionnées.

\subsection{Verbes du deuxième groupe}
Ce groupe correspond en arabe aux verbes qui se conjuguent au présent avec une \textbf{\dh amma}.

La conjugaison pour le verbe \textbf{xraj} (sortir) est la suivante (\textbf{en gras} les préfixes et terminaisons liés à chaque pronom et temps) :

\begin{table}[ht]
\begin{tabularx}{\textwidth}{||X | X | X||}
 \hline
 Pronom & Passé & Présent \\
 \hline\hline
 Éna & xraj\textbf{t} & \textbf{no}xr\textbf{o}j \\
 \hline
 Enti & xraj\textbf{t} & \textbf{to}xr\textbf{o}j\\ 
 \hline
 Huwwa & xraj & \textbf{yo}xr\textbf{o}j\\ 
 \hline
 Hiyya & xarj\textbf{et} & \textbf{to}xr\textbf{o}j\\ 
 \hline
 A\textcrh na & xraj\textbf{naa} & \textbf{no}x\textbf{o}rj\textbf{uu}\\ 
 \hline
 Entuuma & xraj\textbf{tuu} & \textbf{to}x\textbf{o}rj\textbf{uu}\\ 
 \hline
 Huuma & xarj\textbf{uu} & \textbf{yo}x\textbf{o}rj\textbf{uu}\\ 
 \hline
\end{tabularx}
\end{table}

Voici quelques clés de lecture du tableau ci-dessus (les points que j'ai évoqués dans le paragraphe précédent s'appliquent encore) :

\begin{itemize}
    \item Au \textbf{passé}, la conjugaison est strictement \textbf{identique} que pour les verbes du \textbf{premier} groupe;
    \item Cependant au \textbf{présent}, la voyelle change, et se transforme en \textbf{o}. 
\end{itemize}

Une question qui se pose naturellement est la suivante : à partir de la racine uniquement, ou à partir de la forme sous laquelle le verbe est généralement présenté (conjugué avec huwwa au passé), comment peut-on savoir si un groupe appartient au premier ou au deuxième groupe ? 

La réponse est malheureusement décevante : en arabe standard, ces verbes ne se distinguent pas, et il faut apprendre par c\oe ur avec quelle voyelle accorder chaque verbe. Historiquement, il devait peut-être y avoir une raison particulière, un environnement consonantique particulier qui a induit un changement de voyelle, ou une justification sémantique basée sur des schèmes. Mais tout ceci a dû se perdre avec le temps (ou ne fait en tout cas plus partie du savoir commun).

Le tunisien ne fait pas plus d'effort que sa langue-mère sur ce point. Dans la suite de ce cours, je vais essayer de vous présenter, lorsque cela est nécessaire, les verbes sous une forme qui ne laisse pas planer d'ambiguïté (conjugué avec huwwa au présent par exemple).

\textbf{Note :} Vous pourrez entendre certains locuteurs prononcer la conjugaison au présent pour les personnes plurielles avec \textbf{deux} syllabes plutôt que \textbf{trois}, comme je l'ai marqué au tableau ci-dessus. Dans ce cas, c'est la voyelle centrale qui n'est pas prononcée, et les deux première syllabes qui sont fusionnées.

\subsection{Verbes du troisième groupe}\label{ConjSS43}
Ce groupe correspond en arabe aux verbes qui se conjuguent au présent avec une \textbf{kasra}.

La conjugaison pour le verbe \textbf{fhem} (comprendre) est la suivante (\textbf{en gras} les préfixes et terminaisons liés à chaque pronom et temps) :

\begin{table}[ht]
\begin{tabularx}{\textwidth}{||X | X | X||}
 \hline
 Pronom & Passé & Présent \\
 \hline\hline
 Éna & fhem\textbf{t} & \textbf{ne}fhem \\
 \hline
 Enti & fhem\textbf{t} & \textbf{te}fhem\\ 
 \hline
 Huwwa & fhem & \textbf{ye}fhem\\ 
 \hline
 Hiyya & fehm\textbf{et} & \textbf{te}fhem\\ 
 \hline
 A\textcrh na & fhem\textbf{naa} & \textbf{ne}fehm\textbf{uu}\\ 
 \hline
 Entuuma & fhem\textbf{tuu} & \textbf{te}fehm\textbf{uu}\\ 
 \hline
 Huuma & fehm\textbf{uu} & \textbf{ye}fehm\textbf{uu}\\ 
 \hline
\end{tabularx}
\end{table}

Voici quelques clés de lecture du tableau ci-dessus (les points que j'ai évoqués dans le paragraphe précédent s'appliquent encore) :

\begin{itemize}
    \item Au \textbf{passé}, la conjugaison est strictement \textbf{identique} que pour les verbes du \textbf{premier et deuxième} groupe;
    % TODO : Voir pour supprimer cette partie qui n'est pas évidente
    \item Cependant au \textbf{présent}, pour les personnes plurielles, les formes avec \textbf{2} et \textbf{3} syllabes coexistent, et leur prévalence dépendent majoritairement des consonnes du verbe conjugué et de l'usage. 
\end{itemize}

Un avantage majeur de ce groupe est sa démarcation claire au niveau vocalique avec les verbes du premier et deuxième groupe : si vous voyez un \textbf{e} dans une forme conjuguée, vous saurez que c'est un verbe du troisième groupe et pas autre chose ! 

% TODO : voir pour supprimer cette partie ? Porte à confusion
La difficulté réside dans le nombre de syllabes qu'il faut donner pour les trois personnes plurielles au présent. \textbf{Sémantiquement, il n'y pas de réel problème ici :} un tunisophone vous comprendra quoi qu'il arrive. Mais l'usage d'une forme peu employée fera lever plus d'un sourcil. Par exemple :
 \begin{itemize}
     \item Pour \textbf{f\v{s}el} (se fatiguer), on dira systématiquement \textbf{nef\v{s}luu} et non \textbf{nefe\v{s}luu};
     \item Pour \textbf{fhem} (comprendre), on dira plutôt \textbf{nefehmuu}, le préférant à \textbf{nefhmuu};
     \item Pour \textbf{mlek} (posséder), on pourra employer alternativement \textbf{nemelkuu} ou \textbf{nemlkuu}.
 \end{itemize}

Pour être totalement honnête, je n'ai pas encore réussi à trouver de règles fiables qui permettent de séparer l'un ou l'autre des cas. Mon conseil est le suivant : \textbf{utilisez la forme qui vous demande le moins d'effort pour être prononcée}, c'est généralement une bonne façon de discriminer l'une des deux formes.

\section*{Vocabulaire}
Dans cette partie, je vous donne quelques phrases avec des verbes conjugués, appartenant à l'un des trois groupes que nous avons vus. 

TODO : Rajouter des exemples dans cette partie 

\chapter{Former des questions}
\chapterletter{D}ans ce chapitre, nous allons voir comment former des questions en tunisien, aussi bien les questions totales ou partielles. Vous pourrez enfin demander qui a mangé votre déjeuner que vous aviez laissé sur la table !

\section{Un peu d'histoire}
Comme à notre habitude, attardons-nous un peu sur la langue mère du tunisien, pour essayer de comprendre comment la structure actuelle se rattache à la structure historique.

En arabe standard, les questions se forment relativement facilement. Que l'on forme des \textbf{questions totales} (auxquelles on ne peut répondre que par oui ou non) ou des \textbf{questions partielles} (qui demandent à l'interlocuteur une information), la structure est sensiblement identique. Dans tous les cas, la structure et l'ordre de la phrase demeure identique à la phrase déclarative, on ajoute en début de phrase le marqueur correspondant à la question qu'on souhaite poser, et on termine par omettre l'élément sur laquelle la question est posée en cas de question partielle.

Si nous devons faire le parallèle avec le français, les structures peuvent se montrer relativement similaire.

\begin{table}[ht]
\begin{tabularx}{\textwidth}{||X | X | X||}
 \hline
 Français & Arabe & Trad. Littérale \\
 \hline\hline
 Le garçon a mangé une pomme & \RL{أكل الولد تفّاحة} & A-mangé le-garçon pomme \\
 \hline
 Est-ce que le garçon a mangé une pomme ? & \RL{هل أكل الولد تفّاحة؟} & Est-ce-que a-mangé le-garçon pomme ? \\
 \hline
 Qui a mangé une pomme ? & \RL{من أكل تفّاحة ؟} & Qui a-mangé pomme ? \\
 \hline
 Qu'a mangé le garçon ? & \RL{ماذا أكل الولد ؟} & Quoi a-mangé le garçon ? \\
 \hline
\end{tabularx}
\end{table}

Comme vous pouvez le constater, la structure est assez semblable à ce qu'on peut retrouver en français (à la structure de la phrase déclarative près bien sûr, rappelez-vous que l'arabe standard est une langue \textbf{VSO}). J'oserai même dire que la structure est plus simple en arabe car moins permissive en français : on n'autorise typiquement aucune inversion verbe-sujet, comme dans la phrase "As-tu mangé ?". 

\textbf{En tunisien} par contre, l'histoire est un tout petit peu plus compliquée. Si vous vous intéressez à la linguistique, vous savez déjà que les structures des questions sont souvent le domaine dans lesquelles les structures des phrases est le plus libre. 

D'après les recherches que j'ai faites, lors de l'évolution d'une langue depuis un schéma de structure vers un autre, les phrases déclaratives ont plus de facilité à figer et évoluer vers le nouveau schéma, laissant les phrases interrogatives stagner quelque part entre les deux schémas. C'est typiquement le cas du français : l'ancien français est connu pour sa grande liberté sur la structure des phrases, cette dernière s'étant spécialisée vers le \textbf{SVO} jusqu'à se cristalliser en français moderne ; la structure libre des phrases interrogatives en français doivent sans doute être un vestige de cette liberté structurelle.

Nous allons voir dans la suite que \textbf{le tunisien a probablement suivi une voie similaire}, et est devenu beaucoup plus permissif que l'arabe standard (vous aurez compris que c'est souvent le cas).

\section{L'emphase en tunisien}
Juste avant de parler des phrases interrogatives, j'aurais voulu parler de \textbf{l'emphase}.\footnote{C'est sans doute une notion assez avancée pour ce moment précis du cours. J'y consacre un chapitre entier plus loin (chapitre \ref{Emphase}).} 

Nous en parlons maintenant, car il s'agit en réalité d'une méthode \textbf{très populaire} qu'on les tunisophones d'appuyer leurs propos, et qui se manifeste notamment lorsque des questions sont posées, à tel point qu'il a tendance à torturer l'ordre de la phrase interrogative. 

En termes linguistiques, on dira que \textbf{le tunisien emploie des procédés d'emphase par dislocation du sujet}. Cela se traduit typiquement par une omission volontaire du sujet de la phrase, pour aller le catapulter vers la fin de celle-ci.

En réalité, en français moderne parlé, on peut retrouver de telles structures. 

\begin{itemize}
    \item \textbf{Sans emphase} : Pierre me fatigue.
    \item \textbf{Avec emphase} : Il me fatigue, Pierre.
\end{itemize}

En français, on remplace le sujet par le pronom correspondant, et le sujet original est transporté vers la fin de la phrase. Vous verrez qu'en tunisien il s'agit de la même technique, la seule différence étant que les pronoms personnels peuvent être omis (et d'où cette impression que le sujet est inversé).

Retenez toutefois que cela reste cela dit un procédé surtout employé à l'oral, et donc souvent associé à un registre moins soutenu.

Dans la suite de chapitre, je ne vous parlerai pas des structures interrogatives où le sujet se retrouve en fin de phrase, car ces structures sont systématiquement des structures emphatiques. Mais, si cela vous stimule dans votre apprentissage, sachez que nous verrons encore d'autres structures plus loin dans le cours qui vous feront paraître plus naturel à l'oral.

\section{Les questions totales}
Dans cette section, nous allons aborder les \textbf{questions totales}, qui sont les questions qui peuvent être simplement répondues par "oui" ou par "non". En d'autres termes, il s'agit des questions qui contiennent l'ensemble de l'information d'une phrase déclarative.

En tunisien, les questions totales peuvent être formulées de deux façons différentes.

Dans ces deux formes-ci, il est possible de \textbf{suffixer la particule "\v{s}i" au verbe}. Cette particule ne change que très peu le sens de la phrase : on la retrouvera souvent lorsque le locuteur veut \textbf{exprimer son impatience} par exemple. Vous pouvez décider de la mettre ou de l'omettre systématiquement sans crainte, le changement sémantique est en réalité assez mineur.

Je n'ai pas fait d'études linguistiques sur les formes préférées des tunisiens. Je pense d'ailleurs que j'utilise moi-même alternativement l'une et l'autre, surement en fonction du mot sur lequel j'ai le plus envie d'insister et de là où j'ai envie de mettre l'accent dans ma phrase. Mon conseil : ne vous prenez pas la tête, et utilisez celle qui vous convient le mieux.

\subsection{Question totales : première forme}
La première forme, présente dans plusieurs autres langues comme le français et l'anglais, est simplement le changement d'intonation : \textbf{la phrase interrogative obéit aux mêmes règles que la phrase déclarative, le ton est simplement plus haut en fin de phrase.}

Comme évoqué au paragraphe précédent, vous pouvez aussi \textbf{suffixer la particule "\v{s}i" au verbe}.

En voici un exemple : 

\begin{table}[ht]
\begin{tabularx}{\textwidth}{||X | X ||}
 \hline
 Tunisien & Traduction \\
 \hline\hline
 Erraajel fhem. & L'homme a compris. \\
 \hline
 Erraajel fhem ? & L'homme a compris ?\\
 \hline
 Erraajel fhem\v{s}i ? & (idem)\\
 \hline
\end{tabularx}
\end{table}

Contrairement à sa contrepartie en français, cette forme-là en tunisien ne sonne pas aussi peu soutenue. Vous pouvez donc l'utiliser sans trop conscientiser le registre dans lequel vous vous exprimer. 

\subsection{Question totales : seconde forme}
La seconde forme pour les questions totales ressemble fortement à la première, \textbf{à la différence près de l'inversion du sujet et du verbe}. Toutefois, elle est strictement réservées aux \textbf{verbes transitifs}, c'est-à-dire les verbes qui nécessitent un complément d'objet. 

En voici un exemple : 

\begin{table}[ht]
\begin{tabularx}{\textwidth}{||X | X ||}
 \hline
 Tunisien & Traduction \\
 \hline\hline
 Lemraa \textcrh af\dh et eddars. & La femme a appris le cours. \\
 \hline
 \Hwithstroke af\dh et lemraa eddars ? & La femme a-t-elle appris le cours ? \\
 \hline
 \Hwithstroke af\dh et\v{s}i lemraa eddars ? & (idem)\\
 \hline
\end{tabularx}
\end{table}

Prenez bien garde au fait que cette forme ne soit utilisable qu'avec les verbes transitifs. Dans le cas d'une inversion du sujet avec son verbe \textbf{intransitif}, vous serez en réalité en train d'employer une \textbf{emphase}, et ne produirez donc pas \textit{stricto sensu} le même sens.

Sur un sujet tout autre, on pourra noter que c'est également une forme qu'on retrouve dans d'autres langues (le français en est un exemple). Elle tient ses racines de l'arabe standard (rappelez-vous, l'arabe standard est une langue \textbf{VSO}). 

\section{Les questions partielles et leurs marqueurs}
Demander confirmation de quelque chose, c'est bien beau; demander des informations qu'on a pas, c'est autre chose ! 

Dans cette section, nous abordons les \textbf{questions partielles}, qui sont les questions qui ne peuvent pas être simplement répondues par "oui" ou par "non" : l'interlocuteur, s'il daigne vous répondre, vous précisera l'heure de l'action, son lieu, son sujet, etc.

\subsection{Marqueurs interrogatifs}
Abordons en premier lieu les \textbf{marqueurs interrogatifs}. Il s'agit de l'ensemble des mots introduisant la question, et demandant une information particulière. Je nous propose aussi d'aborder les origines de ces marqueurs-là, cela vous aidera si vous êtes arabophone à vous projeter (lisez cette colonne-là de gauche à droite).

\begin{center}
\begin{tabular}{||c | c | c||}
 \hline
 Tunisien & Français & Origine Arabe\\
 \hline\hline
 \v{S} / 'È\v{s} & Quoi / Que & \RL{شيء} (un objet)\\
 \hline
 \v{S}nuwwa & Quoi (objet masculin ou neutre) & \RL{شيء} + \RL{هو} (lui)\\
 \hline
 \v{S}niyya & Quoi (objet féminin) & \RL{شيء} + \RL{هي} (elle)\\
 \hline
 \v{S}nuuma / \v{S}nuhuuma & Quoi (plusieurs objets) & \RL{شيء} + \RL{هم} (eux)\\
 \hline
 \v{S}kuun & Qui & \RL{شيء} + \RL{كون} (un être)\\
 \hline
 Waqtéé\v{s} & Quand & \RL{وقت} (temps) + \RL{شيء} \\
 \hline
 Kiféé\v{s} & Comment & \RL{كيف} (comment) + \RL{شيء} \\
 \hline
 Wiin / Fiin & Où & \RL{أين} (où) (+ \RL{في} (dans))\\
 \hline
 Min wiin / Mniin & Depuis où & \RL{من} (depuis) + \RL{أين}\\
 \hline
 Lwiin & Vers où & \RL{إلى} (vers) + \RL{أين}\\
 \hline
 \c{A}lèè\v{s} & Pourquoi & \RL{على} (sur) + \RL{شيء}\\
 \hline
 Lwèè\v{s} & Pourquoi & \RL{ل} (pour) + \RL{شيء}\\
 \hline
 Anna & Quel / Lequel & ?\\
 \hline
\end{tabular}    
\end{center}

Vous avez sans doute remarqué que la plupart de ces marqueurs comportent le marqueur \textbf{\v{s}}, provenant du mot arabe \RL{شيء} qui désigne un objet ou une chose. C'est sur ce sens que la plupart des marqueurs se sont formés.\footnote{Fait relativement drôle, \RL{شيء} a évolué en tunisien pour donner le mot \textbf{\v{s}èy}, dont un des sens est toujours "objet/chose", mais dont le sens premier est "rien", ce qui en fait un mot \textbf{énantiosémantique} (il s'agit d'un mot qui possède deux sens qui sont antonymes).}

\textbf{Les marqueurs interrogatifs se positionnent généralement en début de phrase}, même s'il est possible à l'oral de leur donner la place du groupe grammatical qu'ils remplacent.

\subsection{Structure des questions partielles}
XXX

\section{Récapitulatif}
XXX

\section*{Vocabulaire}
XXX

\include{Chapitres/ArticlesDéfinis}
\chapter{Le possessif}
\chapterletter{Q}ue ce soit pour parler de son métier, de sa ville natale ou des objets qui nous appartiennent, savoir exprimer le possessif dans une langue est capitale. Regardons ensemble comment celui-ci s'exprime en tunisien.

\section{Un peu d'histoire}
En arabe standard, le possessif s'exprime de façon très régulière, par simple suffixation. Il existe un suffixe différent par pronom personnel, et on peut donc exprimer simplement à qui appartient quelque chose. 

J'oserai même dire qu'il s'exprime plus simplement qu'en français : là où le français fait la distinction entre genre et nombre (\textit{mon, ma, mes}), ce n'est pas le cas en arabe. Ainsi, la donnée du genre et du nombre n'est pas intégrée au suffixe, l'information est déjà comprise dans le nom (pourquoi avoir une information redondante me direz-vous).

En \textbf{tunisien}, l'histoire est un tout petit peu plus compliquée, comme \linebreak d'habitude. Le passage du temps a fait que la prononciation a évolué à différentes vitesses dans des cas de figures différents : la prononciation des noms communs féminins et des possessifs associés en est un exemple. \footnote{On verra également que c'est le cas pour certains noms communs se finissant par une lettre spécifique, en l'occurrence le \textbf{i}.}

Les arabophones voient certainement de quoi je parle. Il s'agit la non-prononciation des \RL{ة} en fin de mot : il se trouve que l'évolution de la langue a fait que l'arrêt de prononciation de cette lettre ne s'est fait que si la lettre termine le mot, ce qui n'est pas le cas du possessif. 

Pour les non-arabophones, afin d'illustrer cet exemple-là dans un cas de figure, nous pouvons parler de cette même évolution (l'arrêt de la prononciation de la dernière lettre d'un mot féminin) en prenant l'exemple de la prononciation des prénoms d'origine arabe en Afrique subsaharienne. Il se trouve que dans ces pays, la lettre finale est toujours prononcée dans les prénoms féminins, ce qui fait que le prénom \RL{أمينة} se prononcera \textbf{Amina} au Maghreb, mais \textbf{Aminatou} dans les pays plus au sud.

Tout ceci pour dire que : le \textbf{tunisien} fait de nos jours la distinction des genres sur l'expression du possessif, mais que cette distinction n'est pas le fruit d'une différence purement grammaticale et structurelle. La logique derrière la formation de ces possessifs est donc la même.

\section{La forme courte du possessif}
En \textbf{tunisien}, il existe deux manières d'exprimer la possessif : une forme \textbf{courte}, qui dérive de l'arabe standard, et une forme \textbf{longue}, spécifique au tunisien.

La forme courte du possessif est très compacte, et est donc assez privilégiée. Elle a cependant le mauvais goût de s'exprimer de trois façons différentes, séparants ainsi les mots en trois groupes distincts :
\begin{itemize}
    \item Les noms \textbf{féminins} ;
    \item Les noms \textbf{masculins ne se terminant pas par une voyelle} ;
    \item Les noms \textbf{masculins se terminant par une voyelle}.
\end{itemize}

Commençons d'abord par les noms féminins et les noms masculins ne se terminant pas par une voyelle, ces deux groupes formant en quelque sorte la forme \textit{régulière} du possessif.

Voici les formes possessives pour les mots \textbf{xobz} (du pain) et \textbf{xobza} (un pain, une baguette) :

\begin{center}
\begin{tabular}{||c | c | c||}
 \hline
 Pronom & \textbf{xobz} & \textbf{xobza}\\
 \hline\hline
 'Éna & xobz\textbf{ii} & xobz\textbf{tii}\\
 \hline
 'Enti & xobz\textbf{ek} & xobz\textbf{tek}\\
 \hline
 Huwwa & xobz\textbf{uu} & xobz\textbf{tuu}\\
 \hline
 Hiyya & xobz\textbf{haa} & xobz\textbf{ethaa}\\
 \hline
 'A\textcrh na & xobz\textbf{naa} & xobz\textbf{etnaa}\\
 \hline
 'Entuuma & xobz\textbf{kom} & xobz\textbf{etkom}\\
 \hline
 Huuma & xobz\textbf{hom} & xobz\textbf{ethom}\\
 \hline
\end{tabular}    
\end{center}

Comme vous pouvez le remarquer, les deux formes sont globalement très similaires, à cela près que la forme du possessif pour les mots féminins remplacent le \textbf{a} terminal du mot par un \textbf{t}.

Également, il faudra noter la présence d'un \textbf{e} pour 4 des 7 personnes au féminin : \textbf{hiyya, 'a\textcrh na, 'entuuma, huuma}. À cause du remplacement du \textbf{a} par un \textbf{t}, un cluster de trois consonnes successives serait apparu sans l'ajout de ce \textbf{e}. En tant que moyen mnémotechnique, vous pouvez donc retenir que si la terminaison commence par une consonne, alors vous devez remplacer le \textbf{a} par un \textbf{et} (et non un \textbf{t}). 

Ceci en poche, vous êtes maintenant capable d'exprimer le possessif pour la quasi-totalité des noms communs ! Reste maintenant à l'exprimer pour les mots masculins finissant par une voyelle.

Prenons pour exemple ces trois mots-ci : \textbf{sbéédrii} (des chaussures de \linebreak sport\footnote{Si vous avez l'esprit affûté, vous aurez remarqué que c'est littéralement le mot \textit{espadrille} qui a été importé et déformé. Le pluriel et le singulier se prononcent de la même façon.}), \textbf{baakuu} (un paquet \footnote{J'espère que votre esprit était affûté cette fois aussi, c'est une fois de plus le même mot}) et \textbf{k\v{o}paa} (un compas\footnote{Je vous laisse deviner.}). Vous remarquerez juste la petite subtilité sur la longueur des voyelles et des consonnes  pour la première personne (les autres personnes ayant des terminaisons identiques).

\begin{center}
\begin{tabular}{||c | c | c | c ||}
 \hline
 Pronom & \textbf{sbéédrii} & \textbf{baakuu} & \textbf{k\v{o}paa}\\
 \hline\hline
 'Éna & sbéédri\textbf{yya} & baakuu\textbf{ya}& k\v{o}paa\textbf{ya}\\
 \hline
 'Enti & sbéédrii\textbf{k} & baakuu\textbf{k}& k\v{o}paa\textbf{k}\\
 \hline
 Huwwa & sbéédrii\textbf{h} & baakuu\textbf{h}& k\v{o}paa\textbf{h}\\
 \hline
 Hiyya & sbéédrii\textbf{haa} & baakuu\textbf{haa} & k\v{o}paa\textbf{haa}\\
 \hline
 'A\textcrh na & sbéédrii\textbf{naa} & baakuu\textbf{naa}& k\v{o}paa\textbf{naa}\\
 \hline
 'Entuuma & sbéédrii\textbf{kom} & baakuu\textbf{kom}& k\v{o}paa\textbf{kom}\\
 \hline
 Huuma & sbéédrii\textbf{hom} & baakuu\textbf{hom}& k\v{o}paa\textbf{hom}\\
 \hline
\end{tabular}    
\end{center}

Il est intéressant de noter que les seules différences se trouvent dans les trois premières personnes, qui se trouvent également être les seules trois personnes pour lesquelles la terminaison \textbf{commencent par une voyelles}. Vous commencez à comprendre maintenant la logique derrière : le tunisien n'aime pas enchaîner les voyelles, et donc l'évolution de la prononciation à travers le temps s'est arrangée pour ne pas créer de telles structures. 

Si vous voulez aller encore plus loin dans votre apprentissage de l'histoire de l'évolution du tunisien, vous pouvez par exemple retenir que la terminaison pour \textbf{huwwa} était initialement la même pour l'ensemble des mots, peu importe leur genre ou leur lettre finale. La terminaison en \textbf{arabe standard} était \textbf{-huu} : on retrouve la première moitié dans le possessif des noms se terminant par une voyelle, et la seconde moitié dans le possessif des noms ne se terminant pas par une voyelle.

\section{La forme longue du possessif}
La seconde forme du possessif, la forme \textbf{longue}, est une innovation du \textbf{tunisien}. Elle s'exprime par l'ajout du mot \textbf{mtèè\c{a}}, venant du mot arabe \RL{متاع} qui veut dire \textit{affaires, bagages}. De nos jours, \textbf{mtèè\c{a}} a perdu son sens lorsqu'employé tout seul en tunisien.

Pour exprimer cette forme longue, il faut procéder comme suit : 
\begin{itemize}
    \item Utiliser la forme \textbf{définie} du nom à qualifier ;
    \item Mettre \textbf{mtèè\c{a}} sous la forme \textbf{possessive courte} (c'est un nom commun masculin);
    \item Juxstaposer le nom à qualifier et \textbf{mtèè\c{a}} (avec sa terminaison appropriée).
\end{itemize}

Ce qui donne, en utilisant l'exemple avec \textbf{xobz} : 

\begin{center}
\begin{tabular}{||c | c | c||}
 \hline
 Pronom & \textbf{xobz - Forme courte} & \textbf{xobz - Forme longue}\\
 \hline\hline
 'Éna & xobz\textbf{ii} & \textbf{el}xobz \textbf{mtèè\c{a}ii}\\
 \hline
 'Enti & xobz\textbf{ek} & \textbf{el}xobz \textbf{mtèè\c{a}ek}\\
 \hline
 Huwwa & xobz\textbf{uu} & \textbf{el}xobz \textbf{mtèè\c{a}uu}\\
 \hline
 Hiyya & xobz\textbf{haa} & \textbf{el}xobz \textbf{mtèè\textcrh\textcrh aa}\\
 \hline
 'A\textcrh na & xobz\textbf{naa} & \textbf{el}xobz \textbf{mtèè\c{a}naa}\\
 \hline
 'Entuuma & xobz\textbf{kom} & \textbf{el}xobz \textbf{mtèè\c{a}kom}\\
 \hline
 Huuma & xobz\textbf{hom} & \textbf{el}xobz \textbf{mtèè\textcrh\textcrh om}\\
 \hline
\end{tabular}    
\end{center}

Le seul point sur lequel je souhaite attirer votre attention est la forme particulière pour \textbf{hiyya} et \textbf{huuma}. Il s'agit là juste d'une évolution de la prononciation, pour s'abstraire de la suite de consonne \textbf{\c{a}h} qui est dure à prononcer. Nous en avons déjà parlé plus tôt, il s'agit de l'assimilation (cf. \ref{Assimilation})\footnote{En toute rigueur, il serait possible de garder la même orthographe et d'admettre que cette suite de consonne est simplifiée en \textcrh\textcrh, mais je ne suis pas convaincu de la décorrélation de la prononciation et de l'orthographe, même si certaines langues se l'autorisent pour des raisons historiques généralement (le français et l'anglais ont sont de parfaits exemples).}.

Pour mieux appréhender l'usage de cette forme, on peut en donner un équivalent en français (certes un peu lourd), qui serait quelque chose du style : 

\begin{center}
    \textbf{elxobz mtèè\c{a}ii <-> Le pain à moi}
\end{center}

Une hypothèse que j'ai , qui ne me semble ne pas être si folle que ça, et qui permet de mieux comprendre l'origine de cette structure innovante est la suivante : \textbf{il s'agirait initialement d'une phrase nominale}, où l'objet à qualifier est le sujet et \textbf{mtèè\c{a}} le complément, et qui a fini par être tant utilisée que son intégration directement en tant que groupe nominal a été autorisé.

D'ailleurs, on peut même noter que, de nos jours, \textbf{utiliser la forme longue toute seule constitue une phrase nominale valide} (le verbe être étant bien entendu sous-entendu comme dans toute phrase nominale).

Voilà, avec tout ceci, vous devrez maintenant être capable de désigner les choses qui vous appartiennent !  

\section*{Vocabulaire}
XXX
\chapter{Expressions courantes}
\h{Votre premier dialogue}
\l{À} partir de maintenant, j'estime qu'on a assez fait de grammaire pour qu'on puisse lire ensemble un dialogue, et l'analyser !

Ne vous inquiétez pas, la grammaire revient dès le prochain chapitre, même si maintenant, nous allons commencer à intégrer de plus en plus de dialogue dans le cours (ça me donnera une bonne excuse pour vous donner du vocabulaire).

\hh{Dialogue}
- \c{A}aslèèma !

- \c{A}aslèèma ! 'Esmii Mo\c{s}\c{t}faa. \v{S}nuwwa 'esmek ? 

- Net\v{s}arrfuu Mo\c{s}\c{t}faa. 'Esmii Mahdii. 

- Net\v{s}arrfuu. \v{S}nuwwa xedemtek, Mahdii ?

- Nexdem felhandsa. Ya\c{a}nii, 'éna muhandes.

- Hattééna muhandes !

- Wiin toskon ? 

- 'Éna noskon fi Bériiz. W 'entii ? 

- Noskon huuni, fi Marsiilya.

- Tnejjem twarriini elbled ? 

- Bi\c{t}\c{t}bii\c{a}a !

\include{Chapitres/Démonstratifs}
\include{Chapitres/ComplémentsNoms}
\chapter{Les verbes dérivés et leur conjugaison}\label{DérivesVerbes}
\chapterletter{N}ous avons vu dans un précédent chapitre la conjugaison des verbes \textbf{simples} (cf. chapitre \ref{ConjSS}). Ces verbes avaient tous la particularité d'être formés de trois consonnes (pour les verbes sains), et d'une voyelle unique. Il existe en opposition à ces verbes-là les verbes \textbf{dérivés} qui sont comme leur nom l'indique produits à partir des verbes simples. Je vous propose de nous attarder quelques instants dessus.

\section{Un peu d'histoire}
\subsection{Les bases triconsonantiques et la dérivation}
Je vous le disais déjà au paragraphe \ref{ConjSS1}, les langues sémitiques s'articulent toutes autour de bases triconsonantiques (des \textbf{triples de consonnes}). Ces bases sont en charge de contenir un \textit{sens principal} à partir duquel on pourra dériver tous les mots appartenant à un champ lexical particulier. 

Par exemple, prenons la racine sémitique \textbf{K-T-B} (telle quelle en \textbf{arabe} \RL{كتب}, et \textbf{K-T-V} en \textbf{hébreu} \<ktb>\footnote{Cela se prononce bien avec un \textbf{/v/}, mais cela s'écrit bien avec un \<b> dont l'équivalent en arabe (\RL{ب}) produit le son \textbf{/b/}.}). Cette racine porte en elle le champ lexical de l'\textbf{écriture} : \textbf{/kataba/} (\textsc{ar}\footnote{Code ISO pour l'arabe}) et /\textbf{ktav}/ (\textsc{iw}\footnote{Code ISO pour l'hébreu}) veulent tous les deux dire \textit{il a écrit} ; tandis que \textbf{/kitéébon/} (\textsc{ar}) veut dire \textit{livre}, \textbf{/maktabon/} (\textsc{ar}) veut dire \textit{école}, et \textbf{/kétuva/} (\textsc{iw}) veut dire \textit{contrat de mariage}.

Comme vous le voyez, il est possible d'agrémenter la base triconsonantique de \textbf{voyelles} et \textbf{consonnes} supplémentaires afin d'en changer le sens. Ces changements de sens obéissent à des règles, et se manifestent par des \textbf{schèmes}, qui sont en réalité des schémas à appliquer, qui sont les mêmes pour l'ensemble des bases. 

Ainsi, en arabe, le schème correspondant au \textbf{sujet} de l'action est \textbf{/féé\c{a}ilon/} \RL{فاعل}\footnote{La base \textbf{F-\c{A}-L}, qui veut dire \textit{faire}, sert d'exemple par défaut à l'application des schèmes en arabe.}, alors que le schème correspondant au \textbf{patient} de l'action (celui qui la subit) est \textbf{/maf\c{a}uulon/} \RL{مفعول}. 

En pratique ce la donne : 

\begin{itemize}
    \item \textbf{/kéétibon/} est un \textit{écrivain}, \textbf{/léé\c{a}ibon/}\footnote{Vous connaissez déjà cette base. Rappelez-vous du verbe \textbf{l\c{a}ab} en tunisien qui veut dire \textit{jouer}.} est un \textit{joueur}, \textbf{/qaari'on/}\footnote{\textbf{Q-R-'} se rapporte à tout ce qui a trait à la \textit{lecture}.} est un \textit{lecteur}.
    \item \textbf{/ma\v{s}ruubéét/}\footnote{\textbf{\v{S}-R-B} veut dire \textit{boire}.} sont des \textit{boissons},\textbf{/ma'kuuléét/}\footnote{\textbf{'-K-L} veut dire \textit{manger}.} veut dire \textit{nourriture}, \textbf{/maktuubon/} veut dire \textit{destin}.
\end{itemize}

Retenez donc qu'il y a en arabe deux notions (qu'on retrouvera en tunisien) : 

\begin{itemize}
    \item La \textbf{base triconsonantique} qui code pour le champ lexical : c'est l'équivalent de la \textbf{racine} en français ;
    \item Le \textbf{schème} qui vient préciser le sens de la \textbf{base} : c'est l'équivalent des \textbf{préfixes} et \textbf{suffixes} en français.
\end{itemize}

\subsection{La dérivation des verbes}
La puissance de l'utilisation du duo \textbf{base/schème} se fait notamment sentir dès lors qu'il s'agit d'extraire de nouveaux verbes à partir de verbes qu'on connaît déjà.

L'arabe redouble de créativité quand il s'agit de trouver des schèmes, et en produit plusieurs dizaines pour les verbes seulement. Chacun apporte sa nuance particulière, et il est donc possible de générer très simplement un sens très précis, pour peu qu'on connaisse suffisamment bien les schèmes à notre disposition\footnote{Je soulève quand même un point noir : certains schèmes appliqués à des bases bien précises n'ont pas de sens bien définis, ou sont des mots un peu désuets. Ainsi, on ne se permettra pas n'importe quel schème avec n'importe quelle base, de la même manière que les mots \textit{rétro-marche} ou \textit{bi-stranguel} ne veulent rien dire en français, contrairement à \textit{rétro-conception} et \textit{bi-hebdomadaire}.}. 

Dans cette sous-partie, je ne souhaite pas détailler l'ensemble des schèmes qui existent en arabe, ce serait trop long. Mais je souhaite quand même évoquer les plus importants et les plus usés, ce qui via des cas d'usage pratique vous aidera à comprendre les formes existantes qui perdurent encore aujourd'hui en tunisien.

Je vous présente dans le tableau suivant des exemples d'application de schèmes sur diverses bases.

\begin{center}
\begin{tabular}{||c | c | c | c | c||}
 \hline
  \textbf{Base} & \textbf{Signif.} & \textbf{Dérivé} & \textbf{Signif.} & \textbf{Signif. schème}\\
 \hline\hline
  \RL{كَتَبَ} & Ecrire & \RL{كُتِبَ} & Être écrit & Forme passive\\
  \textbf{/kataba/} & & \textbf{/kutiba/} & & \\
  \hline
  \RL{أَكَلَ} & Manger & \RL{أَكَّلَ} & Faire manger & Causatif\\
  \textbf{/'èkèlè/} & & \textbf{/'èkkèlè/} & & \\
  \hline
  \RL{زَوَجَ} & Unir & \RL{تَزَوَّجَ} & Se marier & Réflexif\\
  \textbf{/zawaja/} &  & \textbf{/tazawwaja/} & & \\
  \hline
  \RL{عَوَنَ} & Secourir & \RL{تَعَاوَنَ} & S'entraider & Interaction\\
  \textbf{/\c{a}awana/} &  & \textbf{/ta\c{a}aawana/} & & \\
  \hline
  \RL{كَسَبَ} & Posséder & \RL{اكْتَسَبَ} & Acquérir & Action pour soi\\
  \textbf{/kasaba/} & & \textbf{/'ektèsèbè/} & &\\
  \hline
  \RL{خَرَجَ} & Sortir & \RL{اسْتَخْرَجَ} & Extraire & Action minutieuse\\
  \textbf{/xaraja/} &  & \textbf{/'estaxraja/} &  & \\
  \hline
\end{tabular}    
\end{center}

Les exemples ci-dessus ne sont que cela : des exemples. Les schèmes sont en réalité plus compliqués que cela, dans la mesure où certains d'entre eux portent des sens qui sont plus souvent soumis à l'interprétation. Cela se comprend relativement bien : le sens des mots a tendance à évoluer avec le temps, et cette dérive ne prend pas nécessairement le sens du schème en compte. Le sens global d'un schème dérive alors graduellement.

Ne nous reste alors qu'un seul élément à aborder : la \textbf{conjugaison}. En \textbf{arabe}, il faut globalement retenir que chaque \textit{schème verbal} a sa conjugaison qui lui est propre. On pourrait presque parler de groupes de verbes\footnote{C'est ce que j'ai décidé de faire dans le chapitre \ref{ConjSS}.}. Cependant dans l'ensemble, il apparaît que la conjugaison en arabe reste relativement régulière et déductible\footnote{Je vous accorde que cela est subjectif.} : le moyen le plus sécuritaire reste d'apprendre toutes les conjugaisons par c\oe ur, mais il reste tout à fait possible de déduire de façon subconsciente des règles générales via la pratique de la langue. 

Fort heureusement, en \textbf{tunisien}, les choses sont plus simples. Les schèmes verbaux sont encore présents, mais leur nombre s'est considérable réduit. Plusieurs schèmes ressortent clairement du lot, alors que d'autres schèmes présents en arabe se sont fait supplantés par l'usage de \textbf{marqueurs préverbaux} et les \textbf{verbes modaux}\footnote{Il s'agit d'une nouveauté, déjà partiellement présente en arabe, mais qui a été largement développée par le tunisien. J'y consacre deux chapitres plus loin, aux chapitres \ref{VerbMod} et \ref{MarqPVer}.}.

D'une façon générale, je vous conseille de ne pas essayer de sur-analyser l'ensemble des verbes que vous pourrez rencontrer. Découper les verbes en \textit{base + schème} se révélera très utile par moment pour comprendre rapidement le sens d'un mot que vous ne connaissez pas, mais ce ne sera pas une technique infaillible : l'usage faisant la grammaire (et non l'inverse), il vous arrivera de tomber sur des \textbf{schèmes inusités} ou des sur des \textbf{bases verbales qui n'ont plus de sens seule}.

\section{Dérivation des verbes en tunisien}
Rentrons dans le coeur du sujet : la \textbf{dérivation verbale} en tunisien. 

Avant toute chose j'aimerais que vous reteniez une information : \textit{malgré tous vos efforts, il vous serra inutile et quasi-impossible d'identifier tous les schèmes encore en usage en tunisien, et de déterminer leur sens}. Il vous sera par contre beaucoup plus utile de savoir faire ces deux choses que je vais vous présenter. 

La première compétence à maîtriser est la détermination dans chaque verbe  de ce qui relève : 
\begin{itemize}
    \item De la \textbf{base} : c'est la structure consonantique qui porte l'essentiel du sens du verbe (son champ lexical) ; 
    \item Du \textbf{schème} : c'est le moule à partir duquel le verbe est formé, et c'est ce qui termine de définir le sens du verbe ; 
    \item De la \textbf{conjugaison} : c'est ce qui donne l'information sur le temps et la personne à laquelle le verbe se réfère, et qui concrétise l'action dans le temps et le contexte.
\end{itemize}

La deuxième compétence est celle d'apprendre par c\oe ur les schèmes les plus importants du tunisien. Connaître ces schèmes, c'est savoir \textbf{exprimer des nuances efficacement} à partir d'une base verbale que vous connaissez déjà. 

A travers ce chapitre, je vous propose donc de développer ces deux compétences en vous présentant les \textbf{schèmes verbaux} du tunisien. Faites l'effort à chaque exemple de déterminer par vous-même les consonnes composant la base. 

\subsection{Schème de la voix causative}
Le schème de la voix \textbf{causative} est l'un des deux schèmes les plus utilisés en tunisien.

Il permet de transformer un verbe afin d'employer la \textbf{voix causative} dans un phrase, c'est-à-dire que le fait que \textbf{le sujet fasse exécuter l'action au complément, ou change l'état du complément}. En \textbf{français}, la voix causative est formée en ajoutant le verbe \textit{faire} avant un infinitif, comme dans \textit{je l'ai fait sortir} ou \textit{je lui ferai signer}.

Il se forme comme suit à partir de la base : 
\begin{center}
    \Large{\textbf{1 2 3} $\rightarrow$ \textbf{1 a 2 2 e 3}}
\end{center}
où les chiffres désignent chacun une consonne de la base.

\textbf{\textsc{Note :}} En tunisien, le schème de la voix causative produit des verbes \textbf{transitifs}, c'est-à-dire que ces verbes imposent la présence d'un complément d'objet direct ou indirect.

Voici quelques exemples : 

\begin{center}
\begin{tabular}{||c | c | c | c ||}
 \hline
  \textbf{Base} & \textbf{Trad.} & \textbf{Causatif} & \textbf{Trad.} \\
 \hline\hline
  l\c{a}ab & \textit{jouer} & la\c{a}\c{a}eb & \textit{faire jouer}\\
  \hline
  xraj & \textit{sortir} & xarrej & \textit{faire sortir}\\
  \hline
  fhem & \textit{comprendre} & fahhem & \textit{expliquer}\\
  \hline
  wqef & \textit{s'arrêter} & waqqef & \textit{faire s'arrêter}\\
  \hline
  fsed & \textit{devenir corrompu} & fassed & \textit{corrompre}\\
  \hline
  mro\c{\dh} & \textit{devenir malade} & marre\c{\dh} & \textit{contaminer}\\
  \hline
\end{tabular}    
\end{center}

D'une façon plus générale, vous pourrez également retrouver des verbes sous leur forme \textbf{causative}, sans pour autant que la forme \textbf{simple} ne soit usitée.

\begin{center}
\begin{tabular}{||c | c ||}
 \hline
  \textbf{Causatif} & \textbf{Trad.} \\
 \hline\hline
  sakker & \textit{fermer} \\
  \hline
  sawwed & \textit{noircir} \\
  \hline
\end{tabular}    
\end{center}

Et finalement quelques exemples d'utilisation : 

\begin{center}
\begin{tabular}{||c | c ||}
 \hline
 \textbf{Tunisien} & \textbf{Français} \\
 \hline\hline
 La\c{a}\c{a}abt ess\v{r}aar. & \textit{J'ai fait jouer les enfants.} \\ 
 \hline
 Xarrej elkalb. & \textit{Il a fait sortir le chien.} \\ 
 \hline
 Fahhmet eddars lelbnayya. & \textit{Elle a expliqué le cours à la fille.} \\ 
 \hline
 Nwaqqef elmu\c{t}uur ? & \textit{Est-ce que j'arrête le moteur ?} \\ 
 \hline
 Hé\dh uukom errjèèl fassduu ejjaw. & \textit{Ces hommes ont pourri l'ambiance.} \\ 
 \hline
 Huwwa marre\c{\dh} oxtii. & \textit{Il a contaminé ma s\oe ur.} \\
 \hline
 Sakkert elbééb. & \textit{J'ai fermé la porte.} \\
 \hline
 Sawwed yidduu belf\textcrh am. & \textit{Il a noirci sa main avec le charbon.} \\
 \hline
\end{tabular}
\end{center}

\subsection{Schème de la voix passive}
Le schème de \textbf{la voix passive} est le deuxième schème le plus employé en tunisien. 

Il transforme le verbe afin d'employer la \textbf{voix passive} dans une phrase, c'est-à-dire le fait que \textbf{le sujet subit l'action}. En \textbf{français}, la voix passive est formée en juxtaposant le verbe être et le participe passé du verbe, comme dans \textit{la pomme a été mangée}.

En \textbf{tunisien}, la voix passive se construit \textbf{en ajoutant /t/ \underline{avant} le verbe.}. Si le verbe commence par \textbf{deux consonnes}, alors on ajoute un \textbf{/e/} supplémentaire pour aider la prononciation.  

\begin{center}
    \Large{\textbf{verbe} $\rightarrow$ \textbf{t (+ e) + verbe}}
\end{center}

\textbf{\textsc{Notes :}} 
\begin{itemize}
    \item En tunisien, le schème de la voix passive produit des verbes \textbf{intransitifs}, c'est-à-dire que ces verbes n'acceptent pas de complément d'objet direct ou indirect.
    \item L'accent tonique est situé sur la même syllabe que l'accent tonique de la base\footnote{Cela permet notamment de distinguer la voix passive avec la base conjuguée au présent pour la deuxième personne du singulier et la troisième personne féminin du singulier.}.
\end{itemize}

Voici quelques exemples : 

\begin{center}
\begin{tabular}{||c | c | c | c ||}
 \hline
  \textbf{Base} & \textbf{Trad.} & \textbf{Voix passive} & \textbf{Trad.} \\
 \hline\hline
  l\c{a}ab & \textit{jouer} & tel\c{a}ab & \textit{se jouer}\\
  \hline
  fhem & \textit{comprendre} & tefhem & \textit{se faire comprendre}\\
  \hline
  bla\c{a} & \textit{avaler} & tebla\c{a} & \textit{être avalé}\\
  \hline
  kteb & \textit{écrire} & tekteb & \textit{être écrit}\\
  \hline
\end{tabular}    
\end{center}

Et également des exemples d'utilisation : 

\begin{center}
\begin{tabular}{||c | c ||}
 \hline
 \textbf{Tunisien} & \textbf{Français} \\
 \hline\hline
 E\v{ss}kobba tetel\c{a}ab belkwaaret. & \textit{La chkobba se joue avec des cartes.} \\ 
 \hline
 Elxo\c{t}\c{t}a tfehmet. & \textit{Le plan a été compris.} \\ 
 \hline
 El\textcrh arbuu\v{s}a tbel\c{a}et. & \textit{La pilule a été avalée.} \\ 
 \hline
 Elkontraatu tekteb. & \textit{Le contrat a été écrit.} \\ 
 \hline
\end{tabular}
\end{center}

\subsection{Schème de la voix réfléchie}
Le schème de la \textbf{voix réfléchie} est également un schème assez courant en tunisien. 

Il transforme le verbe afin d'employer la \textbf{voix réfléchie} dans une phrase, c'est-à-dire le fait que \textbf{le sujet soit l'objet de sa propre action}. En \textbf{français}, la voix réfléchie s'exprime par l'emploi de verbes pronominaux comme dans \textit{le garçon s'est lavé}. En \textbf{anglais}, on utilisera les constructions dans le style de \textit{he taught himself}.

En \textbf{tunisien}, la voix réfléchie se forme à partir de la base :

\begin{center}
    \Large{\textbf{1 2 3} $\rightarrow$ \textbf{t 1 a 2 2 e 3}}
\end{center}

où les chiffres désignent chacun une consonne de la base.

\textbf{\textsc{Note:}}
\begin{itemize}
    \item Il existe d'autres manières d'exprimer la voix réfléchie. Cela sera abordé dans un chapitre ultérieur (cf. chapitre \ref{PronCompl}).
    \item L'accent tonique est situé sur la même syllabe que celui de la voix \textbf{causative}.
\end{itemize}

Voici quelques exemples : 

\begin{center}
\begin{tabular}{||c | c | c | c ||}
 \hline
  \textbf{Base} & \textbf{Trad.} & \textbf{Voix réfléchie} & \textbf{Trad.} \\
 \hline\hline
  \c{a}ro\c{\dh} & \textit{croiser (qqch/qqn)} & t\c{a}arre\c{\dh} & \textit{se confronter (à qqch) / prévoir}\\
  \hline
  wqa\c{a} & \textit{se dérouler} & twaqqe\c{a} & \textit{s'imaginer (qqch)}\\
  \hline
  \c{a}lem & \textit{informer} & t\c{a}allem & \textit{apprendre}\\
  \hline
  \dh kar & \textit{mentionner} & t\dh akker & \textit{se rappeler}\\
  \hline
\end{tabular}    
\end{center}

Puis d'autres exemples dont la base n'est plus employée : 

\begin{center}
\begin{tabular}{||c | c ||}
 \hline
 \textbf{Voix réfléchie} & \textbf{Trad.} \\
 \hline\hline
 tsakker & \textit{se fermer} \\
 \hline
 tmassex & \textit{se salir} \\
 \hline
 twajjeh & \textit{se diriger} \\
 \hline
 t'axxer & \textit{être en retard} \\
 \hline
 tsalleq & \textit{escalader} \\
 \hline
 tkallem & \textit{parler} \\
 \hline
 t\textcrh adde\th & \textit{discuter} \\
 \hline
\end{tabular}
\end{center}

Finalement des exemples d'utilisation : 

\begin{center}
\begin{tabular}{||c | c ||}
 \hline
 \textbf{Tunisien} & \textbf{Français} \\
 \hline\hline
 T\c{a}arre\dh\ lelmo\v{s}kla kbiira. & \textit{Il a été confronté à un gros problème.} \\ 
 \hline
 Twaqqa\c{a}naa muut erra'iis. & \textit{Nous avons prévu la mort du président.} \\ 
 \hline
 Yet\c{a}allmuu elgitaar. & \textit{Ils apprennent la guitare.} \\ 
 \hline
 Ta\dh kkret elli nséét portabelhaa. & \textit{Elle s'est rappelée qu'elle a oublié son portable.} \\ 
 \hline
 Elbééb tsakker. & \textit{La porte s'est fermée.} \\ 
 \hline
 Eddba\v{s} tmassex. & \textit{Les vêtements se sont salis.} \\ 
 \hline
 Twajjeht lelxruuj. & \textit{Tu t'es dirigé vers la sortie.} \\ 
 \hline
 \c{A}lèè\v{s} t'axxertuu ? & \textit{Pourquoi étiez vous en retard ?} \\ 
 \hline
 Yetsalleq el\textcrh ii\c{t}. & \textit{Il a escaladé le mur.} \\ 
 \hline
 Netkallem m\c{a}aak. & \textit{Je parle avec toi.} \\ 
 \hline
 Tet\textcrh adde\th\ m\c{a}aaya. & \textit{Tu discutes avec toi.} \\ 
 \hline
\end{tabular}
\end{center}

\subsection{Schème de la voix causative-passive}
Le schème de la \textbf{voix causative-passive} est le dernier des grands schèmes à retenir pour la maîtrise du tunisien. 

Il transforme le verbe afin d'employer la \textbf{voix causative-passive} dans une phrase, c'est-à-dire le fait que le \textbf{le sujet soit forcé par quelqu'un d'autre à faire une action}. On on peut retrouver cette voix en \textbf{français} dans des constructions comme \textit{On m'a fait mangé quelque chose que je n'aimais pas}. Le \textbf{japonais} possède d'ailleurs une forme verbale causative-passive : \textbf{/taberu/}veut dire \textit{manger}, \textbf{/tabesaserareru/} veut dire \textit{être forcé à manger}.

En \textbf{tunisien}, la voix causative-passive se forme de la même façon que la voix réflexive\footnote{Ce qui est en soit très logique pour deux raisons. Morphologiquement, la voix réflexive se construit en préfixant un \textbf{/t/} à la forme causative. Sémantiquement, la voix réflexive peut s'interpréter comme le fait de se forcer soi-même à faire quelque chose.} : 

\begin{center}
    \Large{\textbf{1 2 3} $\rightarrow$ \textbf{t 1 a 2 2 e 3}}
\end{center}

où les chiffres désignent chacun une consonne de la base.

\textbf{\textsc{Note :}} 
\begin{itemize}
    \item Contrairement au \textbf{français}, le schème de la voix causative-passive en tunisien produit des verbes intransitifs, c'est-à-dire qu'ils n'admettent pas de complément. Ainsi, on se saura pas \textit{qui} a forcé l'accomplissement de l'action.
    \item L'accent tonique se situe sur la même syllabe que celui de la voix \textbf{causative}.
\end{itemize}


Voici quelques exemples : 

\begin{center}
\begin{tabular}{||c | c | c | c ||}
 \hline
  \textbf{Causatif} & \textbf{Trad.} & \textbf{Voix caus-pass} & \textbf{Trad.} \\
 \hline\hline
  xarraj & \textit{faire sortir} & txarraj & \textit{se faire sortir}\\
  \hline
  sajjal & \textit{enregistrer} & tsajjal & \textit{se faire enregistrer}\\
  \hline
  la\c{s}\c{s}aq & \textit{coller} & tla\c{s}\c{s}aq & \textit{se faire coller}\\
  \hline
  wazza\c{a} & \textit{distribuer} & twazza\c{a} & \textit{se faire distribuer}\\
  \hline
\end{tabular}    
\end{center}

Et des exemples d'utilisation : 

\begin{center}
\begin{tabular}{||c | c ||}
 \hline
 \textbf{Tunisien} & \textbf{Français} \\
 \hline\hline
 E\dh\dh ebbééna txarrjet melkujiina. & \textit{\makecell{La mouche a été sortie de \\ la cuisine.}} \\ 
 \hline
 El\v{r}néyéét tsajjluu filkasèèt. & \textit{\makecell{Les chansons ont été enregistrées \\ dans la cassette.}} \\ 
 \hline
 Elpostèèr tlassaq \c{a}al \textcrh iit. & \textit{Le poster a été collé sur le mur.} \\ 
 \hline
 Lekwaaret twazz\c{a}uu. & \textit{On a distribué les cartes.} \\ 
 \hline
\end{tabular}
\end{center}

\subsection{Schème de la voix réciproque}
Le schème de la \textbf{voix réciproque} est un schème relativement moins utilisé que ceux qu'on a vu jusqu'à présent. 

Il transforme le verbe afin d'employer la \textbf{voix réciproque} dans une phrase, c'est-à-dire le fait que \textbf{le sujet entreprenne une action avec le complément}\footnote{En tunisien, on sous-entend généralement avec la forme réciproque que l'action dure dans le temps.}. En \textbf{français}, cette voix est souvent signifiée par la présence de \textit{avec}, comme dans \textbf{j'ai parlé avec lui}. En \textbf{anglais}, on utilisera plutôt les constructions de la forme \textit{they spoke with each other}.

En \textbf{tunisien}, la voix réfléchie se forme à partir de la base :

\begin{center}
    \Large{\textbf{1 2 3} $\rightarrow$ \textbf{t 1 é é 2 e 3}}
    
    \Large{\textbf{1 2 3} $\rightarrow$ \textbf{t 1 a a 2 e 3}}
\end{center}

où les chiffres désignent chacun une consonne de la base.

\textbf{\textsc{Note:}} Les verbes exprimant la voix réciproque nécessite nécessairement l'emploi de la préposition \textbf{m\c{a}aa} (\textit{avec}) pour signifier avec qui l'action est menée.

Voici quelques exemples : 

\begin{center}
\begin{tabular}{||c | c | c | c ||}
 \hline
  \textbf{Base} & \textbf{Trad.} & \textbf{Voix réfléchie} & \textbf{Trad.} \\
 \hline\hline
  l\c{a}ab & \textit{jouer} & tléé\c{a}eb & \textit{jouer (avec qqn)}\\
  \hline
  fhem & \textit{comprendre} & tfééhem & \textit{s'entendre (avec qqn)}\\
  \hline
  \c{a}mal & \textit{faire} & t\c{a}aamel & \textit{interagir (avec qqn)}\\
  \hline
  - & \textit{-} & t\c{a}aarek & \textit{se battre (avec qqn)}\\
  \hline
  - & \textit{-} & tkéélem & \textit{parler (avec qqn)}\\
  \hline
\end{tabular}    
\end{center}

Et des exemples d'utilisation : 

\begin{center}
\begin{tabular}{||c | c ||}
 \hline
 \textbf{Tunisien} & \textbf{Français} \\
 \hline\hline
 Lulééd yetléé\c{a}buu m\c{a}aa b\c{a}\c{/dh}hom. & \textit{Les enfants jouent ensemble.} \\ 
 \hline
 Monya tfééhmet m\c{a}aah. & \textit{Monia s'est arrangée avec lui.} \\ 
 \hline
 Taw net\c{a}aamel m\c{a}aahaa. & \textit{Je vais voir avec elle.} \\ 
 \hline
 \c{A}lèè\v{s} t\c{a}aarektuu ? & \textit{Pourquoi vous êtes vous battus ?} \\ 
 \hline
 Yetkéélem m\c{a}aaya. & \textit{Il parle avec moi.} \\ 
 \hline
\end{tabular}
\end{center}

\subsection{Schème de l'aspect inchoatif}
Le schème de l'\textbf{aspect inchoatif} est un schème assez rare, qui ne se retrouve bien souvent que dans un cas assez particulier, que nous allons aborder.

Il transforme le verbe afin d'employer l'aspect \textbf{inchoatif} dans une phrase, c'est-à-dire le fait que \textbf{le sujet entre dans un état particulier}. 

En \textbf{tunisien}, l'aspect inchoatif ne se forme qu'avec les \underline{couleurs}. Il se forme à partir de la base : 

\begin{center}
    \Large{\textbf{1 2 3} $\rightarrow$ \textbf{1 2 a a 3}}
    
    \Large{\textbf{1 2 3} $\rightarrow$ \textbf{1 2 è è 3}}
\end{center}

où les chiffres désignent chacun un consonne de la base. 

\textbf{\textsc{Note :}} Les verbes à l'aspect inchoatif sont tous \textbf{intransitifs}, c'est-à-dire qu'ils n'admettent pas de complément d'objet.

Voici quelques exemples : 

\begin{center}
\begin{tabular}{||c | c | c | c ||}
 \hline
  \textbf{Nom} & \textbf{Trad.} & \textbf{Aspect inchoatif} & \textbf{Trad.} \\
 \hline\hline
  'a\textcrh mar & \textit{rouge} & \textcrh maar & \textit{rougir}\\
  \hline
  'a\c{s}far & \textit{jaune} & \c{s}faar & \textit{jaunir}\\
  \hline
  'ax\c{\dh}ar & \textit{vert} & x\c{\dh}aar & \textit{verdir}\\
  \hline
  'ax\c{\dh}ar & \textit{vert} & x\c{\dh}aar & \textit{verdir}\\
  \hline
  'azraq & \textit{bleu} & zrèèq & \textit{bleuir}\\
  \hline
  'ak\textcrh al & \textit{noir} & k\textcrh aal & \textit{noircir}\\
  \hline
  'abya\c{\dh} & \textit{blanc} & byaa\c{\dh} & \textit{blanchir}\\
  \hline
  fèèta\textcrh & \textit{clair} & ftèè\textcrh & \textit{s'éclaircir}\\
  \hline
  \v{r}aameq & \textit{sombre} & \v{r}mèèq & \textit{s'assombrir}\\
  \hline
\end{tabular}    
\end{center}

Ainsi que des exemples d'utilisation :

\begin{center}
\begin{tabular}{||c | c ||}
 \hline
 \textbf{Tunisien} & \textbf{Français} \\
 \hline\hline
 \textcrh maarnaa. & \textit{Nous avons rougi.} \\ 
 \hline
 Elfee x\dh aar. & \textit{Le feu est passé au vert.} \\ 
 \hline
 Elmèè ftèè\textcrh. & \textit{L'eau est devenue claire.} \\ 
 \hline
 Essmèè \v{r}emqet. & \textit{Le ciel s'est assombri.} \\ 
 \hline
\end{tabular}
\end{center}

\section{Conjugaison des verbes dérivés}
Maintenant que nous avons vu les schèmes principaux de la langues tunisienne, je vous propose de prendre un exemple par schème et de parcourir sa conjugaison\footnote{Gardez à l'esprit qu'il existe potentiellement des variations vocaliques mineures, relativement ponctuelles. Les conjugaisons qui sont données après représentent la conjugaison "principale", celle qu'il faut apprendre, mais des variations/exceptions peuvent exister.}.

Dans la suite de cette section, gardez à l'esprit deux choses : 
\begin{itemize}
    \item Les tables de conjugaison seront toutes relativement similaires à celles des verbes sains simples (cf. \ref{ConjSS43}), notamment au niveau des préfixes et suffixes\footnote{La différence majeure réside dans les clusters de consonnes, que le tunisien cherche toujours à éviter, et de la décomposition en syllabes des verbes conjugués.}. 
    \item Faites attention aux personnes et au temps pour lesquels il y a une inversion de lettres (systématiquement une consonne avec une voyelle).
\end{itemize}

\subsection{Schème de la voix causative}
Le verbe qui servira d'exemple est \textbf{fahhem} (\textit{faire comprendre}).

\conjugaison{fahhem}
    {fahhem\textbf{t} , \textbf{n}fahhem}
    {fahhem\textbf{t} , \textbf{t}fahhem} 
    {fahhem , \textbf{y}fahhem}
    {fahhme\textbf{t} , \textbf{t}fahhem}
    {fahhem\textbf{naa} , \textbf{n}fahhm\textbf{uu}}
    {fahhem\textbf{tuu} , \textbf{t}fahhm\textbf{uu}} 
    {fahhm\textbf{uu} , \textbf{y}fahhm\textbf{uu}} 


\subsection{Schème de la voix passive}
Le verbe qui servira d'exemple est \textbf{tekteb} (\textit{être écrit}).


\conjugaison{tekteb}
 {tekteb\textbf{t} , \textbf{ne}tekteb}
 {tekteb\textbf{t} , \textbf{te}tekteb} 
 {tekteb , \textbf{ye}tekteb}
 {tketbe\textbf{t} , \textbf{te}tekteb}
 {tekteb\textbf{naa} , \textbf{ne}tketb\textbf{uu}}
 {tekteb\textbf{tuu} , \textbf{te}tkteb\textbf{uu}}
 {tketb\textbf{uu} , \textbf{ye}tketb\textbf{uu}}

\subsection{Schèmes des voix réfléchie et causative-passive}
Les deux schèmes se ressemblant, le verbe qui servira d'exemple pour les deux est \textbf{t\c{a}allem} (\textit{apprendre}).

\conjugaison{t\c{a}allem}
    {t\c{a}allem\textbf{t} , \textbf{ne}t\c{a}allem}
    {t\c{a}allem\textbf{t} , \textbf{te}t\c{a}allem} 
    {t\c{a}allem , \textbf{ye}t\c{a}allem}
    {t\c{a}allme\textbf{t} , \textbf{te}t\c{a}allem}
    {t\c{a}allem\textbf{naa} , \textbf{ne}t\c{a}allm\textbf{uu}}
    {t\c{a}allem\textbf{tuu} , \textbf{te}t\c{a}allm\textbf{uu}}
    {t\c{a}allm\textbf{uu} , \textbf{ye}t\c{a}allm\textbf{uu}} 

\subsection{Schème de la voix réciproque}
Le verbe qui servira d'exemple est \textbf{tlèè\c{a}eb} (\textit{jouer avec quelqu'un}).

\conjugaison{tlèè\c{a}eb}
 {tlèè\c{a}eb\textbf{t} , \textbf{ne}tlèè\c{a}eb}
 {tlèè\c{a}eb\textbf{t} , \textbf{te}tlèè\c{a}eb} 
 {tlèè\c{a}eb , \textbf{ye}tlèè\c{a}eb}
 {tlèè\c{a}be\textbf{t} , \textbf{te}tlèè\c{a}eb}
 {tlèè\c{a}eb\textbf{naa} , \textbf{ne}tlèè\c{a}b\textbf{uu}}
 {tlèè\c{a}eb\textbf{tuu} , \textbf{te}tlèè\c{a}b\textbf{uu}} 
 {tlèè\c{a}b\textbf{uu} , \textbf{ye}tlèè\c{a}b\textbf{uu}} 

\subsection{Schème de l'aspect inchoatif}
Les deux verbes qui serviront d'exemples sont \textbf{\textcrh maar} (\textit{rougir}) et \textbf{zrèèq} (\textit{bleuir}).

\conjugaison{\textcrh maar}
    {\textcrh mar\textbf{t} , \textbf{ne}\textcrh maar}
    {\textcrh mar\textbf{t} , \textbf{te}\textcrh maar} 
    {\textcrh maar , \textbf{ye}\textcrh maar}
    {\textcrh maare\textbf{t} , \textbf{te}\textcrh maar}
    {\textcrh mar\textbf{naa} , \textbf{ne}\textcrh maar\textbf{uu}}
    {\textcrh mar\textbf{tuu} , \textbf{te}\textcrh maar\textbf{uu}} 
    {\textcrh maar\textbf{uu} , \textbf{ye}\textcrh maar\textbf{uu}} 

\conjugaison{zrèèq}
    {zreq\textbf{t} , \textbf{ne}zrèèq}
    {zreq\textbf{t} , \textbf{te}zrèèq} 
    {zrèèq , \textbf{ye}zrèèq}
    {zrèèqe\textbf{t} , \textbf{te}zrèèq}
    {zreq\textbf{naa} , \textbf{ne}zrèèq\textbf{uu}}
    {zreq\textbf{tuu} , \textbf{te}zrèèq\textbf{uu}} 
    {zrèèq\textbf{uu} , \textbf{ye}zrèèq\textbf{uu}} 

\section{Quelques mots}
Ce chapitre est déjà relativement long, et je vous ai déjà plus que submergé d'informations. Je ne souhaite pas l'allourdir davantage, mais il y a plusieurs points que je tenais quand même à évoquer rapidement, d'une part pour compléter votre compréhension du tunisien, et d'autre part pour votre culture générale (si ça vous intéresse).

\subsection{Sur les verbes importés et leurs dérivés}
Quelque chose de particulièrement vrai à propos du tunisien, et dans une mesure qui m'est inconnue à propos de l'arabe, est la facilité d'importer des mots de d'autres langues, et de les forcer dans des moules existants pour les faire obéir à la logique interne de la langue. 

Dans le cas de \textbf{l'arabe}, on peut par exemple parler de \RL{خَرِيطَة} \textbf{/xarii\c{t}a/} (\textit{une carte}) et \RL{قرْطَاس} \textbf{/qar\c{t}aas/} (\textit{du papier}) qui dérivent tous les deux d'un seul même mot \textbf{grec} $\chi\alpha\rho\tau\eta\zeta$ \textbf{/khártēs/}\footnote{Vous aurez sans reconnu l'ancêtre du mot \textbf{carte} en français.}. Le mot \textbf{/xarii\c{t}a/} tout particulièrement a été introduit dans le moule des bases triconsonantiques, et en est sorti \textit{de force} la base \textbf{X-R-\c{T}}.

De cette base, beaucoup d'autres mots on été dérivés, aux sens potentiellement exotiques pour certains : \textbf{/xara\c{t}a/} (\textit{dépouiller}), \textbf{/maxruu\c{t}/} (\textit{un cône}), \textbf{/'enxara\c{t}a/} (\textit{dégainer}) ou encore \textbf{/'exrawwa\c{t}a/} (\textit{être emmêlé}).

On pourrait trouver encore beaucoup d'autres exemples de mots de langues voisines qui se sont intégrées à l'arabe au fil du temps, certains qui je sui sûr étonneront plus d'un arabophone. Il est important de se dire que ce système d'emprunt existe encore de nos jours en \textbf{tunisien}. Ne vous étonnez donc pas de trouver plusieurs mots qui vous sont familiers qui ont été plus ou moins \textit{charcutés}, soit pour en extraire un équivalent de base triconsonantique, soit pour leur appliquer directement des schèmes existants.

Quelques exemples notables : 
\begin{itemize}
    \item \textbf{riigel} (\textit{réparer}), qui vient du français \textit{réguler}, qui se décline par exemple en \textbf{triigel} (\textit{se faire réparer}) ;
    \item \textbf{trééna} (\textit{s'entraîner}), qui vient du français \textit{s'entraîner}, qui s'est vu extraire une racine : \textbf{rééna}. Cette extraction a notamment été permise par l'assmiliation du \textbf{/t/} de \textit{s'en\underline{t}rainer} à un schème de la \textbf{voix passive}.
    \item \textbf{barres} (\textit{débarasser la table}), qui vient du français \textit{débarasser}, qui provient sans doute de l'assmilation de \textit{tu débarasses} à une forme \textbf{'enti tebarres / tbarres}, d'où la forme \textbf{huwwa barres}.
\end{itemize}

Ces quelques notes pour vous dire que l'ensemble des mots importés en tunisien sont suceptibles d'être adaptés à souhait, et dérivés à leur tour. Le système de dérivation des verbes est donc sans doute plus \textit{fluide} que ce que le début du chapitre peut laisser penser.

\subsection{Sur les bases quadriconsonantiques}
XXX

\subsection*{Dialogue}
\subsection*{Vocabulaire}

\chapter{Conjugaison des verbes simples quasi-sains}
\chaptermark{Verbes simples quasi-sains}

\chapter{Le futur et l'impératif}

\chapter{Les genres et les nombres}

\chapter{Les adjectifs}

\chapter{Conjugaison des verbes simples malades}

\chapter{Conjugaison des verbes irréguliers}

\chapter{Les verbes modaux}\label{VerbMod}

\chapter{Les marqueurs préverbaux}\label{MarqPVer}

\chapter{Les pronoms compléments directs et indirects} \label{PronCompl}

\chapter{L'emphase}
\label{Emphase}


\end{document}
