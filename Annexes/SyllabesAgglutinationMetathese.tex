\chapter{Syllabes, affixes et métathèse}
\chapterletter{D}ans ce chapitre, je vous propose de vous intéresser à la structure des syllabes en tunisien, et comment les syllabes d'un mot peuvent se restructurer lors de l'ajout de préfixes et de suffixes. Ce chapitre ne sera pas simple, mais il offre une description essentielle à la compréhension de certaines "exceptions" que nous avons pu voir. Accrochez-vous, ça vaut le coup !

\section{Qu'est-ce qu'une syllabe ?}
\subsection{Définition}
Cela peut paraître étrange de vous donner une définition d'une notion qu'on connaît tous, mais cela me semble important avant d'attaquer la suite, histoire qu'on nous ayons tous les mêmes termes pour communiquer. 

Une \textbf{syllabe} est définie comme une \textbf{unité intermédiaire} entre le phonème (\textit{un son}) et un mot. C'est donc : 
\begin{itemize}
    \item Une structure composée de \textbf{plusieurs phonèmes}, qui sont considérés comme l'unité insécable du langage ; 
    \item Une structure dont on se sert pour \textbf{former des mots}, qui sont eux les structures qui portent le sens.
\end{itemize}

En pratique donc, on va pouvoir identifier pour chaque mot les syllabes qui le composent, et pour chaque syllabe les sons qui la composent. 

\subsection{Comment se construit une syllabe ?}
En \textbf{linguistique}, on identifie pour chaque langue l'ensemble des syllabes qui peuvent être formées. On essaye donc de définir de façon concise pourquoi on pourrait dire \textit{dextre} mais pas \textit{uknlre}\footnote{Ça ne veut rien dire, vous pourriez peut-être le prononcer, mais ça ne sonne pas très français, non ?}.

Pour ce faire, on note généralement les consonnes avec un \textbf{C} et les voyelles avec un \textbf{V}, on met entre paranthèses les parties optionnelles, et on définit tout autre symbole qui nous semble juste. 

On pourra par exemple voir la notation suivante pour décrire les syllabes attestées en \textbf{français} : 

\begin{center}
    {\LARGE (C) (C) (C) V (C) (C) (C)}
\end{center}

ce qui veut tout simplement dire la chose suivante : \textbf{une syllabe en français se construit autour d'une voyelle, et peut être précédée et/ou suivie par trois consonnes au maximum.}

Voici quelques exemples (notez bien dans la suite la distinction qui est faite entre \textit{sons} et \textit{lettres}) : 

\begin{center}
    \begin{tabular}{||c | c||}
        \hline
        \textbf{Structure} & \textbf{Exemple} \\ \hline \hline
        \textbf{V} & \textit{\oe ufs} \\ \hline
        C\textbf{V} & \textit{mots} \\ \hline
        CCC\textbf{V}CC & \textit{strict} \\ \hline
        \textbf{V}CCC & \textit{arbre} \\ \hline
    \end{tabular}
\end{center}

Dans la plupart des langues du monde, les syllabes se construisent autour d'une \textbf{voyelle}, mais les sons qui la précèdent et la suivent varie en nombre et en qualité en fonction de la langue. Toujours pour l'exemple du \textbf{français}, on compte plusieurs mots commençant par la suite de sons \textbf{/str/} (\textit{structure} par exemple), mais aucun qui ne commence par la suite de sons \textbf{/ktr/}.

Dans d'autres langues, la suite de sons \textbf{/str/} est tout bonnement interdite\footnote{Elle n'existe dans aucun mot de cette langue.}, comme en \textbf{tunisien}, et nous allons voir pourquoi à la section suivante.

\subsection{Structure des syllabes en tunisien}
En \textbf{tunisien}, on retrouvera les structures syllabiques suivantes : 

\begin{center}
    \begin{tabular}{||c | c | c||}
        \hline 
        \textbf{Structure} & \textbf{Ex. mot} &\textbf{Ex. verbe}\\ \hline \hline
        C\textbf{V} & mèè (\textit{eau}) & raa (\textit{voir})\\ \hline
        C\textbf{V}C & \hb uut (\textit{poisson}) & qaal (\textit{dire})\\ \hline
        CC\textbf{V}C & braq (\textit{éclair, foudre}) & xraj (\textit{sortir})\\ \hline
        C\textbf{V}CC & mel\hb{} (\textit{sel}) & mass (\textit{toucher})\\ \hline
        CC\textbf{V}CC & -  & xrajt (\textit{je suis sorti})\\ \hline
        CC\underline{\textbf{V}} & mraa (\textit{femme}) & braa (\textit{guérir})\\ \hline
    \end{tabular}
\end{center}

où \textbf{C} représente une consonne, \textbf{V} représente une voyelle, \textbf{\underline{V}} une voyelle \underline{longue} et \textbf{\underline{v}} une voyelle \underline{courte}.

Comme il est possible de le constater :

\begin{itemize}
    \item On ne retrouvera jamais de syllabe commençant par une voyelle seule, toutes les syllabes commencent nécessairement par une \textbf{consonne} ;
    \item Certaines structures ne sont permises que si la voyelle est \textbf{longue} ;
    \item La structure CCVCC n'apparaît que lors de la conjugaison au passé à la première et deuxième personne du singulier.
\end{itemize}

Puisque j'ai fait le choix dans ce cours d'adopter une orthographe \textbf{phonétique}, il en vient qu'il vous est possible à travers l'orthographe uniquement de deviner la façon dont il faut effectuer la découpe d'un mot en syllabes (cf. section \ref {DécompositionSyllabes}).

On peut étudier les cas suivants à titre d'exemple :

\begin{center}
    \begin{tabular}{|| c | c | c | c ||}
        \hline
        \textbf{Exemple} & \textbf{Découpe} & \textbf{Structure} & \textbf{Traduction} \\ \hline \hline
        noxroj & no | xroj & CV.CCVC & \textit{Je sors} \\ \hline
        noxorjuu & no | xor | juu & CV.CVC.CV & \textit{Nous sortons} \\ \hline
        q\ct aa\ct es& q\ct aa | \ct es & CC\underbar{V}.CVC & \textit{Des chats} \\ \hline
        ordinater & 'or | di | na | ter & CVC.CV.CV.CVC & \textit{Ordinateur}\\ \hline
        \hb maaruu & \hb maa | ruu & CC\underline{V}.CV & \textit{Ils ont rougi}\\ \hline
        talvza & talv | za & CVCC.CV & \textit{Télévision} \\ \hline
    \end{tabular}
\end{center}

\section{Affixes et restructuration des syllabes}
