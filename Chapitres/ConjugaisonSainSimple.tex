\chapter{Conjugaison des verbes sains simples}
\label{ConjSS}
\chapterletter{P}assons maintenant à la conjugaison des verbes réguliers simples en tunisien, qui forment une bonne partie des verbes qui vous pourrez rencontrer.

Je vous propose de passer un peu de temps sur les verbes en arabe standard, et comment le tunisien a évolué à partir de cette langue. Cette partie n'est pas essentielle pour apprendre à parler tunisien, mais comprendre la façon dont tout s'articule en arabe standard et comment ces particularités se sont transmises au tunisien vous aidera à comprendre la logique sous-jacente de la conjugaison tunisienne.

\section{Un peu d'histoire} \label{ConjSS1}
Commençons simplement par remettre dans le contexte la conjugaison et les verbes du tunisien, et faire le pont avec ce qu'on peut retrouver dans l'arabe standard.

Ce qui me semble être un gros avantage dans l'apprentissage de l'arabe standard, une fois la barrière de l'alphabet passé, est que la conjugaison est relativement simple. On rencontre certes 13 personnes différentes, mais l'arabe comporte très peu de temps : \textbf{le passé}, \textbf{le présent} (à partir duquel on déduit d'autre temps en ajoutant des préfixes ou en changeant les voyelles finales), et \textbf{l'impératif}. En plus de cela, on ajoute une conjugaison extrêmement systématique et régulière, qui fait qu'on ne retrouvera pas de listes de verbes irréguliers à apprendre par c\oe ur. 

L'arabe standard se construit autour de ce qu'on appelle des \textbf{racines sémitiques}. Ces racines sont des \textbf{triplets de consonnes} (moins couramment des quadruplets), qui portent toute une famille de sens, et dont dérivent la quasi-totalité des mots de la langue par l'application de \textbf{schèmes} (des structures de voyelles et de consonnes qui peuvent être rajoutées pour changer le sens de la racine). Nous en reparlerons plus tard dans ce cours, car le tunisien se construit également autour de ces racines et de ces schèmes.

Je vous parle de cette particularité pour trois raisons principales :
\begin{itemize}
    \item Premièrement, pour qu'on puisse définir ce qu'est un \textbf{verbe simple}, qui se résume simplement en la phrase suivante : "\textbf{Un verbe est simple si toutes les consonnes qui le composent sont les consonnes de sa racine}". Ainsi, dès que vous verrez un verbe composé d'exactement trois consonnes, vous saurez qu'il est simple.
    \item Deuxièmement, pour vous sensibilisez sur le fait que \textbf{la conjugaison des verbe s'appuiera sur des schèmes}. Ainsi, même si des verbes peuvent se ressembler dans une certaine forme particulière (typiquement, conjugués au passé à la 3ème personne masculin du singulier), le reste de la conjugaison peut varier.
    \item Finalement, parce que les linguistes étudiant l'arabe standard font la distinction entre plusieurs \textbf{groupes de verbes}, et que l'appartenance à ces groupes se déduit par l'agencement des consonnes au sein de la racine. 
\end{itemize}


\section{Évolutions en tunisien}

Fort heureusement, l'arabe tunisien garde quasiment toute la régularité de sa langue mère. Certaines simplifications sont apparues avec le temps, notamment au niveau des pronoms comme nous l'avons déjà vu. Mais comme dans toutes les langues du monde, qui dit simplifications dit également apparition d'irrégularités.

Je vous propose de lister maintenant les points communs et les évolutions qui sont apparues dans le tunisien, tant au niveau de la grammaire qu'au niveau de la prononciation.

\begin{itemize}
    \item Comme dit plus haut, le tunisien ne comporte que \textbf{7 pronoms}, au lieu des 13 de l'arabe standard;
    \item Le tunisien \textbf{a perdu l'ensemble des temps dérivés} du présent en arabe standard, dont le futur, au profit de l'ajout de verbes semi-auxiliaires (nous en reparlerons);
    \item Beaucoup de voyelles se sont perdues dans le temps, ce qui donnera \linebreak l'impression au tunisien l'impression d'avoir une conjugaison beaucoup plus compacte.
    \item Le tunisien a gardé l'ensemble des \textbf{groupes des verbes} des l'arabe standard;
    \item A cause de \textbf{l'omission des coups de glottes} en début et en fin de mot, au profit du rallongement des voyelles voisines, la conjugaisons de certains verbes est devenue semi-irrégulière; 
    \item L'emploi de certaines voyelles en tunisien doit être fait avec plus de précision qu'en arabe standard, même s'il existe beaucoup d'allophonie (des sons que les locuteurs assimilent au même phonème).
\end{itemize}

Étant donnée la nature vivante d'une langue, il est possible qu'en parlant à certains locuteurs tunisiens, vous vous rendiez compte que leur prononciation sera différente de ce que je vais vous présenter dans ce chapitre. Comme toutes les langues, cela ne voudra pas dire que ce que vous appris ou que ce locuteur a prononcé est faux, simplement qu'il a plusieurs accents et plusieurs manières de réaliser certains points de grammaire. 

Ce dernier point est particulièrement important : lors que j'ai fait mes premières recherches sur la conjugaison des verbes, j'ai remarqué qu'il était possible de substituer certaines voyelles à d'autres, voire même d'en omettre certaines et de fusionner des syllabes, et ce chez plusieurs locuteurs qui ont a priori le même accent. 

Ainsi, je vous propose dans la suite de ce chapitre de vous enseigner une conjugaison qui soit facile à retenir, même si la prononciation n'est pas la prononciation majoritaire, vu qu'il n'existe pas de version standardisée du tunisien. Cela facilitera votre apprentissage tout en vous permettant d'être intelligible.

\section{Qu'est-ce qu'un verbe sain simple ?}
On appelle verbe sain simple un verbe qui est à la fois : 
\begin{itemize}
    \item \textbf{Simple :} Comme évoqué plus haut, il s'agit de l'ensemble des verbes dont toutes les consonnes sont exactement celles de sa racine. À quelques exceptions près, vous pouvez admettre que un verbe est sain s'il comportement exactement trois consonnes.
    \item \textbf{Sain :} Il s'agit d'un verbe dont toutes les consonnes de sa racine sont saines.
\end{itemize}

Attardons-nous quelques instants sur la dénomination \textbf{consonne saine}. En arabe standard, et en tunisien, on fait la distinction, au niveau de la conjugaison, entre les lettres saines et les lettres défectueuses. Ces dernières forment l'ensemble des lettres dont la présence dans une racine change la façon de conjuguer le verbe. 

On compte au total quatre marques de la défectuosité d'une racine : \textbf{/w/}, \textbf{/y/}, \textbf{/'/} et \textbf{les voyelles longues}\footnote{En arabe, il n'y en a que trois, le coup de glotte n'étant pas compté comme défectueux, contrairement au tunisien.}. 

Si vous avez l'\oe il, vous remarquerez : 

\begin{itemize}
    \item Pour \textbf{/w/} et \textbf{/y/}, il s'agit tout simpelement des deux \textbf{semi-voyelles} du tunisien.
    \item Pour \textbf{/'/}, il faut savoir que le coup de glotte a un comportement un peu particulier (j'en parle par exemple en \ref{FormeTerrestre}).
    \item Pour les \textbf{voyelles longues}, elles marquent essentiellement l'absence d'une des trois consonnes de la base\footnote{En linguistique arabe, on dit que la racine est justement défectueuse car il lui manque une consonne.}.
\end{itemize}

En somme, pour repérer un verbe sain simple, faites rapidement les vérifications suivantes :
\begin{itemize}
    \item 4 lettres, dont \textbf{3 consonnes}
    \item Pas de \textbf{/w/},  ni de \textbf{/y/}, ni de \textbf{/'/}
    \item \textbf{Pas de voyelle longue}

\end{itemize}

Quelques exemples de verbes sains simples : \textbf{xraj}, \textbf{mro\c{\dh}}, \textbf{mlek}, \textbf{f\v{s}el}, \textbf{l\c{a}ab}, \textbf{\textcrh fa\c{\dh}}.


\section{Les groupes et leur conjugaison}
Ceci étant dit, il convient de séparer les verbes sains simples en plusieurs groupes, en fonction de la conjugaison qui leur convient. En réalité, la conjugaison de ces trois groupes reste très similaire, et on pourrait imaginer une analyse du tunisien qui ne fasse pas la distinction. Pour un apprentissage plus facile, je vous propose donc de faire cette distinction.

La distinction est en réalité déjà faite en arabe standard. Elle n'est pas forcément enseignée en cours, mais vous verrez naturellement les arabophones conjuguer les verbes selon ces groupes-là.

Je vais distinguer trois groupes au total, ils correspondent en réalité à chacune des trois voyelles de l'arabe standard. 

\textbf{Note :} Pour ceux qui ont l'habitude de l'arabe standard, ces trois groupes correspondent aux trois conjugaisons suivantes : 
\begin{itemize}
    \item \RL{لَعَبَ يَلْعَبُ} qui suit le schème \RL{فَعَلَ يَفْعَلُ}
    \item \RL{خَرَجَ يَخْرُجُ} qui suit le schème \RL{فَعَلَ يَفْعُلُ}
    \item \RL{مَلَكَ يَمْلِكُ} qui suit le schème \RL{فَعَلَ يَفْعِلُ}
\end{itemize}

\subsection{Verbes du premier groupe}
Ce groupe correspond en arabe aux verbes qui se conjuguent au présent avec une \textbf{fat\textcrh a}.

La conjugaison pour le verbe \textbf{l\c{a}ab} (jouer) est la suivante (\textbf{en gras} les préfixes et terminaisons liés à chaque pronom et temps) :

\begin{table}[ht]
\begin{tabularx}{\textwidth}{||X | X | X||}
 \hline
 Pronom & Passé & Présent \\
 \hline\hline
 Éna & l\c{a}ab\textbf{t} & \textbf{na}l\c{a}ab \\
 \hline
 Enti & l\c{a}ab\textbf{t} & \textbf{ta}l\c{a}ab\\ 
 \hline
 Huwwa & l\c{a}ab & \textbf{ya}l\c{a}ab\\ 
 \hline
 Hiyya & la\c{a}b\textbf{et} & \textbf{ta}l\c{a}ab\\ 
 \hline
 A\textcrh na & l\c{a}ab\textbf{naa} & \textbf{na}la\c{a}b\textbf{uu}\\ 
 \hline
 Entuuma & l\c{a}ab\textbf{tuu} & \textbf{ta}la\c{a}b\textbf{uu}\\ 
 \hline
 Huuma & la\c{a}b\textbf{uu} & \textbf{ya}la\c{a}b\textbf{uu}\\ 
 \hline
\end{tabularx}
\end{table}

Voici quelques clés de lecture du tableau ci-dessus :

\begin{itemize}
    \item Pour vous aider dans la prononciation, pensez bien à bien identifier où commence et s'arrête chaque syllabe : repérez d'abord les voyelles, et assemblez les consonnes autour pour vous aider à prononcer le mot;
    \item Faites attention à \textbf{hiyya} et \textbf{huuma} au passé, et aux \textbf{pronoms pluriel} au présent : la voyelle est inversée avec la consonne qui la précède;
    \item La conjugaison de ce groupe se trouve, au point précédent près, n'être qu'une conjugaison basée sur des \textbf{préfixes} et des \textbf{suffixes}.
\end{itemize}

En analysant de plus près ce tableau, vous verrez qu'il existe plusieurs points communs entre les différentes personnes, et donc il existe différents moyens mnémotechniques pour le retenir. Vous pourrez par exemple remarquer que les \textbf{voyelles longues} sont systématiquement associées aux \textbf{pronoms pluriel}, ou que le triplet (\textbf{n, t, y}) correspond dans l'ordre aux premières, deuxièmes, et troisièmes personnes.

\textbf{Note :} Vous pourrez entendre certains locuteurs prononcer la conjugaison au présent pour les personnes plurielles avec \textbf{deux} syllabes plutôt que \textbf{trois}, comme je l'ai marqué au tableau ci-dessus. Dans ce cas, c'est la voyelle centrale qui n'est pas prononcée, et les deux première syllabes qui sont fusionnées.

\subsection{Verbes du deuxième groupe}
Ce groupe correspond en arabe aux verbes qui se conjuguent au présent avec une \textbf{\dh amma}.

La conjugaison pour le verbe \textbf{xraj} (sortir) est la suivante (\textbf{en gras} les préfixes et terminaisons liés à chaque pronom et temps) :

\begin{table}[ht]
\begin{tabularx}{\textwidth}{||X | X | X||}
 \hline
 Pronom & Passé & Présent \\
 \hline\hline
 Éna & xraj\textbf{t} & \textbf{no}xr\textbf{o}j \\
 \hline
 Enti & xraj\textbf{t} & \textbf{to}xr\textbf{o}j\\ 
 \hline
 Huwwa & xraj & \textbf{yo}xr\textbf{o}j\\ 
 \hline
 Hiyya & xarj\textbf{et} & \textbf{to}xr\textbf{o}j\\ 
 \hline
 A\textcrh na & xraj\textbf{naa} & \textbf{no}x\textbf{o}rj\textbf{uu}\\ 
 \hline
 Entuuma & xraj\textbf{tuu} & \textbf{to}x\textbf{o}rj\textbf{uu}\\ 
 \hline
 Huuma & xarj\textbf{uu} & \textbf{yo}x\textbf{o}rj\textbf{uu}\\ 
 \hline
\end{tabularx}
\end{table}

Voici quelques clés de lecture du tableau ci-dessus (les points que j'ai évoqués dans le paragraphe précédent s'appliquent encore) :

\begin{itemize}
    \item Au \textbf{passé}, la conjugaison est strictement \textbf{identique} que pour les verbes du \textbf{premier} groupe;
    \item Cependant au \textbf{présent}, la voyelle change, et se transforme en \textbf{o}. 
\end{itemize}

Une question qui se pose naturellement est la suivante : à partir de la racine uniquement, ou à partir de la forme sous laquelle le verbe est généralement présenté (conjugué avec huwwa au passé), comment peut-on savoir si un groupe appartient au premier ou au deuxième groupe ? 

La réponse est malheureusement décevante : en arabe standard, ces verbes ne se distinguent pas, et il faut apprendre par c\oe ur avec quelle voyelle accorder chaque verbe. Historiquement, il devait peut-être y avoir une raison particulière, un environnement consonantique particulier qui a induit un changement de voyelle, ou une justification sémantique basée sur des schèmes. Mais tout ceci a dû se perdre avec le temps (ou ne fait en tout cas plus partie du savoir commun).

Le tunisien ne fait pas plus d'effort que sa langue-mère sur ce point. Dans la suite de ce cours, je vais essayer de vous présenter, lorsque cela est nécessaire, les verbes sous une forme qui ne laisse pas planer d'ambiguïté (conjugué avec huwwa au présent par exemple).

\textbf{Note :} Vous pourrez entendre certains locuteurs prononcer la conjugaison au présent pour les personnes plurielles avec \textbf{deux} syllabes plutôt que \textbf{trois}, comme je l'ai marqué au tableau ci-dessus. Dans ce cas, c'est la voyelle centrale qui n'est pas prononcée, et les deux première syllabes qui sont fusionnées.

\subsection{Verbes du troisième groupe}\label{ConjSS43}
Ce groupe correspond en arabe aux verbes qui se conjuguent au présent avec une \textbf{kasra}.

La conjugaison pour le verbe \textbf{fhem} (comprendre) est la suivante (\textbf{en gras} les préfixes et terminaisons liés à chaque pronom et temps) :

\begin{table}[ht]
\begin{tabularx}{\textwidth}{||X | X | X||}
 \hline
 Pronom & Passé & Présent \\
 \hline\hline
 Éna & fhem\textbf{t} & \textbf{ne}fhem \\
 \hline
 Enti & fhem\textbf{t} & \textbf{te}fhem\\ 
 \hline
 Huwwa & fhem & \textbf{ye}fhem\\ 
 \hline
 Hiyya & fehm\textbf{et} & \textbf{te}fhem\\ 
 \hline
 A\textcrh na & fhem\textbf{naa} & \textbf{ne}fehm\textbf{uu}\\ 
 \hline
 Entuuma & fhem\textbf{tuu} & \textbf{te}fehm\textbf{uu}\\ 
 \hline
 Huuma & fehm\textbf{uu} & \textbf{ye}fehm\textbf{uu}\\ 
 \hline
\end{tabularx}
\end{table}

Voici quelques clés de lecture du tableau ci-dessus (les points que j'ai évoqués dans le paragraphe précédent s'appliquent encore) :

\begin{itemize}
    \item Au \textbf{passé}, la conjugaison est strictement \textbf{identique} que pour les verbes du \textbf{premier et deuxième} groupe;
    % TODO : Voir pour supprimer cette partie qui n'est pas évidente
    \item Cependant au \textbf{présent}, pour les personnes plurielles, les formes avec \textbf{2} et \textbf{3} syllabes coexistent, et leur prévalence dépendent majoritairement des consonnes du verbe conjugué et de l'usage. 
\end{itemize}

Un avantage majeur de ce groupe est sa démarcation claire au niveau vocalique avec les verbes du premier et deuxième groupe : si vous voyez un \textbf{e} dans une forme conjuguée, vous saurez que c'est un verbe du troisième groupe et pas autre chose ! 

% TODO : voir pour supprimer cette partie ? Porte à confusion
La difficulté réside dans le nombre de syllabes qu'il faut donner pour les trois personnes plurielles au présent. \textbf{Sémantiquement, il n'y pas de réel problème ici :} un tunisophone vous comprendra quoi qu'il arrive. Mais l'usage d'une forme peu employée fera lever plus d'un sourcil. Par exemple :
 \begin{itemize}
     \item Pour \textbf{f\v{s}el} (se fatiguer), on dira systématiquement \textbf{nef\v{s}luu} et non \textbf{nefe\v{s}luu};
     \item Pour \textbf{fhem} (comprendre), on dira plutôt \textbf{nefehmuu}, le préférant à \textbf{nefhmuu};
     \item Pour \textbf{mlek} (posséder), on pourra employer alternativement \textbf{nemelkuu} ou \textbf{nemlkuu}.
 \end{itemize}

Pour être totalement honnête, je n'ai pas encore réussi à trouver de règles fiables qui permettent de séparer l'un ou l'autre des cas. Mon conseil est le suivant : \textbf{utilisez la forme qui vous demande le moins d'effort pour être prononcée}, c'est généralement une bonne façon de discriminer l'une des deux formes.

\section*{Vocabulaire}
Dans cette partie, je vous donne quelques phrases avec des verbes conjugués, appartenant à l'un des trois groupes que nous avons vus. 

TODO : Rajouter des exemples dans cette partie 
