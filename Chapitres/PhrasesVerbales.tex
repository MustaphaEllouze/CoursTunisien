\chapter{Introduction aux phrases verbales}
\chapterletter{I}ntéressons nous maintenant aux phrases verbales, qui constituent avec les phrases nominales, les deux structures possibles pour une phrase en tunisien.

\section{Un peu d'histoire}
Tout comme les phrases nominales, les phrases verbales prennent leurs racines dans l'arabe standard. Cette structure correspond à la structure d'une phrase classique qu'on pourrait retrouver dans les autres langues internationales.

Comme son nom l'indique, cette structure comprend nécessairement un \textbf{verbe}, qui servira à décrire une action. Ainsi, cette structure s'oppose directement à la structure de la phrase nominale, qui est elle nécessairement descriptive.

En \textbf{arabe standard}, la structure d'une phrase verbale est \textbf{différent} de celle du \textbf{tunisien}. En effet, l'arabe standard, comme environ 10\% des langues dans le monde, est une \textbf{VSO}, c'est-à-dire qu'elle obéit à la structure \textbf{verbe-sujet-objet}.

\textbf{\textsc{Note} :} Contrairement à ce que pensent certains, ce n'est pas la structure \textbf{SVO (sujet-verbe-objet, 42\%)} qui est la plus répandue, mais bien la structure \textbf{SOV (sujet-objet-verbe, 45\%)} !

Vous verrez dans la suite de ce cours que cette structure historique \textbf{VSO} garde encore des traces jusqu'à maintenant en tunisien. Il est donc utile de la garder en tête.

\section{Structure des phrases verbales}
Contrairement à l'arabe standard, le \textbf{tunisien} arbore une structure \textbf{SVO (sujet-verbe-objet)}, qui est donc la même structure que le langues latines, ou l'anglais.

\begin{table}[ht]
\begin{tabularx}{\textwidth}{||X | X | X||}
 \hline
 Tunisien & Français & Traduction littérale \\ [2.5ex] 
 \hline\hline
 Lemra tékel xobz  & \textit{La femme mange du pain} & \textit{La femme mange pain}\\ 
 \hline
 Leq\c{t}aate\c{s} i\textcrh ebbuu el\textcrh uut  & \textit{Les chats aiment le poisson} & \textit{Les chats aiment le poisson}\\ 
 \hline
 Enti taqraa ktééb  & \textit{Tu lis un livre} & \textit{Tu lis livre}\\ 
 \hline
\end{tabularx}
\end{table}


\section{Omission des pronoms personnels}
Il n'y pas grand chose à retenir de plus sur la structure des phrases verbales. Cependant, je dirais quand même quelques mots sur \textbf{l'omission des pronoms personnels}.

Comme c'est le cas de l'italien, l'espagnol et de l'arabe standard, le tunisien s'autorise à \textbf{omettre les pronoms personnels dans les phrases verbales}. Dans ce cas, c'est la \textbf{conjugaison du verbe} qui indique quelle personne est visée.

Cette habitude a directement été héritée de l'arabe. A l'instar de l'espagnol et de l'italien que j'ai cités plus haut, \textbf{l'arabe standard distingue précisément les conjugaisons} de chaque personne à tous les temps. \textbf{Ce n'est cependant pas le cas du tunisien}, ou le temps a suffisamment érodé la prononciation de certaines formes conjuguées pour que certaines d'entre elles soient maintenant indistinguable.

Fort heureusement, le contexte permet de retrouver quasiment systématiquement la personne à laquelle la phrase se réfère. En cas de doute, les tunisophones n'hésitent alors pas à préciser le pronom personnel en question au lieu de l'omettre et risquer la confusion.

\begin{table}[ht]
\begin{tabularx}{\textwidth}{||X | X | X||}
 \hline
 Tunisien & Français & Traduction littérale \\ [2.5ex] 
 \hline\hline
 N\textcrh ebb elkuura & \textit{J'aime le foot} & \textit{(J') Aime le ballon}\\ 
 \hline
 To\v{s}orbuu & \textit{Vous buvez} & \textit{Buvez}\\ 
 \hline
 Taqraa majalla  & \textit{Tu lis / Elle lit un magazine} & \textit{lis / lit magazine}\\ 
 \hline
\end{tabularx}
\end{table}

\section*{Vocabulaire}
\begin{table}[ht]
\begin{tabularx}{\textwidth}{||X | X | X||}
 \hline
 Vocabulaire & Traduction & Origine \\
 \hline\hline
 mra (fem.) / nsé' (plu.) & femme & (\textsc{ar}) \RL{امراة / نساء} \\
 \hline
 yéékel (verbe) & manger & (\textsc{ar}) \RL{أكل} \\
 \hline
 xobz (masc.) & pain & (\textsc{ar}) \RL{خبز} \\
 \hline
 qa\c{t}\c{t}uu\c{s} (masc.) / q\c{t}aate\c{s} (plu.) & chat & (\textsc{ar}) \RL{قطّ} \\
 \hline
 i\textcrh ebb (verbe) & aimer/vouloir & (\textsc{ar}) \RL{حبّ} \\
 \hline
 \textcrh uut (masc.) & poisson & (\textsc{ar}) \RL{حوت} (baleine) \\
 \hline
 yaqraa (verbe) & lire / étudier & (\textsc{ar}) \RL{قرأ} \\
 \hline
\end{tabularx}
\end{table}

\begin{table}[ht]
\begin{tabularx}{\textwidth}{||X | X | X||}
 \hline
 Vocabulaire & Traduction & Origine \\
 \hline\hline
 ktééb (masc.) & livre & (\textsc{ar}) \RL{كتاب} \\
 \hline
\end{tabularx}
\end{table}
