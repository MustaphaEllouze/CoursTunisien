\chapter{Introduction aux phrases nominales}
\chapterletter{D}ans ce chapitre, nous allons parler de la construction des phrases nominales en tunisien, qui occupe une plus grande place dans la grammaire que les phrases nominales dans les langues latines et germaniques. 

\section{Un peu d'histoire}
La structure des phrases en tunisien est directement héritée de l'arabe standard. Chez la langue-mère du tunisien, \textbf{deux types} de structures sont possibles :
\begin{itemize}
    \item Les phrases \textbf{verbales} : elles comprennent sujet, verbe et compléments éventuels;
    \item Les phrases \textbf{nominales} : elles se caractérisent par l'absence de verbe, mais cela ne veut pas dire qu'elles sont pauvre en sens.
\end{itemize}

Une des caractéristiques qui marquent souvent le plus chez les nouveaux apprenant de l'arabe est l'absence de verbes qui peuvent paraître essentiels dans d'autres langues, notamment les verbes \textbf{être} et \textbf{avoir}. L'arabe, et le tunisien au même titre, se permettent d'échapper à cette nécessité en se reposant sur une structure grammaticale plus rigide. 

Ainsi, vous l'aurez compris, \textbf{les phrases nominales} servent dans tous les langues arabes à former des phrases \textbf{descriptives} et des \textbf{constats}, puisqu'il manquera systématiquement un verbe décrivant l'action. Quelques adverbes permettent néanmoins de véhiculer un sens plus nuancé, mais cela ne relève pas de ce cours. 

\section{Structure des phrases nominales}
Une phrase nominale en tunisien est constituée de \textbf{deux} constituants principaux :
\begin{itemize}
    \item \textbf{Le sujet} : C'est l'élément sur lequel une information sera donnée;
    \item \textbf{Un complément} : C'est l'information qui est donnée sur le sujet. En arabe, on l(appelle "l'information".
\end{itemize}

Pour former la phrase, \textbf{il suffit de juxtaposer le sujet et le complément}.

\textbf{Exemples :}
\begin{center}
 \begin{tabular}{||c | c | c||} 
 \hline
 Tunisien & Français & Traduction littérale \\ [2.5ex] 
 \hline\hline
 Esmi Mo\c{s}\c{t}fa &\textit{Mon prénom est Mostafa}  & \textit{Prénom-mon Mostafa} \\ 
 \hline
 Es-smèè' zarqa &\textit{Le ciel est bleu}  & \textit{Le ciel bleu} \\ 
 \hline
 E\c{t}-\c{t}aq\c{s} sxuun &\textit{Il fait chaud}  & \textit{Le temps chaud} \\ 
 \hline
 E\c{t}-\c{t}aawla \c{a}aalya &\textit{La table est haute}  & \textit{La table haute} \\ 
 \hline
 Enti bnayya &\textit{Tu es une fille}  & \textit{Toi fille} \\ 
 \hline
 Huwwa raajel &\textit{Il est un homme}  & \textit{Lui homme} \\ 
 \hline
\end{tabular}
\end{center}

Si vous avez l'\oe il affûté, vous aurez déjà remarqué un point commun dans toutes ses phrases : \textbf{dans toutes les phrases nominales, le sujet est nécessairement sous une forme définie}. On dira bien : \underline{le} temps, \underline{la} table, \underline{mon} prénom, \underline{toi}, etc. Et à l'inverse, \textbf{le complément sera nécessairement sous la forme indéfinie}. On peut donc facilement faire la distinction entre chaque groupe grammatical.

Vous aurez aussi remarqué que le complément peut être de différentes natures : nom commun, adjectif ou encore nom propre.

\section{Pronoms personnels}
Puisqu'ils peuvent vous aider à former vos propres phrases, parlons rapidement des \textbf{pronoms personnels}. Ils prennent tous racine dans l'arabe standard.

\begin{center}
    \begin{tabular}{||c | c | c||}
    \hline
        \textbf{Tunisien} & \textbf{Français} & \textbf{Arabe standard} \\ [2.5ex] 
        \hline\hline
        \je & Je / Moi & \RL{انا}\\ \hline
        \tu & Tu / Toi & \RL{انت}\\ \hline
        \il & Il / Lui & \RL{هو}\\ \hline
        \elle & Elle & \RL{هي}\\ \hline
        \nous & Nous & \RL{نحن}\\ \hline
        \vous& Vous & \RL{انتم}\\ \hline
        \ils & Ils / Elles / Eux & \RL{هم}\\ \hline
    \end{tabular}
\end{center}

Ils ont beaucoup évolué depuis l'arabe standard, le tunisien perdant en tout plus de \textbf{6} pronoms personnels :
\begin{itemize}
    \item Toutes les \textbf{formes duelles} (3 au total) :  ce sont les pronoms qui qualifient exactement deux personnes (il y en avait une pour la deuxième personne, et deux pour la troisième personne).
    \item Tous les \textbf{pluriels féminins} (2 au total) : ces pronoms correspondent aux groupes constitués entièrement de sujets féminins, à la deuxième et troisième personne.
    \item Le pronom \textbf{féminin singulier de la deuxième personne} (1 au total) : cela correspond au pronom "tu" accordé au féminin. Il est quand même intéressant de noter que certains dialectes du tunisien ont conservé ce pronom (ce cours ne couvrira pas ces formes-là).
\end{itemize}

\section{Quelques variations}
Il est tout à fait possible d'agrémenter le sujet avec d'autres adjectifs ou des démonstratifs par exemple. Dans ce cas-là, le complément sera toujours composé d'un \textbf{mot unique}, et tout le reste des mots formeront le \textbf{groupe sujet}.

\textbf{Exemples :}


\begin{center}
    \begin{tabular}{||c | c | c||}
        \hline
        \textbf{Tunisien} & \textbf{Français} & \textbf{\makecell{Traduction \\littérale}} \\ [2.5ex] 
        \hline\hline
        \makecell{Héé\dh i el-karehba\\ \textcrh amra.} &\textit{\makecell{Cette voiture-ci \\est rouge.}}  & \textit{\makecell{Cette la voiture \\rouge.}} \\ 
        \hline
        \makecell{Héé\dh a er-raajel\\ \c{t}wiil.} &\textit{\makecell{Cet homme \\est grand.}}  & \textit{\makecell{Ce le homme\\ grand.}} \\ 
        \hline
        \makecell{El-kosksi et-tuunsi\\ bniin.} &\textit{\makecell{Le couscous tunisien\\ est bon.}}  & \textit{\makecell{Le couscous le \\tunisien bon.}} \\ 
        \hline
        \makecell{Héé\dh a l-ordinater e\c{s}-\c{s}\v{r}iir \\w el-\v{r}aali ak\textcrh al.} &\textit{\makecell{Ce petit ordinateur \\cher est noir.}}  & \textit{\makecell{Ce l'ordinateur petit \\et cher noir.}} \\ 
        \hline
    \end{tabular}
\end{center}

Dans les exemples ci-dessus, le complément est systématiquement le dernier mot de la phrase et \textbf{sous forme indéfinie}. Tout le reste des mots forme le sujet. Sémantiquement, cela suppose que toutes les informations \textbf{en dehors du complément} sont soit déjà connues, soit moins importantes.

\begin{minipage}{10cm}

\section*{Vocabulaire}
\begin{center}
    \begin{tabular}{||c | c | c||}
        \hline
        Vocabulaire & Traduction & Origine \\\hline\hline
        'esm (masc.) & prénom / nom & (\textsc{ar}) \RL{اسم} \\\hline
        smèè' (fem.) & ciel & (\textsc{ar}) \RL{سماء} \\\hline
        azraq / zarqa (adj.) & bleu & (\textsc{ar}) \RL{أزرق} \\\hline
        \c{t}aq\c{s} (masc.) & temps/météo & (\textsc{ar}) \RL{طقس} \\\hline
        sxuun / sxuuna (adj.) & chaud & (\textsc{ar}) \RL{ساخن} \\\hline
        \c{t}aawla (fem.) & table & (\textsc{ar}) \RL{طاولة} \\\hline
        \c{a}aali / \c{a}aalya (adj.) & haut & (\textsc{ar}) \RL{عالي} \\\hline
        bnayya (fem.) & fille & (\textsc{ar}) \RL{ابن} (fils) \\\hline
        raajel (masc.) & homme & (\textsc{ar}) \RL{راجل} \\\hline
        héé\dh a / héé\dh i (démons.) & celui-ci / celle-ci & (\textsc{ar}) \RL{هذا / هذه} \\\hline
        a\textcrh mar / \textcrh amra (adj.) & rouge & (\textsc{ar}) \RL{أحمر} \\\hline
        karehba (fem.) & voiture & (\textsc{ar}) \RL{كهرباء} (électricité) \\\hline
        \c{t}wiil / \c{t}wiila (adj.) & grand (en taille) & (\textsc{ar}) \RL{طويل / طويلة} \\\hline
        tuunsii / tuunsiyya (adj.) & tunisien/tunisienne & (\textsc{ar}) \RL{تونسي / تونسية} \\\hline
        bniin / bniina (adj.) & bon/bonne (en goût) & --- \\\hline
        ordinater (masc.) & ordinateur & (\textsc{fr}) ordinateur \\\hline
        \c{s}\v{r}iir / \c{s}\v{r}iira (adj.) & petit / petite & \textsc{ar} \RL{صغير / صغيرة} \\\hline
        \v{r}aali / \v{r}aalya (adj.) & cher / chère (en prix) & (\textsc{ar}) \RL{غالي / غالية} \\\hline
        ak\textcrh al / ka\textcrh laa (adj.) & noire / noire & (\textsc{ar}) \R{اكحل} \\\hline
    \end{tabular}
\end{center}

\end{minipage}


