\chapter{Transcription adoptée et prononciation}
\chapterletter{A}vant même de commencer à parler de grammaire, il est utile de parler de deux choses essentielles : 
\begin{itemize}
    \item L'ensemble des \textbf{sons} qui existent en tunisien, et comment les prononcer ;
    \item La \textbf{transcription} qui sera utilisée dans le cours pour représenter ces sons.
\end{itemize}

Je vous propose de parcourir ensemble ces deux points dans ce chapitres.

\section{Sons que vous pouvez rencontrer en tunisien}
Lorsqu'on parle de \textbf{sons}, on ne parle pas des lettres qui composent l'alphabet, mais bien de l'ensemble des phonèmes qu'un locuteur peut prononcer et sait distinguer. Ainsi, si l'alphabet français comprend \textbf{5 voyelles}, un locuteur français peut en réalité prononcer jusqu'à \textbf{20 phonèmes vocaliques} différents (en fonction du dialecte et de l'accent) !

L'inventaire phonétique \textbf{consonantique} du tunisien est particulièrement riche. Ainsi, on dit souvent en Tunisie que lorsqu'on parle tunisien, ou une autre langue d'origine arabe en général, il est relativement aisé de prononcer correctement la plupart des consonnes du français, des langues latines en général et de l'anglais. 

À l'inverse, l'inventaire phonétique \textbf{vocalique} de l'arabe étant relativement pauvre, celui du tunisien l'est également, notamment quand on le compare à celui du français. On comprend alors facilement pourquoi les locuteurs d'origine arabe ont souvent beaucoup de mal avec certaines voyelles dont ils n'ont pas l'habitude. L'exemple parfait de ce phénomène est le mot \textit{électricité}, où un tunisophone prononcera toutes les voyelles comme des \textbf{/i/}.

\subsection{Consonnes d'origine arabe}
Les consonnes que vous pouvez retrouver en tunisien ont pour la plupart une origine arabe\footnote{Certaines des autres ont fait leur apparition via des mots d'origine étrangère et n'apparaissent donc exclusivement que dans des mots importés.}. 

Vous trouverez dans le tableau suivant l'ensemble des consonnes qui ont été passées de l'arabe au tunisien. J'y liste trois informations :
\begin{itemize}
    \item La \textbf{transcription arabe}, qui est la lettre correspondante dans l'alphabet arabe\footnote{Techniquement, il s'agit d'un \textbf{abjad}, mais je ne ferai pas la distinction dans le reste du cours.} ;
    \item La \textbf{transcription phonétique}, qui correspond à la représentation de ce son dans \textbf{l'alphabet phonétique international}\footnote{Pas d'inquiétude, je détaille plus loin dans cette partie comment prononcer chaque son.};
    \item La \textbf{romanisation} usuelle, qui correspond aux caractères qu'on emploie habituellement pour représenter ce son (\underline{mais} qui ne sera pas nécessairement la transcription que j'utilise dans ce cours\footnote{Ces romanisations ont diverses origines, notamment l'écriture en alphabet latin des prénoms et noms d'origine arabe, mais aussi les échanges informels entre tunisophones (par SMS par exemple), faute d'avoir accès aux caractère d'origine arabe.}). Retenir cette romanisation n'a que peu d'importance pour la suite, je ne la donne qu'à titre informatif.
\end{itemize}

\begin{center}
\begin{tabular}{||c c c||} 
 \hline
 \textbf{\makecell{Transcription\\ arabe}} & \textbf{\makecell{Transcription \\ phonétique}} & \textbf{\makecell{Romanisation\\usuelle}}\\ [2.5ex] 
 \hline\hline
 \RL{ا} / \RL{ء} & ['] & ' \\ 
 \hline
 \RL{ب} & [b] & b \\ 
 \hline
 \RL{ت} & [t] & t \\ 
 \hline
 \RL{ث} & [\texttheta] & th \\ 
 \hline
 \RL{ج} & [\textyogh] & j \\ 
 \hline
 \RL{ح} & [h] & h / 7 \\ 
 \hline
 \RL{خ} & [\textchi] & kh / 5 \\ 
 \hline
 \RL{د} & [d] & d \\ 
 \hline
 \RL{ذ} & [\dh] & dh / 4 \\ 
 \hline
 \RL{ر} & [r] & r \\ 
 \hline
 \RL{ز} & [z] & z \\ 
 \hline
 \RL{س} & [s] & s \\ 
 \hline

\end{tabular}
\end{center}

\begin{center}
\begin{tabular}{||c c c||} 
\hline
\textbf{\makecell{Transcription\\ arabe}} & \textbf{\makecell{Transcription \\ phonétique}} & \textbf{\makecell{Romanisation\\usuelle}}\\ [2.5ex] 
\hline\hline

\RL{ش} & [\textesh] & ch / sh \\ 
\hline
\RL{ص} & [s\super \textrevglotstop] & s \\ 
\hline
 \RL{ض}/\RL{ظ} & [\dh \super \textrevglotstop] & dh / 4 \\ 
 \hline
 \RL{ط} & [t\super \textrevglotstop] & t \\ 
 \hline
 \RL{ع} & [\textrevglotstop] & aa / \textroundcap{a} / 3 \\ 
 \hline
 \RL{غ} & [\textinvscr] & gh / 8 \\ 
 \hline
 \RL{ف} & [f] & f \\ 
 \hline
 \RL{ق} & [q] & q / 9 \\ 
 \hline
 \RL{ك} & [k] & c / k \\ 
 \hline
 \RL{ل} & [l] & l \\ 
 \hline
 \RL{م} & [m] & m \\ 
 \hline
 \RL{ن} & [n] & n \\ 
 \hline
 \RL{ه} & [\texthth] & h \\ 
 \hline
 \RL{و} & [w] & w \\ 
 \hline
 \RL{ي} & [y] & y \\  [2.5ex] 
 \hline
\end{tabular}
\end{center}

Pour fabriquer le tableau ci-dessus, j'ai simplement repris dans son intégralité \textbf{l'alphabet arabe}, et listé chacune des lettres qui le composent. Il semble logique de conclure que la phonologie du \textbf{tunisien} est très proche de celle de l'\textbf{arabe}, ce qui serait totalement justifié \footnote{Je dirais qu'il y a une seule grande différence majeure qui la prononciation des deux consonnes \RL{ض} et \RL{ظ}, que j'ai disposées sur la même ligne du tableau. Au fur des années, les prononciations de ces deux se sont rapprochées, jusqu'à devenir identique (du moins dans l'accent \textit{standard} tunisien, qui est celui entendu à la télé et à la radio). Dans certaines parties de la Tunisie, les anciennes prononciations de ces consonnes peuvent subsister.}!

Parcourons ensemble chacun de ces sons, et détaillons sa prononciation. Quelques éléments de langage avant toute chose :
\begin{itemize}
    \item Consonnes \textbf{voisées} et consonnes \textbf{sourdes} - Certains sons dans les langues du monde se distinguent uniquement par la vibration de vos cordes vocales. 
    \begin{itemize}
        \item Une consonne est dite \textbf{sourde} si les cordes vocales ne sont pas sollicitées pour la prononcer (\textbf{/t/}, \textbf{/p/}, \textbf{/s/}).
        \item À l'inverse, une consonne est dite \textbf{voisée} si les cordes vocales sont sollicitées (\textbf{/d/}, \textbf{/b/}, \textbf{/z/}).
        \item Certaines consonnes vont par pair, une version \textbf{voisée} et une version \textbf{sourde}. Par exemple : \textbf{/z/} et \textbf{/s/}, \textbf{/b/} et \textbf{/p/}, \textbf{/d/} et \textbf{/t/}.
    \end{itemize}
    \item Consonnes \textbf{emphatiques} (consonnes \textbf{pharyngalisées}) - C'est une \textit{technique} de prononciation spécifique aux langues sémitiques. Il s'agit de modifier subtilement la prononciation d'un son qui existe par ailleurs. Quand on parlera plus bas de consonnes \textbf{emphatique}, faites ceci (ce n'est pas un exercice très facile) :
    \begin{itemize}
        \item Contractez votre \textbf{pharynx} ; 
        \item Rapprochez l'arrière de votre langue de votre \textbf{luette}.
    \end{itemize}
    Pour le son \textbf{/s/}, le meilleur exemple que je puisse vous donner par écrit est de penser au mot \textit{ça} prononcé à la \textbf{québecoise}.
    \item \textbf{Environnement consonantique} - Ce sont les \textit{autres} consonnes qui entourent le son duquel on parle.
\end{itemize}

C'est parti ! Faisons le tour de toutes ces consonnes-ci \footnote{N'hésitez pas à revenir de temps en temps dans cette section, certains sons sont très particuliers et spécifiques aux langues d'origine arabe. Il n'est pas impossible que vous ne réussissiez pas du premier coup.}!

\begin{description}
    \item[Prononciation de \RL{ء} /'/ -] 
    
    Ce son correspond au \textbf{/h/ aspiré} en français. On peut l'entendre par exemple entre les deux mots dans \textbf{les haricots} : la liaison n'est pas faite et une légère pause est marquée; on ne dira ni \textbf{/lézarico/} ni \textbf{/léaricot/}, mais bien \textbf{/lé arico/}. C'est ce qui fait aussi la différence entre \textbf{/Léa/} et \textbf{/Les "A"/}.
    
    \item[Prononciation de \RL{ب} /b/ -] 
    
    Ce son se prononce comme le \textbf{/b/} en français, comme dans les mots \textbf{bébé} ou \textbf{bateau}.

    \item[Prononciation de \RL{ت} /t/ -] 
    
    Ce son se prononce comme le \textbf{/t/} en français, comme dans les mots \textbf{tuyau} ou \textbf{table}.

    \item[Prononciation de \RL{ث} /\texttheta/ -] 
    
    Ce phonème se retrouve en anglais avec la retranscription \textbf{/th/}, comme dans \textbf{thorn} ou \textbf{thin}. Pour prononcer ce son, il suffit de prononcer un \textbf{/s/} avec la langue coincée entre les dents. Il s'agit de la verison \textbf{sourde} du \textbf{/th/} dans le mot \textbf{then}.
    
    \item[Prononciation de \RL{ج} /\textyogh/ -] 
    
    Derrière ce symbole phonétique se cache simplement le son \textbf{/j/} comme on peut le retrouver en français dans les mots \textbf{jeu} et \textbf{girouette}.

    \item[Prononciation de \RL{ح} /h/ -] 
    
    Ce son correspond à la version \textbf{sourde} du son \textbf{/h/} dans les mots anglais \textbf{hungry} ou \textbf{high}. Une façon de réaliser cette consonne est de s'imaginer expirer de l'air sur sa main, comme si on voulait sentir son haleine.

    \item[Prononciation de \RL{خ} /\textchi/ -]

    C'est un son qui peut sembler difficile à prononcer pour un francophone, car il n'est prononcé en français que dans un environnement consonantique assez particulier. On peut le retrouver lorsqu'on prononce le son \textbf{/r/} lorsqu'il suit les sons \textbf{/k/} et \textbf{/t/}, comme par exemple dans les mots \textbf{crin} et \textbf{train}. C'est également la pronociation de la lettre \textbf{j} en \textbf{espagnol}.

    \item[Prononciation de \RL{د} /d/ -] 
    
    Ce son se prononce comme le \textbf{/d/} en français, comme dans les mots \textbf{décoration} ou \textbf{diminuer}.

    \item[Prononciation de \RL{ذ} /\dh/ -] 
    
    Il est la version \textbf{voisée} du phonème \RL{ث} [\texttheta]. A ce titre, on le retrouve dans les mots anglais \textbf{the} ou \textbf{then}. Pour prononcer ce son, il suffit de prononcer un \textbf{/z/} avec la langue coincée entre les dents.

    \item[Prononciation de \RL{ر} /r/ -]
    
    C'est le son qui correspond à ce qu'on désigne par \textbf{r battu} en français. On le retrouve en espagnol, dans le mot \textbf{pero (mais)}. Si vous avez des difficultés pour le prononcer, faites en sorte que la pointe de votre langue touche une fois (et une seule fois seulement) votre palais (zone \textbf{alvéolaire} de la bouche), très rapidement, tout en insufflant de l'air dans la zone \textbf{centrale} de la bouche\footnote{Attention, si vous faites circuler l'air dans la zone \textbf{latérale} de la bouche, vous prononcerez un \textbf{/l/}.}.

    \item[Prononciation de \RL{ز} /z/ -] 
    
    Ce son se prononce comme le \textbf{/z/} en français, comme dans les mots \textbf{zèbre} ou \textbf{zoo}.

    \item[Prononciation de \RL{س} /s/ -]

    Ce son se prononce comme le \textbf{/s/} en français, comme dans les mots \textbf{sauter} ou \textbf{salade}.

    \item[Prononciation de \RL{ش} /\textesh/ -]

    Ce son se prononce comme le \textbf{/ch/} en français, comme dans les mots \textbf{cheval} ou \textbf{chute}.

    \item[Prononciation de \RL{ص} /s\super \textrevglotstop/ -]

    Ce son est la version \textbf{emphatique} du son \textbf{/s/}. Vous pouvez penser à la prononciation du mot \textbf{ça} en québecois.

    \item[Prononciation de \RL{ض}/\RL{ظ} /\dh \super \textrevglotstop/ -]

    Ce son correspond à la version \textbf{emphatique} du son \textbf{[\dh]} vu plus haut. De la même manière, vous pouvez penser à la prononciation du mot \textbf{ça} en québecois, en remplaçant la consonne par celle qui va bien.

    \item[Prononciation de \RL{ط} /t \super \textrevglotstop/ -]

    Ce son correspond à la version \textbf{emphatique} du son \textbf{/t/}. De la même manière, vous pouvez penser à la prononciation du mot \textbf{ça} en québecois, en remplaçant la consonne par celle qui va bien.


\end{description}




\textbf{| Prononciation de \RL{ع} [\textrevglotstop]}

Ce phonème correspond à la lettre arabe \RL{ع}. Le son n'est pas systématiquement reconnu comme étant une \textbf{consonne emphatique} mais c'est bien le son désigné par \RL{ع} [\textrevglotstop] qui sert de base à la prononciation des autres consonnes emphatiques. Par ailleurs, en linguistique, on parlera de \RL{ع} comme d'une consonne spirante ou d'une approximante ;  alors qu'en dehors de la sphère linguistique, on parlera de semi-voyelle comme \textbf{/w/} ou \textbf{/y/}, même si d'un point de vue grammatical l'arabe ne considère pas \RL{ع} comme telle. Afin de prononcer correctement ce son, il convient de \textbf{contracter le pharynx} tout en \textbf{rapprochant l'arrière de votre langue de votre luette}. Vu la complexité de la tâche et le fait que le son ne soit accompagné d'aucun autre (voir la prononciation de \RL{ص} [s\super \textrevglotstop] par exemple), il fait parti des deux phonèmes avec lequel les francophones ont le plus de mal\footnote{Une manière pas très jolie de vous aider à prononcer cette consonne est de vous souvenir du son que vous avez produit avec votre gorge la dernière fois que vous avez abusé de l'alcool, ou manger des fruits de mer pas frais.}.


\textbf{| Prononciation de \RL{غ} [\textinvscr]}

Ce phonème correspond à la lettre arabe \RL{غ}. C'est un son qui peut sembler très familier pour un francophone, car il ressemble très fortement au \textbf{/r/} du français. Cependant, la prononciation exacte est légèrement différente, car il n'est prononcé en français que dans un contexte consonantique assez particulier. On peut le retrouver lorsqu'on prononce le son \textbf{/r/} lorsqu'il suit les phonèmes \textbf{/g/} et \textbf{/d/}, comme par exemple dans les mots \textbf{grain} et \textbf{drain}. Contrairement à la croyance générale, les réalisations de ces \textbf{/r/} diffèrent du "r classique" : essayez de prononcer le mot \textbf{rein} par exemple\footnote{Ce son n'est en réalité que la version sonore de \RL{خ} [\textchi].}\footnote{En pratique, vous pouvez vous contenter de prononcer ce son comme le \textbf{/r/} de \textbf{rein}, un tunisophone ne fera pas nécessairement la différence, d'autant plus que le son \textbf{/r/} du français n'existe pas en tunisien.}.

\textbf{| Prononciation de  \RL{ف} [f]}

Ce phonème correspond à la lettre arabe \RL{ف}. Il est l'un des phonèmes les plus répandus dans les langues internationales, et se prononce comme le \textbf{/f/} en français, comme dans les mots \textbf{faire} ou \textbf{foin}.


\textbf{| Prononciation de  \RL{ق} [q]}

Ce phonème correspond à la lettre arabe \RL{ق}. Avec \RL{ع}, il constitue facilement le phonème avec lequel les non-arabophones ont le plus de mal. Sa prononciation se fait similairement au son \textbf{/k/}, sauf que les deux parties de votre bouche qui doivent être en contact sont \textbf{l'arrière de votre langue et votre luette}\footnote{Afin de vous aider à prononcer ce son, vous pouvez vous allonger tête vers le haut, et essayer de reproduire le son \textbf{/k/} : la gravité fera naturellement en sorte que votre langue se rapproche de votre luette. En cas de difficulté, essayez d'imaginer le son que quelqu'un produit lorsqu'il ronfle, qu'il se bloque la respiration vers l'arrière de la bouche, et qu'il essaie quand même d'expirer.}.


\textbf{| Prononciation de  \RL{ك} [k]}

Ce phonème correspond à la lettre arabe \RL{ك}. Il est l'un des phonèmes les plus répandus dans les langues internationales, et 

\textbf{| Prononciation de  \RL{ل} [l]}

Ce phonème correspond à la lettre arabe \RL{ل}. Il est l'un des phonèmes les plus répandus dans les langues internationales, et se prononce comme le \textbf{/l/} en français, comme dans les mots \textbf{lumière} ou \textbf{livre}.


\textbf{| Prononciation de  \RL{م} [m]}

Ce phonème correspond à la lettre arabe \RL{م}. Il est l'un des phonèmes les plus répandus dans les langues internationales, et se prononce comme le \textbf{/m/} en français, comme dans les mots \textbf{montre} ou \textbf{manteau}.


\textbf{| Prononciation de  \RL{ن} [n]}

Ce phonème correspond à la lettre arabe \RL{ن}. Il est l'un des phonèmes les plus répandus dans les langues internationales, et se prononce comme le \textbf{/n/} en français, comme dans les mots \textbf{notre} ou \textbf{niveau}.


\textbf{| Prononciation de \RL{ه} [\texthth]}

Ce phonème correspond à la lettre arabe \RL{ه}. Il correspond à ce qui est appelé \textbf{h expiré}, et apparaît dans des onomatopées et interjection en français, comme dans \textbf{hé !}. On rencontre ce son en anglais également, dans des mots comme \textbf{heavy} ou \textbf{hallelujah}.


\textbf{| Prononciation de \RL{و} [w]}

Ce phonème correspond à la lettre arabe \RL{و}. Il correspond au son \textbf{/w/} en français, comme dans les mots \textbf{wasabi} ou \textbf{web}.


\textbf{| Prononciation de \RL{ي} [y]}

Ce phonème correspond à la lettre arabe \RL{ي}. Il correspond au son \textbf{/y/} en français, comme dans les mots \textbf{yaourt} ou \textbf{youpi}.


\subsection{Consonne ayant évolué depuis l'arabe}

Il existe également \textbf{une} consonne qui a évolué depuis l'arabe depuis la  consonne \textbf{\RL{ق} [q]}.

Le phonème \textbf{[g]} apparaît dans plusieurs mots d'origine arabe, comme \textbf{digla (datte)} par exemple. Sa prononciation est similaire au son \textbf{/g/} qu'on peut retrouver en français dans les mots \textbf{garage} ou \textbf{gueule}. Dans plusieurs transcriptions, on pourra retrouver l'écriture \textbf{\RL{ڨ}}, mais elle ne fait pas l'unanimité puisqu'elle n'est pas originaire.

On retrouvera surtout cette consonne dans des dialectes tunisiens qui remplacent le son \textbf{[q]} par le son \textbf{[g]} (on parle de dialectes hilaliens). Dans la version "standardisée" du tunisien, cette consonne est utilisée pour prononcer les mots importés, même si la proportion de mots non importés l'utilisant est non négligeable. On retrouvera donc des mots assez modernes qui changent de sens en fonction de l'emploi de \textbf{[q]} ou \textbf{[g]}, la substitution de l'un par l'autre n'est donc pas nécessairement anodine. Par exemple, on pourra retrouver les mots \textbf{/qammer/ (parier)} et \textbf{/gammer/ (viser)}.

\subsection{Consonnes d'origine étrangère}

A l'ensemble des consonnes qui vous ont été présentées plus haut s'ajoutent deux autres consonnes, provenant toutes les deux de langues étrangères, probablement du \textbf{français} et de \textbf{l'italien}.

Ces deux consonnes servent uniquement à prononcer des mots importés : 
\begin{itemize}
    \item Le son \textbf{[p]}, comme dans les mots \textbf{port} ou \textbf{papa}.
    \item Le son \textbf{[v]}, comme dans les mots \textbf{valise} ou \textbf{voiture}.
\end{itemize}

On notera que deux écritures "arabisantes" existent, mais ne sont pas systématiquement reconnues comme orthographes officielles. Ainsi, pour \textbf{[p]}, on pourra voir \RL{پ}, alors que pour \textbf{[v]}, on pourra trouver \RL{ڥ}.

Il est intéressant de noter que ces deux consonnes peuvent être remplacées par le son le plus proche existant nativement en arabe, généralement lorsqu'il y a une difficulté de prononciation, ou par habitude. Ainsi, le \textbf{[p]} pourra se transformer en \textbf{[b]}, alors que \textbf{[v]} pourra se transformer en \textbf{[f]}.

\subsection{Système vocalique}\label{Système vocalique}
Le tunisien étant dérivé de l'arabe, le système vocalique, en tout cas dans la façon qu'on les tunisiens de se l'imaginer, tourne autour de \textbf{trois} voyelles uniquement. Passons les d'abord en revue :
\begin{itemize}
    \item La \textbf{fatha}, correspondant au son \textbf{[a-e]};
    \item La \textbf{dhamma}, correspondant au son \textbf{[u]};
    \item La \textbf{kasra}, correspondant au son \textbf{[i]}.
\end{itemize}

Cependant, les années passant, la réalisation de certaines voyelles a évolué. Les linguistes modernes ont du mal à s'accorder sur le nombre de voyelles que distingue le tunisien. C'est en réalité un exercice assez difficile dans la mesure où les voyelles sont réalisées très différemment en fonction de la région du locuteur, à l'instar du français\footnote{Le mot \textit{rose} ne se prononce pas pareil à Paris qu'à Toulouse.} et de l'anglais\footnote{Le RP English distingue beaucoup plus de diphtongues que le General American par exemple.}. 

Je vous propose ci-dessous d'en parcourir la quasi-totalité afin de vous rendre compte de l'étendue de l'inventaire phonétique tunisien. Certaines ne présentent que des différences mineures entre elles, auquel cas il n'est pas nécessaire de se forcer à les prononcer de manière exacte (un tunisophone ne fera lui-même que peu la distinction). Gardez à l'esprit que \textbf{grammaticalement}, le tunisien ne fait bien la différence qu'entre trois voyelles.
\begin{center}
\begin{tabular}{||c | c | c | c||} 
 \hline
 \textbf{\makecell{Transcription\\phonétique}} & \textbf{Transcriptions} & \textbf{Équivalent FR/EN} & \textbf{Tunisien}\\ [2.5ex] 
 \hline\hline
 [a]  & a & \underline{a}ller / g\underline{u}t & \RL{قَرْنْ}\\ 
 \hline
 [\ae]\texttildelow[\textepsilon]  & è & \underline{é}couter / b\underline{e}d & \RL{عْلاَشْ}\\
 \hline
 [\textsc{i}]  & é & m\underline{é}chant / b\underline{i}t & \RL{مَاتْ}\\  
 \hline
 [ \textschwa]  & e & kill\underline{e}r & \RL{ظَاهِرْ}\\ 
 \hline
 [i]  & i & r\underline{i}vière / m\underline{ee}t & \RL{فِيسَعْ}\\ 
 \hline
 [\textopeno]\texttildelow[\textupsilon]  & o & s\underline{o}rtir / c\underline{o}re & \RL{مُخْ}\\ 
 \hline
 [u]  & u & m\underline{ou}ton / d\underline{oo}m & \RL{مَاهُوشْ}\\ 
 \hline
\end{tabular}
\end{center}

Dans le tableau ci-dessous, vous aurez pu remarquer que certains sons sont relativement proches. Je me suis également permis de regrouper des sons qui, même si tous réalisés par des tunisophones, ne sont pas conscientisés comme étant des voyelles différentes. 

Historiquement, les voyelles ont évolué comme suit :
\begin{itemize}
    \item La \textbf{fatha} a évolué pour donner les voyelles [a], [\ae]\texttildelow[\textepsilon] et [\textsc{i}] ;
    \item La \textbf{dhamma} a évolué pour donner les voyelles [\textopeno]\texttildelow[\textupsilon] et [u] ;
    \item La \textbf{kasra} a évolué pour donner les voyelles [ \textschwa] et [i].
\end{itemize}

En pratique, l'utilisation de certaines voyelles dépend de l'environnement consonantique, et plus particulièrement de la consonne qui précède la voyelle d'intérêt. Ainsi, 

\begin{itemize}
    \item Les voyelles dérivées d'une \textbf{fatha} se prononcent [a] quand elles se situent juste avant ou juste après les phonèmes suivants : \RL{ز} [z], \RL{ر} [r], \RL{ق} [q], \RL{خ} [\textchi], \RL{غ} [\textinvscr], \RL{ه} [\texthth], \RL{ع} [\textrevglotstop], \RL{ح} [h], \RL{ض} [\dh \super\textrevglotstop], \RL{ط} [t\super\textrevglotstop] et \RL{ص} [s\super\textrevglotstop] ;
    \item Les voyelles dérivées d'une \textbf{fatha} se prononcent [\ae]\texttildelow[\textepsilon] autour du  phonème \RL{ل} [l];
    \item Dans les autres cas, ces voyelles là se prononcent [\textsc{i}].
\end{itemize}

\textbf{\textsc{Note} :} On pourra parler en quelque sorte d'une harmonie vocalique.

Il faut également noter que le tunisien fait la distinction entre les voyelles courtes et les voyelles longues, comme en arabe standard. Voyelles courtes et voyelles longues, en fonction de leur environnement consonantique, ont une sémantique différente. Nous aborderons dans la suite du cours ces différences-là.

Il faudra également parler des \textbf{voyelles nasales}. La Tunisie et les tunisiens sont restés en contact assez longtemps avec la langue française pour que plusieurs mots passent d'une langue à l'autre, le sens nous intéressant en l'occurrence étant du français vers le tunisien. Ces mots importés ont réussi à imposer avec eux l'import de trois voyelles nasales : 

\begin{center}
\begin{tabular}{||c c c||} 
 \hline
 Transcription en français & Alphabet phonétique & Exemple en français\\ [2.5ex] 
 \hline\hline
 \textbf{an}  & \~\textscripta & Pl\underline{an} \\
 \textbf{on}  & \~\textopeno & C\underline{om}pas \\
 \textbf{in}  & \~\textepsilon & \underline{In}ternet\\ 
 \hline
\end{tabular}
\end{center}

Vous avez dû remarquer qu'on prononce une quatrième voyelle nasale en français, couramment notée \textbf{/en/} [\~\oe] comme dans \textbf{r\underline{en}dez-vous}. Cette voyelle nasale est remplacée dans toutes les occurrences des mots importés par le \textbf{/an/}, en tout cas chez la majorité des tunisiens. Cependant, il vous est tout à fait envisageable de la prononcer comme bon vous semble, \textbf{/an/} ou \textbf{/en/}.

Il faut également savoir que la plupart de ces voyelles nasales ne sont pas complètement nasalisées, c'est-à-dire que vous pourrez souvent entendre sur la fin de la voyelle quelque chose qui ressemble à un \textbf{/n/}\footnote{C'est le même phénomène qu'on peut retrouver dans l'accent du sud-ouest de la France, où on assiste à la dénasalisation des voyelles nasales}. 

\section{Transcription utilisée dans ce cours}

Maintenant que nous avons vu l'ensemble des sons qui sont réalisables en  tunisien, je vous propose d'établir ensemble une transcription, c'est-à-dire le système de substitution des sons par des lettres (autrement dit, le système d'écriture).

Je nous fixe quelques règles pour cette transcription, en espérant que nous puissions les respecter le plus possible :
\begin{itemize}
    \item Chaque son devra être représenté par un seul et unique symbole ou combinaison de symboles ;
    \item On doit limiter le nombre de son produits par une combinaison de plusieurs symboles (comme en français où \textbf{/o/} et \textbf{/i/} s'associent pour faire le son \textbf{[wa]}) ;
    \item On s'autorise à regrouper des sons qui seront représentés par le même symboles, du moment que le contexte phonétique permette de déduire le son que représente le symbole sans ambiguïté.
\end{itemize}

\subsection{Transcription des consonnes et des voyelles}
Après quelques itérations, j'en arrive au système suivant, qui a ses défauts et ses avantages, comme tout système. 

\begin{center}
\begin{tabular}{||c c c||} 
 \hline
 \textbf{\makecell{Transcription\\langue d'origine}} & \textbf{\makecell{Transcription\\phonétique}} & \textbf{\makecell{Transcription\\retenue}}\\ [2.5ex] 
 \hline\hline
 \RL{ا} / \RL{ء} & ['] & ' / $\emptyset$ \\ 
 \hline
 \RL{ب} & [b] & b \\ 
 \hline
 \RL{ت} & [t] & t \\ 
 \hline
 \RL{ث} & [\texttheta] & \th \\ 
 \hline
 \RL{ج} & [\textyogh] & j \\ 
 \hline
 \RL{ح} & [h] & \textcrh \\ 
 \hline
 \RL{خ} & [\textchi] & x \\ 
 \hline
 \RL{د} & [d] & d \\ 
 \hline
 \RL{ذ} & [\dh] & \dh \\ 
 \hline
 \RL{ر} & [r] & r \\ 
 \hline
 \RL{ز} & [z] & z \\ 
 \hline
 \RL{س} & [s] & s \\ 
 \hline
 \RL{ش} & [\textesh] & \v{s} \\ 
 \hline
 \RL{ص} & [s\super \textrevglotstop] & \c{s} \\ 
 \hline
 \RL{ض}/\RL{ظ} & [\dh \super \textrevglotstop] & \c{\dh} \\ 
 \hline
 \RL{ط} & [t\super \textrevglotstop] & \c{t} \\ 
 \hline
 \RL{ع} & [\textrevglotstop] & \c{a} \\ 
 \hline
 \RL{غ} & [\textinvscr] & \v{r} \\ 
 \hline
 \RL{ف} & [f] & f \\ 
 \hline
 \RL{ق} & [q] & q \\ 
 \hline
 \RL{ك} & [k] & k \\ 
 \hline
 \RL{ل} & [l] & l \\ 
 \hline
 \RL{م} & [m] & m \\ 
 \hline
 \RL{ن} & [n] & n \\ 
 \hline
 \RL{ه} & [\texthth] & h \\ 
 \hline
\end{tabular}
\end{center}


\begin{center}
 \begin{tabular}{||c c c||} 
 \hline
 \textbf{\makecell{Transcription\\langue d'origine}} & \textbf{\makecell{Transcription\\phonétique}} & \textbf{\makecell{Transcription\\retenue}}\\ [2.5ex] 
 \hline\hline
 \RL{و} & [w] & w \\ 
 \hline
 \RL{ي} & [y] & y \\ [2.5ex] 
 \hline
 \RL{ڨ} & [g] & g \\  
 \hline
 \RL{پ} & [p] & p \\  
 \hline
 \RL{ڥ} & [v] & v \\ 
 \hline
 $\emptyset$ & [a]  & a \\ 
 \hline
 $\emptyset$ & [\ae]\texttildelow[\textepsilon]  & è \\
 \hline
 $\emptyset$ & [\textsc{i}]  & é \\  
 \hline
 $\emptyset$ &[ \textschwa]  & e \\ 
 \hline
 $\emptyset$ &[i]  & i \\ 
 \hline
 $\emptyset$ &[\textopeno]\texttildelow[\textupsilon]  & o \\ 
 \hline
 $\emptyset$ &[u]  & u \\
 \hline
 \textbf{an} &[\~\textscripta]  & \v{a} \\
 \hline
 \textbf{on} &[\~\textopeno]  & \v{o} \\
 \hline
 \textbf{in} &[\~\textepsilon]  & \v{i} \\ [2.5ex] 
 \hline
\end{tabular}
\end{center}

Quelques commentaires sur cette transcription.

\begin{itemize}
    \item J'ai décidé de laisser la possibilité d'omettre le symbole pour le \textbf{coup de glotte [']}. Ce choix est motivé par le fait que les tunisophones ont de plus en plus tendance à l'omettre, que ce soit au début ou à la fin des mots. Je laisse donc la possibilité de le rajouter au milieu des mots, en lui affectant un symbole \footnote{En pratique, je vais essayer de le noter tant que faire se peut au cours de ce cours, notamment en début de mot, pour des raisons \textbf{grammaticale} et \textbf{étymologique}.};
    \item Tous les symboles des \textbf{consonnes emphatiques} portent une \textbf{cédille}. J'ai choisi ce système afin d'aider à la prononciation, et aider si besoin pour l'harmonie consonantique;
    \item La consonne [\textrevglotstop] se note aussi avec une \textbf{cédille}, en se servant d'un \textbf{/a/} comme support. J'ai préféré cette notation pour ne pas induire en erreur en proposant une lettre support qui était sans rapport. L'alphabet maltais a par exemple fait le choix de se servir d'un /g/ comme support;
    \item J'ai emprunté deux lettres à l'alphabet du moyen anglais : \textsc{thorn} "\textbf{\th}" et \textsc{eth} "\textbf{\dh}". Elles servaient à l'époque à marquer les mêmes sons, mais les symboles ont disparu avec l'importation de l'imprimerie depuis la France;
    \item J'ai marqué d'un diacritique les symboles pour \textbf{[\textesh]} et \textbf{[\textinvscr]}. On pourrait envisager deux symboles pour marquer ces sons-là, comme par exemple /sh/ et /gh/, mais il me semblait que ça porterait à confusion avec le son \textbf{[h]};
    \item J'ai mis la même diacritique pour les \textbf{voyelles nasales} : ce qui motive ce choix est que cette diacritique sert à noter une prononciation différente mais proche de la lettre support;
    \item Pour les autres \textbf{voyelles}, j'ai choisi un système qui soit intuitif pour les locuteurs francophones : la transcription correspond à peu de choses près à la prononciation qu'on aurait en français. La seule exception que je me suis autorisée est pour le son \textbf{[u]}, qui n'est marqué que d'un symbole plutôt que de deux en français (/ou/).
\end{itemize}

\subsection{Voyelles longues et consonnes géminées}
En plus de la retranscription des sons, il faut parler du cas des voyelles longues et des consonnes géminées (les consonnes doublées).

Le tunisien, comme l'arabe, fait une distinction sémantique entre :
\begin{itemize}
    \item \textbf{Voyelles courtes et longues} : La longueur d'une voyelle change le sens d'un mot, par exemple sa fonction grammaticale comme dans \textbf{[mut] (meurs, verbe à l'impératif)} et \textbf{[mu:t] (la mort)}.
    \item \textbf{Consonnes simples et consonnes géminées (doublées)} : Le doublage des consonnes en tunisien change également le sens d'un mot, par exemple \textbf{[ba\textrevglotstop \textschwa d] (après)} et \textbf{[ba\textrevglotstop\textrevglotstop \textschwa d] (éloigne, verbe à l'impératif)}.
\end{itemize}

Pour continuer de faire cette distinction à l'écrit, je propose dans la suite de \textbf{doubler} les symboles qui représentent les voyelles ou les consonnes longues. Ainsi, en reprenant les exemples précédents : 

\begin{itemize}
    \item \textbf{[mut]} $\rightarrow$ \textbf{mut}
    \item \textbf{[mu:t]} $\rightarrow$ \textbf{muut}
    \item \textbf{[ba\textrevglotstop \textschwa d]} $\rightarrow$ \textbf{ba\c{a}ed}
    \item \textbf{[ba\textrevglotstop\textrevglotstop \textschwa d]} $\rightarrow$ \textbf{ba\c{a}\c{a}ed}
\end{itemize}

\section{Et maintenant ? }
Avec tout ceci en poche, je crois que nous avons l'ensemble des outils nécessaires pour commencer à apprendre le tunisien dans de bonnes conditions. C'est parti !
