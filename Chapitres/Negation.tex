\chapter{La négation}\label{Négation}
\chapterletter{D}ans ce chapitre, nous allons voir comment former la négation dans une phrase. Notamment, vous allez voir la façon qu'a trouvé le tunisien afin  d'exprimer la négation dans une phrase nominale, qui n'a pourtant pas de verbe ! 

\section{Un peu d'histoire}
En \textbf{arabe}, il existe de très nombreuses formes qui permettent d'exprimer la négation. Pire que ça, en grammaire arabe, il existe de catégories de conjugaison au présent qui ne s'utilisent \textit{que} pour exprimer négation\footnote{Ce sont \RL{almDArP al} et \RL{almDArP almnSwb}.}. Il faut compter pas moins de \textit{sept} formes\footnote{Et encore, ce ne sont que les plus utilisées, il doit exister des formes plus rares qui me sont inconnues.}, car chaque forme a son utilisation particulière :

\begin{itemize}
    \item Pour la phrase \textbf{verbale}, pour exprimer la négation dans le \textbf{passé}, on compte \textbf{deux} formes, l'une se formant à partir de \textbf{mèè (\RL{mA})+ verbe au passé (\RL{almADy})}, et l'autre à partir de \textbf{lam (\RL{lm}) + verbe au présent (\RL{almDArP almGzwm})} ;
    \item Pour la phrase \textbf{verbale}, pour exprimer la négation dans le \textbf{présent}, on utilise la forme \textbf{lèè (\RL{lA}) + verbe au présent (\RL{almDArP almrfwP})} ; 
    \item Pour la phrase \textbf{verbale}, pour exprimer la négation dans le \textbf{futur}, on utilise la forme \textbf{lan (\RL{ln}) + verbe au présent (\RL{almDArP almnSwb})} ;
    \item Pour la phrase \textbf{nominale}, on peut compte \textbf{trois} formes, la première se formant à partir de l'ajout du verbe \textbf{laysa\footnote{C'est un verbe incomplet (cf. chapitre \ref{VerbesDefectueux}) qui se conjugue.} (\RL{lys}) + phrase nominale\footnote{Il y a alors un changement de voyelle sur l'attribut de la phrase nominale, qui correspond alors à une sorte de cas grammatical.}}, la deuxième en ajoutant le mot \textbf{\v{r}ayr (\RL{Ryr})} avant l'attribut de la phrase nominale, et la dernière en exprimant la négation comme s'il s'agissait d'une phrase verbale avec le verbe \textbf{kèèna\footnote{C'est un verbe concave (cf. chapitre \ref{VerbesDefectueux}) qui se conjugue.} (\RL{kAn}) + phrase nominale\footnote{La phrase nominale subit alors les mêmes modifications qu'avec l'emploi de \textbf{laysa}.}}.
\end{itemize}

Et toutes ses formes concenrnent uniquement la négation simple, c'est-à-dire l'équivalent en français de la construction \textit{ne ... pas} ! On retrouvera alors également d'autres constructions en arabe qui correspondent par exemple à \textit{ne ... jamais}, \textit{ni ... ni ...}, etc.

Comme vous avez pu déjà le remarqué dans un précédent chapitre (cf. chapitre \ref{ConjSS}), les temps de l'\textbf{arabe} spécifiques à la négation ont disparu du \textbf{tunisien}\footnote{La disparition s'est sans doute faite en parallèle de la perte en tunisien de la pronociation des voyelles finales (\textbf{/taxruju/}, \textbf{/taxruj/} et \textbf{/taxruja/} étant sans doute toutes les trois devenues \textbf{/toxroj/}), mais cela reste de la pure spéculation de ma part.}. 

Avec cette simplification, on constatera en \textbf{tunisien} une très grande simplification de l'expression de la négation, puisqu'une de toutes ces six formes a pris progressivement le dessus sur toutes les autres.

Vous verrez donc dans la suite du chapitre que \textbf{la négation en tunisien} se construit quasi-exclusivement autour de la particule \textbf{/mèè/ (\RL{mA})}, dont  l'utilisation dans tous les contextes a nécessité une légère adaptation de la syntaxe.

\section{La négation dans la phrase verbale au passé ou au présent}
Commençons tout d'abord par l'expression de la négation dans la phrase \textbf{verbale} au \textbf{passé} ou au \textbf{présent}, qui est sans doute la forme la plus simple d'entre toutes. 

Au passé et au présent, le passage à la phrase négative se fait assez similairement à la construction \textit{ne ... pas} en français. 

\begin{center}
    \textbf{\Large mèè + verbe + (e)\v{s}}
\end{center}

où \textbf{/mèè/} représente la particule pour exprimer la négation dans le passé en \textbf{arabe} (\RL{mA}), et où la particule \textbf{/\v{s}/} se comporte comme le \textit{pas} français.

Comme nous l'avons déjà soulevé dans d'autres chapitres, le \textbf{tunisien} a horreur des clusters de consonnes et de l'enchaînement de deux voyelles, c'est pourquoi la particule finale de la négation a deux formes possibles : 
\begin{itemize}
    \item Lorsque le verbe se termine par une \textbf{voyelle}, la particule prend la forme \textbf{\v{s}} ; 
    \item Lors le verbe se termine par une \textbf{consonne}, la particule prend la forme \textbf{e\v{s}}.
\end{itemize}

Mettons cela tout de suite en pratique à l'aide d'exemples :

\begin{center}
    \begin{tabular}{|| c | c | c ||}
        \hline
        \textbf{Affirmatif} & \textbf{Négatif} & \textbf{Tradution}\\ \hline\hline
        Xrajt mel kujiina. & \textbf{Mèè} xrajt\textbf{e\v{s}} mel kujiina. & \textit{\makecell{Je ne suis pas sorti de \\la cuisine.}} \\
        \hline
        Kliituu lef\c{t}uur. & \textbf{Mèè} kliituu\textbf{\v{s}} lef\c{t}uur. & \textit{\makecell{Vous n'avez pas mangé\\le déjeuner.}} \\
        \hline
        \makecell{\c{S}abri yen\c{a}at \\fes-sééyaq.} & \makecell{\c{S}abri \textbf{mèè} yen\c{a}at\textbf{e\v{s}} \\fes-sééyaq.} & \textit{\makecell{Sabri ne guide pas le \\conducteur}} \\
        \hline
    \end{tabular}
\end{center}

Je vais tout de même ajouter une légère subtilité\footnote{Sans laquelle vous risquez d'avoir un accent particulier.} : lorsque le verbe se termine par un \textbf{/a/}, comme par exemple lorsqu'on le conjugue au passé à la première personne du pluriel (\textbf{'a\textcrh na}), le \textbf{/a/} du verbe \textbf{se transforme en /è/}.

\begin{center}
    \begin{tabular}{|| c | c | c ||}
        \hline
        \textbf{Affirmatif} & \textbf{Négatif} & \textbf{Tradution}\\ \hline\hline
        Fhemnaa el-wa\c{\dh}\c{a}iyya. & \makecell{\textbf{Mèè} fhemn\textbf{èè\v{s}} \\el-wa\c{\dh}\c{a}iyya.} & \textit{\makecell{Nous n'avons pas\\ compris la situation.}} \\
        \hline
        Xsarnaa fel-lo\c{a}ba. & \makecell{\textbf{Mèè} xsarn\textbf{èè\v{s}} \\fel-lo\c{a}ba.} & \textit{\makecell{Nous n'avons pas perdu\\ au jeu.}} \\
        \hline
    \end{tabular}
\end{center}

\section{La négation dans la phrase nominale et au futur}
Les formes négatives dans la phrase \textbf{nominale} ou dans la phrase \textbf{verbale au futur} sont très semblables car elles se construisent autour d'une même forme contractée que nous allons voir tout de suite.

\subsection{Contraction de mèè + pronom + \v{s}}
La contraction de la particule \textbf{mèè}\footnote{Il s'agit bien de la même que celle de la phrase verbale.} avec le \textbf{pronom personnel} permet  d'identifier à qui se rapporte la négation. Cette forme étant souvent employée, elle est nécessairement soumise à quelques irrégularités, et vous vous doutez bien qu'il ne s'agit pas de juxtaposer les deux parties. 

Voici chacune des formes de la contraction : 

\begin{center}
    \begin{tabular}{|| c | c | c ||}
        \hline
        \textbf{Pronom} & \textbf{Contraction} & \textbf{\makecell{Forme \\alternative}}\\ \hline \hline
        \jegras & mènii\v{s} & - \\ \hline 
        \tugras & mèke\v{s} & - \\ \hline 
        \ilgras & mèhuwwèè\v{s} & mèhuu\v{s}\\ \hline 
        \ellegras & mèhiyyèè\v{s} & mèhèè\v{s}\\ \hline 
        \nousgras & ma\textcrh nèè\v{s} & mènèè\v{s}\\ \hline 
        \vousgras & mèkom\v{s} & - \\ \hline 
        \ilsgras & mèhumèè\v{s} & mèhom\v{s}\\ \hline 
    \end{tabular}
\end{center}

Comme vous pouvez le voir, certaines contractions possèdent une forme longue et une forme courte, qui sont totalement interchangeables.

J'attire également votre attention sur les \textbf{deuxièmes personnes} qui prennent un \textbf{/k/}, ce qui fait d'elles le \textit{vilain petit canard} de cette forme contractée\footnote{Si vous avez l'\oe il vous aurez remarquer que cette contraction ressemblement grandement à la forme du possessif (cf. chapitre \ref{Possessif}).}.

Cette contraction connue, on peut alors directement l'employer pour former la négation dans la phrase nominale ou au futur.

\subsection{Négation dans la phrase nominale}
La syntaxe de la phrase nominale négative est la suivante : 

\begin{center}
    \textbf{\Large sujet + forme contractée + attribut}
\end{center}

Il convient alors de choisir la forme contractée qui correspond au pronom du sujet de la phrase.

Voici quelques exemples : 

\begin{center}
    \begin{tabular}{|| c | c | c ||}
        \hline
        \textbf{Affirmatif} & \textbf{Négatif} & \textbf{Tradution} \\ \hline \hline
        
        \makecell{L-uulayyed q\c{s}iir.} & \makecell{L-uulayyed \textbf{mèhuu\v{s}}\\ q\c{s}iir.} & \textit{\makecell{Le garçon n'est \\pas petit.}}\\ \hline
        
        \makecell{Héé\dh i mu\v{s}awwra.} & \makecell{Héé\dh i \textbf{mèhiyyèè\v{s}}\\ mu\v{s}awwra.} & \textit{\makecell{Ce n'est pas \\un appareil photo.}}\\ \hline

        \makecell{'Entuuma \textcrh aa\c{\dh}riin.} & \makecell{'Entuuma \textbf{mèkom\v{s}}\\\textcrh aa\c{\dh}riin.} & \textit{\makecell{Vous n'êtes \\ pas prêts.}}\\ \hline

        \makecell{\nous mé\v{s}iin \\ler-restor\v{a}.} & \makecell{\nous \textbf{ma\textcrh nèè\v{s}}\\mé\v{s}iin  ler-restor\v{a}.} & \textit{\makecell{Nous n'allons pas \\aller au restaurant.}}\\ \hline
        
    \end{tabular}
\end{center}

La forme contractée \underline{seule} peut se suffire à elle-même puisqu'elle contient déjà l'information sur le sujet. Ainsi, il est possible d'omettre le sujet.

\begin{center}
    \begin{tabular}{|| c | c | c ||}
        \hline
        \textbf{Sans omission} & \textbf{Avec omission} & \textbf{Traduction} \\ \hline \hline
        
        \je mènii\vs{} xaarej. & Ménii\vs{} xaarej. & \textit{\makecell{Je ne vais \\ pas sortir.}} \\\hline

        \ils mèhom\vs{} twènsa. & Mèhom\vs{} twènsa. & \textit{\makecell{Ils ne sont \\pas tunisiens.}} \\\hline

        \makecell{Héé\dh a mèhuwwè\vs{}\\ \ca aadi.} & Mèhuwwè\vs{} \ca aadi. & \textit{Ce n'est pas normal.}\\ \hline
        
    \end{tabular}
\end{center}

\subsection{Négation au futur}
La négation au futur s'exprime uniquement dans la phrase verbale, et se construit de la façon suivante :

\begin{center}
    \textbf{\Large sujet + forme contractée + bèè\vs{} + verbe au présent}
\end{center}

Dans cette construction-là, \textbf{bèè\vs{}} est la particule relative au \textbf{futur}. D'autres marques qui permettent d'exprimer le futur usuellement ne sont pas permises, comme \textbf{tawwa} par exemple\footnote{Pour la construction du futur, référez-vous au chapitre \ref{Futur}.}.

La syntaxe est très similaire à celle de la phrase \textbf{nominale} : 

\begin{itemize}
    \item D'une part, il faut \textbf{accorder} la forme contractée avec le sujet ; 
    \item D'autre part, la forme contractée peut se suffir à elle-même, et le sujet peut être omis.
\end{itemize}

De la même façon que pour la phrase nominale, il faut \textbf{accorder} la forme contractée avec le sujet. 

Voici quelques exemples : 

\begin{center}
    \begin{tabular}{|| c | c | c ||}
        \hline
        \textbf{Affirmatif} & \textbf{Négatif} & \textbf{Tradution} \\ \hline \hline
        
        \makecell{Bèè\vs{} nem\vs iw \\lel-uutil.} & \makecell{\textbf{Ma\hb nèè\vs{}} bèè\vs{} \\ nem\vs iw lel-uutil.} & \textit{\makecell{Nous n'allons pas \\aller à l'hôtel.}}\\ \hline

        \makecell{Es-secrétiirra bèè\vs \\ teb\ca a\th{} el-kontraatu.} & \makecell{Es-secrétiirra \textbf{mahiyyèè\vs{}} \\bèè\vs{} teb\ca a\th{} el-kontraatu.} & \textit{\makecell{La secrétaire n'enverra \\pas le contrat.}}\\ \hline

        \makecell{Tawwa no\vs reb \\qahutii.} & \makecell{\textbf{Mènii\vs{}} bèè\vs{} no\vs reb \\ qahutii.} & \textit{\makecell{Je ne vais pas boire\\ mon café.}}\\ \hline
        
    \end{tabular}
\end{center}

\section*{Dialogue}
\section*{Vocabulaire}