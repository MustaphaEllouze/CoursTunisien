\chapter{La négation}\label{Négation}
\chapterletter{D}ans ce chapitre, nous allons voir comment former la négation dans une phrase. Notamment, vous allez voir la façon qu'a trouvé le tunisien afin \linebreak d'exprimer la négation dans une phrase nominale, qui n'a pourtant pas de verbe ! 

\section{Un peu d'histoire}
En \textbf{arabe}, il existe de très nombreuses formes qui permettent d'exprimer la négation. Pire que ça, en grammaire arabe, il existe de catégories de conjugaison au présent qui ne s'utilisent \textit{que} pour exprimer négation\footnote{Ce sont \RL{almDArP al} et \RL{almDArP almnSwb}.}. Il faut compter pas moins de \textit{six} formes\footnote{Et encore, ce ne sont que les plus utilisées, il doit exister des formes plus rares qui me sont inconnues.}, car chaque forme a son utilisation particulière :

\begin{itemize}
    \item Pour la phrase \textbf{verbale}, pour exprimer la négation dans le \textbf{passé}, on compte \textbf{deux} formes, l'une se formant à partir de \textbf{mèè (\RL{mA})+ verbe au passé (\RL{almADy})}, et l'autre à partir de \textbf{lam (\RL{lm}) + verbe au présent (\RL{almDArP al})} ;
    \item Pour la phrase \textbf{verbale}, pour exprimer la négation dans le \textbf{présent}, on utilise la forme \textbf{lèè (\RL{lA}) + verbe au présent (\RL{almDArP almrfwP})} ; 
    \item Pour la phrase \textbf{verbale}, pour exprimer la négation dans le \textbf{futur}, on utilise la forme \textbf{lan (\RL{ln}) + verbe au présent (\RL{almDArP almnSwb})} ;
    \item Pour la phrase \textbf{nominale}, on peut compte \textbf{deux} formes, l'unse se formant à partir de \textbf{laysa\footnote{C'est un verbe incomplet (cf. chapitre \ref{VerbesDefectueux}) qui se conjugue.} (\RL{lys}) + phrase nominale\footnote{Il y a alors un changement de voyelle sur l'attribut de la phrase nominale, qui correspond alors à une sorte de cas grammatical.}}, l'autre en ajoutant le mot \textbf{\v{r}ayr (\RL{Ryr})} avant l'attribut de la phrase nominale.
\end{itemize}

Et toutes ses formes concenrnent uniquement la négation simple, c'est-à-dire l'équivalent en français de la construction \textit{ne ... pas} ! On retrouvera alors également d'autres constructions en arabe qui correspondent par exemple à \textit{ne ... jamais}, \textit{ni ... ni ...}, etc.

Comme vous avez pu déjà le remarqué dans un précédent chapitre (cf. chapitre \ref{ConjSS}), les temps de l'\textbf{arabe} spécifiques à la négation ont disparu du \textbf{tunisien}\footnote{La disparition s'est sans doute faite en parallèle de la perte en tunisien de la pronociation des voyelles finales (\textbf{/taxruju/}, \textbf{/taxruj/} et \textbf{/taxruja/} étant sans doute toutes les trois devenues \textbf{/toxroj/}), mais cela reste de la pure spéculation de ma part.}. 

Avec cette simplification, on constatera en \textbf{tunisien} une très grande simplification de l'expression de la négation, puisqu'une de toutes ces six formes a pris progressivement le dessus sur toutes les autres.

Vous verrez donc dans la suite du chapitre que \textbf{la négation en tunisien} se construit quasi-exclusivement autour de la particule \textbf{/mèè/ (\RL{mA})}, dont \linebreak l'utilisation dans tous les contextes a nécessité une légère adaptation de la syntaxe.

\section{La négation dans la phrase verbale}
Commençons tout d'abord par l'expression de la négation dans la phrase \textbf{verbale}, qui est sans doute la forme la plus simple d'entre toutes. 

\subsection{La négation au passé et au présent}
Au passé et au présent, le passage à la phrase négative se fait assez similairement à la construction \textit{ne ... pas} en français. 

\begin{center}
    \textbf{\Large mèè + verbe + (e)\v{s}}
\end{center}

où \textbf{/mèè/} représente la particule pour exprimer la négation dans le passé en \textbf{arabe} (\RL{mA}), et où la particule \textbf{/\v{s}/} se comporte comme le \textit{pas} français.

Comme nous l'avons déjà soulevé dans d'autres chapitres, le \textbf{tunisien} a horreur des clusters de consonnes et de l'enchaînement de deux voyelles, c'est pourquoi la particule finale de la négation a deux formes possibles : 
\begin{itemize}
    \item Lorsque le verbe se termine par une \textbf{voyelle}, la particule prend la forme \textbf{\v{s}} ; 
    \item Lors le verbe se termine par une \textbf{consonne}, la particule prend la forme \textbf{e\v{s}}.
\end{itemize}

Mettons cela tout de suite en pratique à l'aide d'exemples :

\begin{center}
    \begin{tabular}{|| c | c | c ||}
        \hline
        \textbf{Affirmatif} & \textbf{Négatif} & \textbf{Tradution}\\ \hline\hline
        Xrajt mel kujiina. & \textbf{Mèè} xrajt\textbf{e\v{s}} mel kujiina. & \textit{\makecell{Je ne suis pas sorti de \\la cuisine.}} \\
        \hline
        Kliituu lef\c{t}uur. & \textbf{Mèè} kliituu\textbf{\v{s}} lef\c{t}uur. & \textit{\makecell{Vous n'avez pas mangé\\le déjeuner.}} \\
        \hline
        \makecell{\c{S}abri yen\c{a}at \\fes-sééyaq.} & \makecell{\c{S}abri \textbf{mèè} yen\c{a}at\textbf{e\v{s}} \\fes-sééyaq.} & \textit{\makecell{Sabri ne guide pas le \\conducteur}} \\
        \hline
    \end{tabular}
\end{center}

Je vais tout de même ajouter une légère subtilité\footnote{Sans laquelle vous risquez d'avoir un accent particulier.} : lorsque le verbe se termine par un \textbf{/a/}, comme par exemple lorsqu'on le conjugue au passé à la première personne du pluriel (\textbf{'a\textcrh na}), le \textbf{/a/} du verbe \textbf{se transforme en /è/}.

\begin{center}
    \begin{tabular}{|| c | c | c ||}
        \hline
        \textbf{Affirmatif} & \textbf{Négatif} & \textbf{Tradution}\\ \hline\hline
        Fhemnaa el-wa\c{\dh}\c{a}iyya. & \makecell{\textbf{Mèè} fhemn\textbf{èè\v{s}} \\el-wa\c{\dh}\c{a}iyya.} & \textit{\makecell{Nous n'avons pas\\ compris la situation.}} \\
        \hline
        Xsarnaa fel-lo\c{a}ba. & \makecell{\textbf{Mèè} xsarn\textbf{èè\v{s}} \\fel-lo\c{a}ba.} & \textit{\makecell{Nous avons perdu\\ au jeu.}} \\
        \hline
        \hline
    \end{tabular}
\end{center}

\subsection{La négation au futur}
\section{La négation dans la phrase nominale}
\section{Autres formes}

\section*{Dialogue}
\section*{Vocabulaire}