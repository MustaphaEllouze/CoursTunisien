\chapter{Les démonstratifs}
\chapterletter{P}assons maintenant quelques instants à parler des adjectifs démonstratifs. Ils vous permettront de mieux désigner ce qui vous entoure !

\section{Un peu d'histoire}
Pour être parfaitement honnête, j'ai essayé de combiner mes souvenirs de l'arabe en ce qui concerne les démonstratifs aux informations que j'ai pu grappiller en ligne. Il se trouve qu'il y a beaucoup de choses que j'ignorais et que j'ignore encore, et que les démonstratifs en arabe standard sont assez complexes à expliquer. En plus de cela, je dois vous avouer qu'il y a beaucoup de formes qui se ressemblent beaucoup entre elles, et aux différences assez subtiles.

De ce fait, j'aurais beaucoup de mal à vous expliquer exactement comment s'inspire le tunisien de l'arabe, parce que je ne saurais vous dire quelle forme en arabe a abouti à quelle autre forme en tunisien, j'en suis désolé. 

Je voudrais juste attirer votre attention sur les manières dont se déclinent les adjectifs démonstratifs en arabe : 

\begin{itemize}
    \item En \textbf{genre} : masculin (ce) et féminin (cette)
    \item En \textbf{nombre} : singulier (celui-là), duel (ces deux-là) et pluriel (ceux-là)
    \item En \textbf{proximité au locuteur} : proche (celui-ci) et loin (celui-là)
\end{itemize}

Le \textbf{tunisien} s'inspire de ce système-ci et garde les trois même inflexions\footnote{Les formes relatives au duel ont disparu.}, en réduisant cependant grandement le nombre d'adjectifs.

Le \textbf{tunisien} possède deux formes pour les adjectifs : une \textbf{forme de base} (la forme longue) et une \textbf{forme abrégée}. Comme vous l'avez déjà compris, cette distinction s'est créée par simplification progressive et perte de syllabes dans le flux du temps. 

Cependant, vous entendrez beaucoup plus souvent la forme \textbf{longue}, l'autre forme pointant le bout de son nez occasionnellement, souvent dans un parlé plus familier et plus rapide.

\section{Forme longue des démonstratifs}
Le \textbf{tunisien} comprend les inflexions suivantes des adjectifs démonstratifs : 

\begin{itemize}
    \item \textbf{Genre} : masculin / féminin 
    \item \textbf{Nombre} : singulier / pluriel
    \item \textbf{Proximité au locuteur} : proche / intermédiaire / éloigné
\end{itemize}

Cela nous donne donc \textbf{12} formes au total, la forme plurielle ne faisant pas la distinction de genre, et la forme \textbf{proche} ayant deux variantes\footnote{En réalité, il y a plusieurs formes régionales toutes valides, donc il y a sûrement moins de variations que ce que je vous ai présenté. J'ai grandi à cheval sur deux dialectes, donc je ne saurais vous dire quelle forme est plutôt employée à Tunis ou à Sfax.}.

\begin{center}
\begin{tabular}{||c | c | c | c||}
 \hline
  & \textbf{Masculin} & \textbf{Féminin} & \textbf{Pluriel}\\
 \hline\hline
 \textbf{Proche} & \makecell{Hèè\dh a \\ Hè\dh èèya} & \makecell{Hèè\dh i \\ Hè\dh iyya} & \makecell{Hèè\dh om \\ Hè\dh uuma}\\
 \hline
 \textbf{Intermédiaire} & Hèèka & Hèèki & Hèèkom \\
 \hline
 \textbf{Éloigné} & Hè\dh èèka & Hè\dh iika & Hè\dh uukom \\
 \hline
\end{tabular}    
\end{center}

Pour former le groupe nominal, il suffit de juxtaposer l'adjectif démonstratif au groupe nominal correspondant. Notez bien que \textbf{le groupe nominal doit être sous une forme définie}. Il est intéressant de noter que \textbf{l'ordre est libre}.

Voici quelques exemples : 

\begin{center}
\begin{tabular}{||c | c | c ||}
 \hline
  \textbf{Gr.Nom.-Adj.Dém.} & \textbf{Adj.Dém.-Gr.Nom.} & \textbf{Traduction}\\
 \hline\hline
  Ettriinuu hèè\dh a & Hèè\dh a ettriinuu & Ce train-ci\\
  \hline
  Ettriinuu hèèka & Hèèka ettriinuu & Ce train-là\\
  \hline
  Ettriinuu hè\dh èèka & Hè\dh èèka ettriinuu & Ce train là-bas\\
  \hline
\end{tabular}    
\end{center}

Les deux ordres sont utilisés de façon interchangeable. Il n'y a pas de réelle différence sémantique, cependant l'attention de l'interlocuteur sera généralement plus attirée vers le deuxième mot.

\section{Forme courte des démonstratifs}
Le \textbf{tunisien} possède également des formes abrégées pour les adjectifs démonstratifs présentés au paragraphe précédent.

On pourra noter les différences suivantes : 

\begin{itemize}
    \item Les distinctions en \textbf{nombre} et \textbf{genre} ne sont plus faites (ce qui ne laisse que l'inflexion sur la \textbf{proximité});
    \item On n'autorise qu'\textbf{un seul ordre} : \textbf{Adjectif-Groupe Nominal} ; 
    \item Les formes courtes sont analysées en tant que \textbf{préfixes} et non en tant que mots à part entière;
    \item Cette forme est plutôt associée à un discours plus rapide et moins formel.
\end{itemize}

On se retrouve donc avec \textbf{3} nouveaux adjectifs démonstratifs : 
\begin{itemize}
    \item \textbf{Proche} : Hèè
    \item \textbf{Intermédiaire} : Hèèk
    \item \textbf{Éloigné} : Hè\dh èèk
\end{itemize}

J'attire votre attention sur le fait que la forme courte \textbf{hèè} se termine par une \textbf{voyelle}, et que le groupe nominal est nécessairement à la forme définie. Ainsi, la première voyelle de l'article défini (\textbf{/e/}) sera assimilée au \textbf{/è/} de \textbf{hèè}\footnote{Bien sûr cela n'est vrai que si l'on utilise la forme \textbf{solaire} ou \textbf{lunaire} (cf. \ref{Forme solaire} et \ref{Forme lunaire})}.

Voici donc des exemples utilisant la forme courte des démonstratifs : 

\begin{center}
\begin{tabular}{||c | c ||}
 \hline
  \textbf{Adj.Dém.-Gr.Nom.} & \textbf{Traduction}\\
 \hline\hline
  Hèèttriinuu & Ce train-ci\\
  \hline
  Hèèkettriinuu & Ce train-là\\
  \hline
  Hè\dh èèkettriinuu & Ce train là-bas\\
  \hline
\end{tabular}    
\end{center}

\section{En tant que sujet d'une phrase}
La forme \textbf{longue} des démonstratifs peut également servir à remplacer le sujet dans certaines phrases nominales ou verbales. 

Dans ce cas-là, le démonstratif joue le même rôle que \textit{celui-ci/celui-là} en français. 

Voici quelques exemples : 

\begin{center}
\begin{tabular}{||c | c ||}
 \hline
  \textbf{Tunisien} & \textbf{Traduction}\\
 \hline\hline
  Hèè\dh a fhem & Celui-ci a compris / Il a compris\\
  \hline
  Hèèki Maryem & C'est Mariem \\
  \hline
  Hè\dh uukom muhandsiin & Ceux-là sont ingénieurs\\
  \hline
\end{tabular}    
\end{center}

\section*{Dialogue}
\section*{Vocabulaire}