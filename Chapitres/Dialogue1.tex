\h{Votre premier dialogue}
\l{À} partir de maintenant, j'estime qu'on a assez fait de grammaire pour qu'on puisse lire ensemble un dialogue, et l'analyser !

Ne vous inquiétez pas, la grammaire revient dès le prochain chapitre, même si maintenant, nous allons commencer à intégrer de plus en plus de dialogue dans le cours (ça me donnera une bonne excuse pour vous donner du vocabulaire).

\hh{Dialogue}
- \c{A}aslèèma !

- \c{A}aslèèma ! 'Esmii Mo\c{s}\c{t}faa. \v{S}nuwwa 'esmek ? 

- Net\v{s}arrfuu Mo\c{s}\c{t}faa. 'Esmii Mahdii. 

- Net\v{s}arrfuu. \v{S}nuwwa xedemtek, Mahdii ?

- Nexdem felhandsa. Ya\c{a}nii, 'éna muhandes.

- Hattééna muhandes !

- Wiin toskon ? 

- 'Éna noskon fi Bériiz. W 'entii ? 

- Noskon huuni, fi Marsiilya.

- Tnejjem twarriini elbled ? 

- Bi\c{t}\c{t}bii\c{a}a !
