\chapter{Les verbes géminés}\label{VerbesGéminés}
\chapterletter{N}ous avons parlé plus tôt des verbes sains simples (cf. chapitre \ref{ConjSS}) et de la dérivation de ces verbes (cf. chapitre \ref{DérivesVerbes}). Il existe une autre catégorie de verbes \textit{relativement} réguliers, que nous allons abordé dans ce chapitre : les \textbf{verbes géminés}.


\section{Un peu d'histoire}
Dans l'apprentissage de l'arabe classique en cours, les professeurs séparent généralement les verbes en \textbf{quatre} groupes distincts, en fonction des lettres qui se retrouvent dans la base : 

\begin{itemize}
    \item Les verbes sains \RL{أفعال صحيحة سالمة} qui sont ceux que nous avons abordés en \ref{ConjSS} ; 
    \item Les verbes dont la base comporte un \textbf{/'/} \RL{أفعال مهموزة}, dont la conjugaison est légèrement différente. 
    \item Les verbes qui comportent une consonne géminée \RL{أفعال مضعّفة}, dont la conjugaison s'adapte au fait que deux consonnes de la base soient identiques.
    \item Les verbes défectueux \RL{أفعال معتلّة}, dont la racine porte au moins une consonne défectueuse \footnote{Bien que réguliers, ce sont sans doute les verbes qui se rapprochent le plus d'une conjugaison irrégulière.}. 
\end{itemize}

Au sens strict en grammaire arabe, les verbes défectueux sont mis à part tandis que les trois autres catégories sont qualifiées de verbes sains \RL{أفعال صحيحة}\footnote{Ce que j'ai appelé verbes sains jusqu'à maintenant constitueraient en réalité un groupe de verbes \textit{super sains} si on voulait reprendre la terminologie arabe, mais ce nom me paraîssait un peu ridicule.}, puisque beaucoup plus réguliers.

En \textbf{tunisien}, la logique de séparer les verbes en plusieurs catégories porte encore beaucoup de sens. Cependant, les verbes comportant un \textbf{/'/} en \textbf{arabe} devenant quasiment tous défectueux en \textbf{tunisien}, il semble plus logique de revoir la classification, d'autant plus que les verbes \textbf{sains} sont plus nombreux que les verbes \textbf{géminés}. Une classification de ce type me paraît relativement plus appropriée, et c'est celle que j'ai décidé de retenir pour ce cours : 
\begin{itemize}
    \item Les verbes \textbf{sains}, dans le sens que je leur donne au chapitre \ref{ConjSS} ;
    \item Les verbes \textbf{géminés} (le présent chapitre) ;
    \item Les verbes \textbf{défectueux}, dans le sens que je leur donne au chapitre \ref{ConjSS}, et que j'aborde au chapitre \ref{VerbesDefectueux}.
\end{itemize}

Isoler les verbes \textbf{géminés} me semble relativement sensé : en arabe comme en tunisien, leur conjugaison est quelque peut particulière dans la mesure où on peut entendre la séparation des deux consonnes géminées dans certains cas. 

Le reste des principes que nous avons vu autre part dans ce cours s'appliquent toujours, ces verbes se dérivent identiquement (cf. chapitre \ref{DérivesVerbes}), et on peut retrouver plusieurs sous-groupes possibles dont la conjugaison diffère sensiblement\footnote{C'était déjà le cas des verbes sains, chose que nous avons vu dans la section \ref{GroupesVerbesSimples}}. 

\section{Exemples de verbes géminés}
Il y a relativement peu de choses à dire sur les verbes géminés en tunisien, outre leur forme particulière : \textbf{la consonne doublée se situe systématiquement à la fin du verbe}. 

Je vous propose de parcourir ensemble quelques exemples en tunisien.

\begin{center}
    \begin{tabular}{||c | c ||}
     \hline
     \textbf{Verbe géminé} & \textbf{Traduction}\\
     \hline\hline
      mass & \textit{toucher}\\
      \hline
      ma\c{s}\c{s} & \textit{sucer}\\
     \hline
      kabb & \textit{\makecell{renverser\\un liquide}}\\
      \hline
      \textcrh abb & \textit{aimer, vouloir}\\
      \hline
      ka\textcrh\textcrh & \textit{tousser}\\
      \hline
      lamm & \textit{ramasser}\\
     \hline
     \c{s}abb & \textit{verser}\\
    \hline
    \c{t}all & \textit{\makecell{jeter\\un \oe il}}\\
   \hline
    \end{tabular}    
\end{center}

Comme vous pouvez le constater, la structure de tous ces verbes est très similaire. Il est donc relativement aisé de les repérer.

\section{Conjugaison des verbes géminés}
En \textbf{tunisien}, on peut catégoriser les verbes gémiéns en deux sous-catégories, qui dépendent uniquement de la conjugaison du verbe au \textbf{présent}. Cette différence dans la conjugaison vient \textit{sans doute} elle même de l'arabe, mais elle ne porte en elle aucune sémantique particulière\footnote{On peut observer cette différence aussi chez les verbes sains, cf. section \ref{GroupesVerbesSimples}.}.

Parcourons donc chacun des deux groupes. 

\subsection{Sous-groupe 1 : \textit{mass}}
Le premier sous-groupe se conjugue comme le verbe \textbf{mass} (\textit{toucher}). On y retrouve également les verbes : \textbf{kabb} (\textit{renverser un liquide}), \textbf{\textcrh abb} (\textit{aimer, vouloir}) et \textbf{lamm} (\textit{ramasser}). 

On conjugue ces verbes comme ceci : 

\begin{center}
    \begin{tabular}{||c | c | c||}
     \hline
     \textbf{Pronom} & \textbf{Passé} & \textbf{Présent} \\
     \hline\hline
     'Éna & mass\textbf{iit} & \textbf{n}m\textbf{e}ss \\
     \hline
     'Enti & mass\textbf{iit} & \textbf{t}m\textbf{e}ss\\ 
     \hline
     Huwwa & mass & \textbf{y}m\textbf{e}ss\\ 
     \hline
     Hiyya & mass\textbf{et} & \textbf{t}m\textbf{e}ss\\ 
     \hline
     'A\textcrh na & mass\textbf{iinaa} &\textbf{n}m\textbf{e}ss\textbf{uu}\\ 
     \hline
     'Entuuma & mass\textbf{iituu} & \textbf{t}m\textbf{e}ss\textbf{uu}\\ 
     \hline
     Huuma & mass\textbf{uu} & \textbf{y}m\textbf{e}ss\textbf{uu}\\ 
     \hline
    \end{tabular}
\end{center}

\subsection{Sous-groupe 2 : \textit{ma\c{s}\c{s}}}
Le second sous-groupe se conjugue comme le verbe \textbf{ma\c{s}\c{s}} (\textit{sucer}). On y retrouve également les verbes : \textbf{ka\textcrh\textcrh} (\textit{tousser}), \textbf{\c{s}abb} (\textit{verser}) et \textbf{\c{t}all} (\textit{jeter un \oe il}).

On conjugue ces verbes comme ceci :

\begin{center}
    \begin{tabular}{||c | c | c||}
     \hline
     \textbf{Pronom} & \textbf{Passé} & \textbf{Présent} \\
     \hline\hline
     'Éna & ma\c{s}\c{s}\textbf{iit} & \textbf{n}m\textbf{o}\c{s}\c{s} \\
     \hline
     'Enti & ma\c{s}\c{s}\textbf{iit} & \textbf{t}m\textbf{o}\c{s}\c{s}\\ 
     \hline
     Huwwa & ma\c{s}\c{s} & \textbf{y}m\textbf{o}\c{s}\c{s}\\ 
     \hline
     Hiyya & ma\c{s}\c{s}\textbf{et} & \textbf{t}m\textbf{o}\c{s}\c{s}\\ 
     \hline
     'A\textcrh na & ma\c{s}\c{s}\textbf{iinaa} &\textbf{n}m\textbf{o}\c{s}\c{s}\textbf{uu}\\ 
     \hline
     'Entuuma & ma\c{s}\c{s}\textbf{iituu} & \textbf{t}m\textbf{o}\c{s}\c{s}\textbf{uu}\\ 
     \hline
     Huuma & ma\c{s}\c{s}\textbf{uu} & \textbf{y}m\textbf{o}\c{s}\c{s}\textbf{uu}\\ 
     \hline
    \end{tabular}
\end{center}

\section{Dérivation des verbes géminés}
Il reste maintenant uniquement à dire un mot sur la dérivation de ces verbes. Il n'y a pas de grande différence pour dériver les verbes géminés ou les verbes sains, les grands principes restent les mêmes\footnote{Notez que certaines des formes que je donne ici ne sont pas très utilisées. Je vous les donne quand même pour que vous puissiez vous faire une idée sur la dérivation des verbes géminés.}. 

\begin{center}
    \begin{tabular}{||c | c | c||}
     \hline
     \textbf{Base} & \textbf{mass} & \textbf{ma\c{s}\c{s}} \\
     \hline\hline
     \textbf{Causatif} & masses & ma\c{s}\c{s}a\c{s} \\
    \hline
    \textbf{Passif} & tmass & tma\c{s}\c{s} \\
   \hline
   \textbf{Réflexif} & tmasses & tma\c{s}\c{s}a\c{s} \\
  \hline
  \textbf{Réciproque} & tmééses & tmaa\c{s}e\c{s} \\
 \hline
    \end{tabular}
\end{center}

Plusieurs points d'attention sur le tableau ci-dessous: 
\begin{itemize}
    \item Vous remarquerez que la consonne doublée de la base se comporte bien comme deux consonnes distinctes lors de la dérivation : certains dérivés séparent bien une consonne de sa jumelle.
    \item La forme \textbf{passive} prend bien un \textbf{/t/} plutôt qu'un \textbf{/te/}, puisque le base commence par une \underline{consonne suivie d'une voyelle} plutôt que par deux consonnes.
    \item Vous pourrez potentiellement rencontrer des formes alternatives de certains dérivés, où certaines voyelles seront changées\footnote{A Sfax par exemple, on dira plutôt \textbf{/ma\c{s}\c{s}e\c{s}/} au lieu de \textbf{/ma\c{s}\c{s}a\c{s}/}. J'ai choisi dans ce chapitre de privilégier la version tunisoise.}. Cela n'a aucune influence sur la sémantique.
\end{itemize}

La conjugaison reste la même pour l'ensemble de ces verbes dérivés. Référez-vous simplement au chapitre \ref{DérivesVerbes} pour savoir comment les conjuguer.


\section*{Dialogue}
\section*{Vocabulaire}