\chapter{Les articles définis}
\chapterletter{J}usqu'à maintenant, vous vous êtes peut-être insurgés intérieurement quand j'ai utilisé la forme définie de certains noms communs. Le temps est venu pour moi de m'excuser en vous apprenant comme dire \textbf{le} soleil, ou \textbf{la} lune !

\section{Un peu d'histoire}
Le système arabe pour définir le caractère défini ou indéfini d'un objet n'est en soi pas très compliqué : 
\begin{itemize}
    \item Pour exprimer le fait qu'un nom est indéfini, aucun article n'est nécessaire : il suffit de ne rien mettre ;
    \item Pour eprimer le fait qu'un nom est défini, il suffit de rajouter l'article \textbf{el} (c'est le préfixe qui apparaît souvent dans les noms arabes, comme par exemple au début de mon nom (\textbf{El}louze) ou au début de noms communs d'origine arabe comme \textbf{al}cool ou \textbf{al}gèbre.
\end{itemize}

Il existe juste une toute petite subtilité de \textbf{prononciation}, qui fait que le \textbf{l} peut être remplacé par la première lettre du nom commun qu'il qualifie (nous en avons déjà parlé, il s'agit d'une assimilation, cf. \ref{Assimilation}).

De ce fait, en \textbf{arabe}, on parlera deux types de \textbf{el} différente : les \textbf{solaires} et les \textbf{lunaires}. Ce nom prend son origine dans les mots \textit{soleil} (\RL{شمس}) et \textit{lune} (\RL{قمر}), qui se trouvent être respectivement un mot qui commence par une lettre assimilant le \textbf{l} et un mot qui commence par une lettre ne l'assimilant pas.

Cette distinction \textbf{solaire / lunaire} est relativement simple à apprendre, car elle ne se fait que sur la base de la première consonne du mot suivant \textbf{el} : c'est-à-dire qu'il suffit de connaître l'ensemble des consonnes concernées par l'assimilation, et l'ensemble des consonnes non concernées.

Le \textbf{tunisien} reprend d'ailleurs le même système, sauf qu'il a été légèrement adapté au fil du temps pour correspondre à la prononciation. Ainsi, les consonnes \textbf{solaires} et \textbf{lunaires} sont légèrement différentes, et le tunisien a introduit une nouvelle innovation grammaticale, avec une manière de s'extirper des cas où l'ajout d'un \textbf{el} provoque un cluster de consonnes trop difficiles à enchaîner.

\section{La forme solaire}
\label{Forme solaire}
Commençons par le cas de loin le plus simple : la forme \textbf{solaire}. 
Cette forme s'applique \textbf{systématiquement} pour les noms commençant par les consonnes suivantes : 

\begin{center}
    \textbf{t, \th, j, d, \dh, r, z, s, \v{s}, l, n, \c{t}, \c{\dh}, \c{s}} 
\end{center}

Dans ce cas, il suffit juste d'ajouter \textbf{e + la première consonne} du nom pour former la forme définie\footnote{Au sens strict, le \textbf{e} est porté par un coup de glotte \textbf{/'/}.}. Voici une liste d'exemples (un par lettre) :

\begin{center}
\begin{tabular}{||c | c | c||}
 \hline
 \textbf{Forme indéfinie} & \textbf{Forme définie} & \textbf{Traduction}\\
 \hline\hline
  Talvza & Et-talvza & La télévision \\
 \hline
  \th a\c{a}leb & E\th-\th  a\c{a}leb & Le renard \\
 \hline
  Jarrééya & Ej-jarrééya & Le matelas \\
 \hline
  Dabbuuza & Ed-dabbuuza & La bouteille \\
 \hline
  \dh iib & E\dh-\dh iib & Le loup \\
 \hline
  Rommèèn & Er-rommèèn & La grenade (\textit{fruit}) \\
 \hline
  Zituun & Ez-zituun & L'olive \\
 \hline
  Salluum & Es-salluum & L'échelle \\
 \hline
  \v{S}ams & E\v{s}-\v{s}ams & Le soleil \\
 \hline
  Lomja & El-lomja & Le goûter \\
 \hline
  Na\c{a}néé\c{a} & En-na\c{a}néé\c{a} & La menthe \\
 \hline
  \c{T}aawla & E\c{t}-\c{t}aawla & La table \\
 \hline
  \c{\dh}lèèm & E\c{\dh}-\c{\dh}lèèm & L'obscurité \\
 \hline
  \c{S}abuura & E\c{s}-\c{s}abuura & Le tableau noir \\
 \hline
\end{tabular}    
\end{center}

Si vous avez du mal à retenir les consonnes, essayez de vous rappeler qu'elles ne sont pas choisies au hasard : ce sont les consonnes \textbf{qui ont du mal à s'enchaîner avec un /l/} au niveau de la prononciation\footnote{Si vous vous êtes intéressé à l'endroit de l'articulation de ces consonnes, vous aurez remarqué que ce sont en réalité la totalité des consonnes du tunisien qui sont \textbf{dentales, alvéolaires ou post-alvéolaires}, donc articulées très proches du \textbf{/l/}.}.

\section{La forme lunaire}
\label{Forme lunaire}
La forme \textbf{lunaire} n'est pas plus compliqué à former que la forme solaire, cependant les règles de son application sont un tout petit peu plus strictes : 
\begin{itemize}
    \item Elle concerne toutes les \textbf{consonnes non concernées par la forme solaire} (cf. \ref{Forme solaire}) hormis \textbf{/'/};
    \item Elle n'est généralement pas utilisée si le nom commun \textbf{commence par une consonne suivie d'une voyelle}.
\end{itemize}

Si on fait le récapitulatif, cela veut dire que la forme \textbf{lunaire} s'applique à ces consonnes-ci\footnote{Pareillement que pour la forme \textbf{solaire} (cf. \ref{Forme solaire}), vous remarquerez sans doute que ce sont les consonnes dont les places d'articulation sont situées suffisamment loin du \textbf{/l/}, c'est-à-dire qu'elles ne sont \textbf{ni dentales, ni, alvéolaires, ni post-alvéolaires}.} : 

\begin{center}
    \textbf{
    b, p, \textcrh, x, \c{a}, \v{r}, f, v, q, g, k, m, h, w, y
    }
\end{center}

Dans le cas où la forme \textbf{lunaire} s'applique, il suffit d'ajouter \textbf{/el/} pour former la forme définie\footnote{Idem, au sens strict, le \textbf{/e/} est porté par un coup de glotte \textbf{/'/}.}. Voici une liste d'exemples (un par lettre) :

\begin{center}
\begin{tabular}{||c | c | c||}
 \hline
 \textbf{Forme indéfinie} & \textbf{Forme définie} & \textbf{Traduction}\\
 \hline\hline
  Babuur & El-babuur & Le bateau \\
 \hline
  Pisiin & El-pisiin & La piscine \\
 \hline
  \textcrh ii\c{t} & El-\textcrh ii\c{t} & Le mur \\
 \hline
  Xaatem & El-xaatem & La bague \\
 \hline
  \c{A}iin & El-\c{a}iin & L'oeil \\
 \hline
  \v{R}amza & El-\v{r}amza & Le clin d'oeil \\
 \hline
  Far\c{t}a\c{t}\c{t}u & El-far\c{t}a\c{t}\c{t}u & Le papillon de nuit \\
 \hline
  Viranda & El-viranda & La véranda \\
 \hline
  Qa\c{t}\c{t}uus & El-qa\c{t}\c{t}uus & Le chat \\
 \hline
  Gamra & El-gamra & La lune \\
 \hline
  Korraasa & El-korraasa & Le cahier \\
 \hline
  Mel\textcrh & El-mel\textcrh & Le sel \\
 \hline
  Hendi & El-hendi & La figue de barbarie \\
 \hline
  Wuraa\c{t}a & El-wuraa\c{t}a & La daurade \\
 \hline
  Yajuura & El-yajuura & La brique \\
 \hline
\end{tabular}    
\end{center}

\section{La forme terrestre}\label{FormeTerrestre}
La forme \textbf{terrestre}\footnote{J'ai sorti le nom de mon chapeau, donc vous le verrez sans doute nul part ailleurs. Mais j'en suis particulièrement fier, j'ai réussi à trouver un nom qui reste dans le champ lexical des astres, et dont la forme correspondante s'applique quand même au mot arabe qui veut dire \textit{Terre} (\textbf{'Ar\dh /\RL{أرض}}) !} est une innovation du tunisien, qui tient ses origines dans la forme \textbf{lunaire}, qui a été adaptée au niveau de la prononciation pour éviter les gros clusters de consonnes. \newpage

Ainsi, la forme \textbf{terrestre} s'applique dans ces cas-ci : 
\begin{itemize}
    \item La première consonne du nom à qualifier est une consonne \textbf{lunaire} (cf. \ref{Forme lunaire}) \underline{ou} un \textbf{coup de glotte /'/};
    \item Le nom à qualifier commence par \textbf{un enchaînement de deux consonnes}.
\end{itemize}\vspace{0.25cm}

Cependant, retenez également que la forme \textbf{terrestre} est dérivée de la formule \textbf{lunaire}. Ainsi, dans certains cas vous pourrez entendre des locuteurs, notamment ceux qui ont tendance à détacher les syllabes, utiliser la forme \textbf{lunaire} historique\footnote{On verra aussi que dans certains cas, on sera obligés d'abandonner la forme terrestre, typiquement quand on souhaite préfixer une consonne à l'article défini (le cas de la particule \textbf{fi} (\textit{dans})), mais ce sera une leçon pour un autre jour.}.

Dans le cas où la forme \textbf{terrestre} s'applique : 
\begin{itemize}
    \item Pour les consonnes \textbf{lunaires}, sauf pour \textbf{/w/} et \textbf{/y/}, il suffit d'ajouter \textbf{/le/};
    \item Pour le \textbf{coup de glotte}, on remplace \textbf{/'/} par \textbf{/l/};
    \item Pour \textbf{/w/}, on remplace le théorique \textbf{/lew/} par \textbf{/luu/};
    \item Pour \textbf{/y/}, on remplace le théorique \textbf{/ley/} par \textbf{/lii/}.
\end{itemize}\vspace{0.5cm}

Voici une liste d'exemples (un par lettre)\footnote{Pour le mot en \textbf{/v/}, je dois avouer que j'ai un peu triché, \textbf{vwayèèl} n'est pas vraiment utilisé.} : 

\begin{center}
\begin{tabular}{||c | c | c||}
 \hline
 \textbf{Forme indéfinie} & \textbf{Forme définie} & \textbf{Traduction}\\
 \hline\hline
  'Ar\c{\dh} & L-ar\c{\dh} & La terre / La Terre \\
 \hline
  Bruudu & Le-bruudu & Le potage \\
 \hline
  Proof & Le-proof & Le professeur \\
 \hline
  \textcrh luu & Le-\textcrh luu & Les pâtisseries \\
 \hline
  Xzééna & Le-xzééna & L'armoire \\
 \hline
  \c{A}rèè & Le-\c{a}rèè & La nudité \\
 \hline
  \v{R}nèè & Le-\v{r}nèè & Le chant \\
 \hline
  Fluus & Le-fluus & L'argent \\
 \hline
  Vwayèèl  & Le-vwayèèl  & La voyelle \\
 \hline
  Qraaya & Le-qraaya & Les études \\
 \hline
  Glaas & Le-glaas & La glace \\
 \hline
  Klèèm & Le-klèèm & La parole \\
 \hline
  M\c{t}ar & Le-m\c{t}ar & La pluie \\
 \hline
  Hlèèl & Le-hlèèl & Le croissant de lune\\
 \hline
  Wraaq & L-uuraaq & Les feuilles \\
 \hline
  Ymiin & L-iimiin & La droite \\
 \hline
\end{tabular}    
\end{center}

Et voilà, avec tout ça, vous pourrez décemment exprimer le caractère défini d'un objet !

\section*{Vocabulaire}
