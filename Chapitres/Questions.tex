\chapter{Former des questions}
\chapterletter{D}ans ce chapitre, nous allons voir comment former des questions en tunisien, aussi bien les questions totales ou partielles. Vous pourrez enfin demander qui a mangé votre déjeuner que vous aviez laissé sur la table !

\section{Un peu d'histoire}
Comme à notre habitude, attardons-nous un peu sur la langue mère du tunisien, pour essayer de comprendre comment la structure actuelle se rattache à la structure historique.

En arabe standard, les questions se forment relativement facilement. Que l'on forme des \textbf{questions totales} (auxquelles on ne peut répondre que par oui ou non) ou des \textbf{questions partielles} (qui demandent à l'interlocuteur une information), la structure est sensiblement identique. Dans tous les cas, la structure et l'ordre de la phrase demeure identique à la phrase déclarative, on ajoute en début de phrase le marqueur correspondant à la question qu'on souhaite poser, et on termine par omettre l'élément sur laquelle la question est posée en cas de question partielle.

Si nous devons faire le parallèle avec le français, les structures peuvent se montrer relativement similaire.

\begin{table}[ht]
\begin{tabularx}{\textwidth}{||X | X | X||}
 \hline
 Français & Arabe & Trad. Littérale \\
 \hline\hline
 Le garçon a mangé une pomme & \RL{أكل الولد تفّاحة} & A-mangé le-garçon pomme \\
 \hline
 Est-ce que le garçon a mangé une pomme ? & \RL{هل أكل الولد تفّاحة؟} & Est-ce-que a-mangé le-garçon pomme ? \\
 \hline
 Qui a mangé une pomme ? & \RL{من أكل تفّاحة ؟} & Qui a-mangé pomme ? \\
 \hline
 Qu'a mangé le garçon ? & \RL{ماذا أكل الولد ؟} & Quoi a-mangé le garçon ? \\
 \hline
\end{tabularx}
\end{table}

Comme vous pouvez le constater, la structure est assez semblable à ce qu'on peut retrouver en français (à la structure de la phrase déclarative près bien sûr, rappelez-vous que l'arabe standard est une langue \textbf{VSO}). J'oserai même dire que la structure est plus simple en arabe car moins permissive en français : on n'autorise typiquement aucune inversion verbe-sujet, comme dans la phrase "As-tu mangé ?". 

\textbf{En tunisien} par contre, l'histoire est un tout petit peu plus compliquée. Si vous vous intéressez à la linguistique, vous savez déjà que les structures des questions sont souvent le domaine dans lesquelles les structures des phrases est le plus libre. 

D'après les recherches que j'ai faites, lors de l'évolution d'une langue depuis un schéma de structure vers un autre, les phrases déclaratives ont plus de facilité à figer et évoluer vers le nouveau schéma, laissant les phrases interrogatives stagner quelque part entre les deux schémas. C'est typiquement le cas du français : l'ancien français est connu pour sa grande liberté sur la structure des phrases, cette dernière s'étant spécialisée vers le \textbf{SVO} jusqu'à se cristalliser en français moderne ; la structure libre des phrases interrogatives en français doivent sans doute être un vestige de cette liberté structurelle.

Nous allons voir dans la suite que \textbf{le tunisien a probablement suivi une voie similaire}, et est devenu beaucoup plus permissif que l'arabe standard (vous aurez compris que c'est souvent le cas).

\section{L'emphase en tunisien}
Juste avant de parler des phrases interrogatives, j'aurais voulu parler de \textbf{l'emphase}.\footnote{C'est sans doute une notion assez avancée pour ce moment précis du cours. J'y consacre un chapitre entier plus loin (chapitre \ref{Emphase}).} 

Nous en parlons maintenant, car il s'agit en réalité d'une méthode \textbf{très populaire} qu'on les tunisophones d'appuyer leurs propos, et qui se manifeste notamment lorsque des questions sont posées, à tel point qu'il a tendance à torturer l'ordre de la phrase interrogative. 

En termes linguistiques, on dira que \textbf{le tunisien emploie des procédés d'emphase par dislocation du sujet}. Cela se traduit typiquement par une omission volontaire du sujet de la phrase, pour aller le catapulter vers la fin de celle-ci.

En réalité, en français moderne parlé, on peut retrouver de telles structures. 

\begin{itemize}
    \item \textbf{Sans emphase} : Pierre me fatigue.
    \item \textbf{Avec emphase} : Il me fatigue, Pierre.
\end{itemize}

En français, on remplace le sujet par le pronom correspondant, et le sujet original est transporté vers la fin de la phrase. Vous verrez qu'en tunisien il s'agit de la même technique, la seule différence étant que les pronoms personnels peuvent être omis (et d'où cette impression que le sujet est inversé).

Retenez toutefois que cela reste cela dit un procédé surtout employé à l'oral, et donc souvent associé à un registre moins soutenu.

Dans la suite de chapitre, je ne vous parlerai pas des structures interrogatives où le sujet se retrouve en fin de phrase, car ces structures sont systématiquement des structures emphatiques. Mais, si cela vous stimule dans votre apprentissage, sachez que nous verrons encore d'autres structures plus loin dans le cours qui vous feront paraître plus naturel à l'oral.

\section{Les questions totales}
Dans cette section, nous allons aborder les \textbf{questions totales}, qui sont les questions qui peuvent être simplement répondues par "oui" ou par "non". En d'autres termes, il s'agit des questions qui contiennent l'ensemble de  l'information d'une phrase déclarative.

En tunisien, les questions totales peuvent être formulées de deux façons différentes.

Dans ces deux formes-ci, il est possible de \textbf{suffixer la particule "\v{s}i" au verbe}. Cette particule ne change que très peu le sens de la phrase : on la retrouvera souvent lorsque le locuteur veut \textbf{exprimer son impatience} par exemple. Vous pouvez décider de la mettre ou de l'omettre systématiquement sans crainte, le changement sémantique est en réalité assez mineur.

Je n'ai pas fait d'études linguistiques sur les formes préférées des tunisiens. Je pense d'ailleurs que j'utilise moi-même alternativement l'une et l'autre, surement en fonction du mot sur lequel j'ai le plus envie d'insister et de là où j'ai envie de mettre l'accent dans ma phrase. Mon conseil : ne vous prenez pas la tête, et utilisez celle qui vous convient le mieux.

\subsection{Question totales : première forme}
La première forme, présente dans plusieurs autres langues comme le français et l'anglais, est simplement le changement d'intonation : \textbf{la phrase interrogative obéit aux mêmes règles que la phrase déclarative, le ton est simplement plus haut en fin de phrase.}

Comme évoqué au paragraphe précédent, vous pouvez aussi \textbf{suffixer la particule "\v{s}i" au verbe}.

En voici un exemple : 

\begin{table}[ht]
\begin{tabularx}{\textwidth}{||X | X ||}
 \hline
 Tunisien & Traduction \\
 \hline\hline
 Erraajel fhem. & L'homme a compris. \\
 \hline
 Erraajel fhem ? & L'homme a compris ?\\
 \hline
 Erraajel fhem\v{s}i ? & (idem)\\
 \hline
\end{tabularx}
\end{table}

Contrairement à sa contrepartie en français, cette forme-là en tunisien ne sonne pas aussi peu soutenue. Vous pouvez donc l'utiliser sans trop conscientiser le registre dans lequel vous vous exprimer. 

\subsection{Question totales : seconde forme}
La seconde forme pour les questions totales ressemble fortement à la première, \textbf{à la différence près de l'inversion du sujet et du verbe}. Toutefois, elle est strictement réservées aux \textbf{verbes transitifs}, c'est-à-dire les verbes qui nécessitent un complément d'objet. 


En voici un exemple : 

\begin{table}[ht]
\begin{tabularx}{\textwidth}{||X | X ||}
 \hline
 Tunisien & Traduction \\
 \hline\hline
 Lemraa \textcrh af\dh et eddars. & La femme a appris le cours. \\
 \hline
 \HB af\dh et lemraa eddars ? & La femme a-t-elle appris le cours ? \\
 \hline
 \HB af\dh et\v{s}i lemraa eddars ? & (idem)\\
 \hline
\end{tabularx}
\end{table}

Prenez bien garde au fait que cette forme ne soit utilisable qu'avec les verbes transitifs. Dans le cas d'une inversion du sujet avec son verbe \textbf{intransitif}, vous serez en réalité en train d'employer une \textbf{emphase}, et ne produirez donc pas \textit{stricto sensu} le même sens.

Sur un sujet tout autre, on pourra noter que c'est également une forme qu'on retrouve dans d'autres langues (le français en est un exemple). Elle tient ses racines de l'arabe standard (rappelez-vous, l'arabe standard est une langue \textbf{VSO}). 

\section{Les questions partielles et leurs marqueurs}
Demander confirmation de quelque chose, c'est bien beau; demander des informations qu'on a pas, c'est autre chose ! 

Dans cette section, nous abordons les \textbf{questions partielles}, qui sont les questions qui ne peuvent pas être simplement répondues par "oui" ou par "non" : l'interlocuteur, s'il daigne vous répondre, vous précisera l'heure de l'action, son lieu, son sujet, etc.

\subsection{Marqueurs interrogatifs}
Abordons en premier lieu les \textbf{marqueurs interrogatifs}. Il s'agit de l'ensemble des mots introduisant la question, et demandant une information particulière. Je nous propose aussi d'aborder les origines de ces marqueurs-là, cela vous aidera si vous êtes arabophone à vous projeter (lisez cette colonne-là de gauche à droite).

\begin{center}
\begin{tabular}{||c | c | c||}
 \hline
 Tunisien & Français & Origine Arabe\\
 \hline\hline
 \v{S} / 'È\v{s} & Quoi / Que & \RL{شيء} (un objet)\\
 \hline
 \v{S}nuwwa & Quoi (objet masculin ou neutre) & \RL{شيء} + \RL{هو} (lui)\\
 \hline
 \v{S}niyya & Quoi (objet féminin) & \RL{شيء} + \RL{هي} (elle)\\
 \hline
 \v{S}nuuma / \v{S}nuhuuma & Quoi (plusieurs objets) & \RL{شيء} + \RL{هم} (eux)\\
 \hline
 \v{S}kuun & Qui & \RL{شيء} + \RL{كون} (un être)\\
 \hline
 Waqtéé\v{s} & Quand & \RL{وقت} (temps) + \RL{شيء} \\
 \hline
 Kiféé\v{s} & Comment & \RL{كيف} (comment) + \RL{شيء} \\
 \hline
 Wiin / Fiin & Où & \RL{أين} (où) (+ \RL{في} (dans))\\
 \hline
 Min wiin / Mniin & Depuis où & \RL{من} (depuis) + \RL{أين}\\
 \hline
 Lwiin & Vers où & \RL{إلى} (vers) + \RL{أين}\\
 \hline
 \c{A}lèè\v{s} & Pourquoi & \RL{على} (sur) + \RL{شيء}\\
 \hline
 Lwèè\v{s} & Pourquoi & \RL{ل} (pour) + \RL{شيء}\\
 \hline
 Anna & Quel / Lequel & ?\\
 \hline
\end{tabular}    
\end{center}

Vous avez sans doute remarqué que la plupart de ces marqueurs comportent le marqueur \textbf{\v{s}}, provenant du mot arabe \RL{شيء} qui désigne un objet ou une chose. C'est sur ce sens que la plupart des marqueurs se sont formés.\footnote{Fait relativement drôle, \RL{شيء} a évolué en tunisien pour donner le mot \textbf{\v{s}èy}, dont un des sens est toujours "objet/chose", mais dont le sens premier est "rien", ce qui en fait un mot \textbf{énantiosémantique} (il s'agit d'un mot qui possède deux sens qui sont antonymes).}

\textbf{Les marqueurs interrogatifs se positionnent généralement en début de phrase}, même s'il est possible à l'oral de leur donner la place du groupe grammatical qu'ils remplacent.

\subsection{Structure des questions partielles}
XXX

\section{Récapitulatif}
XXX

\section*{Vocabulaire}
XXX
