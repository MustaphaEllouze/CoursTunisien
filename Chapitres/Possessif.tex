\chapter{Le possessif}
\chapterletter{Q}ue ce soit pour parler de son métier, de sa ville natale ou des objets qui nous appartiennent, savoir exprimer le possessif dans une langue est capitale. Regardons ensemble comment celui-ci s'exprime en tunisien.

\section{Un peu d'histoire}
En arabe standard, le possessif s'exprime de façon très régulière, par simple suffixation. Il existe un suffixe différent par pronom personnel, et on peut donc exprimer simplement à qui appartient quelque chose. 

J'oserai même dire qu'il s'exprime plus simplement qu'en français : là où le français fait la distinction entre genre et nombre (\textit{mon, ma, mes}), ce n'est pas le cas en arabe. Ainsi, la donnée du genre et du nombre n'est pas intégrée au suffixe, l'information est déjà comprise dans le nom (pourquoi avoir une information redondante me direz-vous).

En \textbf{tunisien}, l'histoire est un tout petit peu plus compliquée, comme \linebreak d'habitude. Le passage du temps a fait que la prononciation a évolué à différentes vitesses dans des cas de figures différents : la prononciation des noms communs féminins et des possessifs associés en est un exemple. \footnote{On verra également que c'est le cas pour certains noms communs se finissant par une lettre spécifique, en l'occurrence le \textbf{i}.}

Les arabophones voient certainement de quoi je parle. Il s'agit la non-prononciation des \RL{ة} en fin de mot : il se trouve que l'évolution de la langue a fait que l'arrêt de prononciation de cette lettre ne s'est fait que si la lettre termine le mot, ce qui n'est pas le cas du possessif. 

Pour les non-arabophones, afin d'illustrer cet exemple-là dans un cas de figure, nous pouvons parler de cette même évolution (l'arrêt de la prononciation de la dernière lettre d'un mot féminin) en prenant l'exemple de la prononciation des prénoms d'origine arabe en Afrique subsaharienne. Il se trouve que dans ces pays, la lettre finale est toujours prononcée dans les prénoms féminins, ce qui fait que le prénom \RL{أمينة} se prononcera \textbf{Amina} au Maghreb, mais \textbf{Aminatou} dans les pays plus au sud.

Tout ceci pour dire que : le \textbf{tunisien} fait de nos jours la distinction des genres sur l'expression du possessif, mais que cette distinction n'est pas le fruit d'une différence purement grammaticale et structurelle. La logique derrière la formation de ces possessifs est donc la même.

\section{La forme courte du possessif}
En \textbf{tunisien}, il existe deux manières d'exprimer la possessif : une forme \textbf{courte}, qui dérive de l'arabe standard, et une forme \textbf{longue}, spécifique au tunisien.

La forme courte du possessif est très compacte, et est donc assez privilégiée. Elle a cependant le mauvais goût de s'exprimer de trois façons différentes, séparants ainsi les mots en trois groupes distincts :
\begin{itemize}
    \item Les noms \textbf{féminins} ;
    \item Les noms \textbf{masculins ne se terminant pas par une voyelle} ;
    \item Les noms \textbf{masculins se terminant par une voyelle}.
\end{itemize}

Commençons d'abord par les noms féminins et les noms masculins ne se terminant pas par une voyelle, ces deux groupes formant en quelque sorte la forme \textit{régulière} du possessif.

Voici les formes possessives pour les mots \textbf{xobz} (du pain) et \textbf{xobza} (un pain, une baguette) :

\begin{center}
\begin{tabular}{||c | c | c||}
 \hline
 Pronom & \textbf{xobz} & \textbf{xobza}\\
 \hline\hline
 'Éna & xobz\textbf{i} & xobz\textbf{ti}\\
 \hline
 'Enti & xobz\textbf{ek} & xobz\textbf{tek}\\
 \hline
 Huwwa & xobz\textbf{u} & xobz\textbf{tu}\\
 \hline
 Hiyya & xobz\textbf{ha} & xobz\textbf{etha}\\
 \hline
 'A\textcrh na & xobz\textbf{na} & xobz\textbf{etna}\\
 \hline
 'Entuuma & xobz\textbf{kom} & xobz\textbf{etkom}\\
 \hline
 Huuma & xobz\textbf{hom} & xobz\textbf{ethom}\\
 \hline
\end{tabular}    
\end{center}

Comme vous pouvez le remarquer, les deux formes sont globalement très similaires, à cela près que la forme du possessif pour les mots féminins remplacent le \textbf{a} terminal du mot par un \textbf{t}.

Également, il faudra noter la présence d'un \textbf{e} pour 4 des 7 personnes au féminin : \textbf{hiyya, 'a\textcrh na, 'entuuma, huuma}. À cause du remplacement du \textbf{a} par un \textbf{t}, un cluster de trois consonnes successives serait apparu sans l'ajout de ce \textbf{e}. En tant que moyen mnémotechnique, vous pouvez donc retenir que si la terminaison commence par une consonne, alors vous devez remplacer le \textbf{a} par un \textbf{et} (et non un \textbf{t}). 

Ceci en poche, vous êtes maintenant capable d'exprimer le possessif pour la quasi-totalité des noms communs ! Reste maintenant à l'exprimer pour les mots masculins finissant par une voyelle.

Prenons pour exemple ces trois mots-ci : \textbf{sbéédrii} (des chaussures de \linebreak sport\footnote{Si vous avez l'esprit affûté, vous aurez remarqué que c'est littéralement le mot \textit{espadrille} qui a été importé et déformé. Le pluriel et le singulier se prononcent de la même façon.}), \textbf{baakuu} (un paquet \footnote{J'espère que votre esprit était affûté cette fois aussi, c'est une fois de plus le même mot}) et \textbf{k\v{o}paa} (un compas\footnote{Je vous laisse deviner.}). Vous remarquerez juste la petite subtilité sur la longueur des voyelles et des consonnes  pour la première personne (les autres personnes ayant des terminaisons identiques).

\begin{center}
\begin{tabular}{||c | c | c | c ||}
 \hline
 Pronom & \textbf{sbéédrii} & \textbf{baakuu} & \textbf{k\v{o}paa}\\
 \hline\hline
 'Éna & sbéédri\textbf{yya} & baakuu\textbf{ya}& k\v{o}paa\textbf{ya}\\
 \hline
 'Enti & sbéédrii\textbf{k} & baakuu\textbf{k}& k\v{o}paa\textbf{k}\\
 \hline
 Huwwa & sbéédrii\textbf{h} & baakuu\textbf{h}& k\v{o}paa\textbf{h}\\
 \hline
 Hiyya & sbéédrii\textbf{ha} & baakuu\textbf{ha} & k\v{o}paa\textbf{ha}\\
 \hline
 'A\textcrh na & sbéédrii\textbf{na} & baakuu\textbf{na}& k\v{o}paa\textbf{na}\\
 \hline
 'Entuuma & sbéédrii\textbf{kom} & baakuu\textbf{kom}& k\v{o}paa\textbf{kom}\\
 \hline
 Huuma & sbéédrii\textbf{hom} & baakuu\textbf{hom}& k\v{o}paa\textbf{hom}\\
 \hline
\end{tabular}    
\end{center}

Il est intéressant de noter que les seules différences se trouvent dans les trois premières personnes, qui se trouvent également être les seules trois personnes pour lesquelles la terminaison \textbf{commencent par une voyelles}. Vous commencez à comprendre maintenant la logique derrière : le tunisien n'aime pas enchaîner les voyelles, et donc l'évolution de la prononciation à travers le temps s'est arrangée pour ne pas créer de telles structures. 

Si vous voulez aller encore plus loin dans votre apprentissage de l'histoire de l'évolution du tunisien, vous pouvez par exemple retenir que la terminaison pour \textbf{huwwa} était initialement la même pour l'ensemble des mots, peu importe leur genre ou leur lettre finale. La terminaison en \textbf{arabe standard} était \textbf{-hu} : on retrouve la première moitié dans le possessif des noms se terminant par une voyelle, et la seconde moitié dans le possessif des noms ne se terminant pas par une voyelle.

\section{La forme longue du possessif}
La seconde forme du possessif, la forme \textbf{longue}, est une innovation du \textbf{tunisien}. Elle s'exprime par l'ajout du mot \textbf{mtèè\c{a}}, venant du mot arabe \RL{متاع} qui veut dire \textit{affaires, bagages}. De nos jours, \textbf{mtèè\c{a}} a perdu son sens lorsqu'employé tout seul en tunisien.

Pour exprimer cette forme longue, il faut procéder comme suit : 
\begin{itemize}
    \item Utiliser la forme \textbf{définie} du nom à qualifier ;
    \item Mettre \textbf{mtèè\c{a}} sous la forme \textbf{possessive courte} (c'est un nom commun masculin);
    \item Juxstaposer le nom à qualifier et \textbf{mtèè\c{a}} (avec sa terminaison appropriée).
\end{itemize}

Ce qui donne, en utilisant l'exemple avec \textbf{xobz} : 

\begin{center}
\begin{tabular}{||c | c | c||}
 \hline
 Pronom & \textbf{xobz - Forme courte} & \textbf{xobz - Forme longue}\\
 \hline\hline
 'Éna & xobz\textbf{i} & \textbf{el}xobz \textbf{mtèè\c{a}i}\\
 \hline
 'Enti & xobz\textbf{ek} & \textbf{el}xobz \textbf{mtèè\c{a}ek}\\
 \hline
 Huwwa & xobz\textbf{u} & \textbf{el}xobz \textbf{mtèè\c{a}u}\\
 \hline
 Hiyya & xobz\textbf{ha} & \textbf{el}xobz \textbf{mtèè\textcrh\textcrh a}\\
 \hline
 'A\textcrh na & xobz\textbf{na} & \textbf{el}xobz \textbf{mtèè\c{a}na}\\
 \hline
 'Entuuma & xobz\textbf{kom} & \textbf{el}xobz \textbf{mtèè\c{a}kom}\\
 \hline
 Huuma & xobz\textbf{hom} & \textbf{el}xobz \textbf{mtèè\textcrh\textcrh om}\\
 \hline
\end{tabular}    
\end{center}

Le seul point sur lequel je souhaite attirer votre attention est la forme particulière pour \textbf{hiyya} et \textbf{huuma}. Il s'agit là juste d'une évolution de la prononciation, pour s'abstraire de la suite de consonne \textbf{\c{a}h} qui est dure à prononcer. Nous en avons déjà parlé plus tôt, il s'agit de l'assimilation (cf. \ref{Assimilation})\footnote{En toute rigueur, il serait possible de garder la même orthographe et d'admettre que cette suite de consonne est simplifiée en \textcrh\textcrh, mais je ne suis pas convaincu de la décorrélation de la prononciation et de l'orthographe, même si certaines langues se l'autorisent pour des raisons historiques généralement (le français et l'anglais ont sont de parfaits exemples).}.

Pour mieux appréhender l'usage de cette forme, on peut en donner un équivalent en français (certes un peu lourd), qui serait quelque chose du style : 

\begin{center}
    \textbf{elxobz mtèè\c{a}i <-> Le pain à moi}
\end{center}

Une hypothèse que j'ai , qui ne me semble ne pas être si folle que ça, et qui permet de mieux comprendre l'origine de cette structure innovante est la suivante : \textbf{il s'agirait initialement d'une phrase nominale}, où l'objet à qualifier est le sujet et \textbf{mtèè\c{a}} le complément, et qui a fini par être tant utilisée que son intégration directement en tant que groupe nominal a été autorisé.

D'ailleurs, on peut même noter que, de nos jours, \textbf{utiliser la forme longue toute seule constitue une phrase nominale valide} (le verbe être étant bien entendu sous-entendu comme dans toute phrase nominale).

Voilà, avec tout ceci, vous devrez maintenant être capable de désigner les choses qui vous appartiennent !  

\section*{Vocabulaire}
XXX