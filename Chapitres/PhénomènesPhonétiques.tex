\chapter{Phénomènes phonétiques en tunisien}
\chapterletter{A}vant d'aborder votre premier point de grammaire, il me paraît essentiel de parler de certains phénomènes phonétiques que vous pourrez rencontrer, qui affecteront a minima la prononciation, et jusqu'à la justification d'une retranscription différente de certaines formes grammaticales.

\section{Décomposition des mots en syllabes}\label{DécompositionSyllabes}
Pour la prononciation des mots que vous allez rencontrer, il est important de savoir les découper en syllabes. Pour cela, la linguistique s'intéresse souvent à la \textbf{structure syllabique} d'une langue, c'est-à-dire quels sont les successions autorisées (et interdites donc) de sons. Par la suite, c'est bien la structure des syllabes et leur agencement entre elles qui définissent comme se prononce un mot. Au-delà des sons présents dans une langue, c'est bien la structure syllabique et la structure des mots qui donne son \textit{feeling} à une langue.

Ainsi, en \textbf{français} par exemple, on pourra retrouver les syllabes \textbf{/ba/}, \textbf{/bar/} et \textbf{/bwar/}, mais on ne pourra jamais retrouver une syllabe du style \textbf{/mlorj/}, même si on n'aurait pas spécialement de difficulté à la prononcer.

Rassurez-vous, il ne s'agit pas dans cette section de disséquer totalement les structures des syllabes en tunisien, simplement d'aborder des points importants qui vous permettront de lire un mot \textit{de la bonne façon}. 

Pour ce faire, voici les points importants à retenir : 
\begin{itemize}
    \item Le tunisien, comme l'arabe, fait commencer \textbf{toutes} ses syllabes par \textbf{une consonne}\footnote{Dans le cas des mots qui \textit{semblent} commencer par une voyelle, un \textbf{coup de glotte /'/} est prononcé en début de syllabe.}.
    \item Les \textbf{consonnes doubles} ne sont \textbf{jamais} dans la même syllabe, sauf si c'est la \underline{dernière} syllabe du mot.
    \item De façon plus générale, le tunisien hait les clusters de consonnes, et aura tendance tant que faire se peut de \textbf{séparer} les consonnes proches pour les mettre dans des syllabes différentes.
    \item Vous ne trouverez jamais \textbf{trois} consonnes successives, ni \textbf{deux} voyelles successives.
\end{itemize}

De ces quelques règles-ci, voici un petit algorithme qui vous aidera dans la majorité des cas :
\begin{itemize}
    \item Si une consonne est encadrée par deux voyelles, alors cette consonne marque le début d'une nouvelle syllabe ;
    \item Si une consonne est répétée dans une syllabe non-terminale, alors la consonne répétée marque le début d'une nouvelle syllabe ;
    \item Si deux consonnes sont encadrées par deux voyelles, alors la deuxième consonne marque le début d'une nouvelle syllabe ;
    \item Si trois consonnes se suivent, alors les deuxième et troisième consonnes démarrent une nouvelle syllabe avec la voyelle qui les suit.
\end{itemize}

Voici quelques exemples pour vous faire la main :

\begin{center}
    \begin{tabular}{||c | c | c||} 
    \hline
    \textbf{Tunisien} & \textbf{Syllabes} & \textbf{Trad.}\\
    \hline\hline
    net\c{a}allmuu & net | \c{a}al | lmuu & \textit{nous apprenons}\\ 
    \hline
    tetekteb & te | tek | teb & \textit{elle est écrite}\\ 
    \hline
    ordinater & or | di | na | ter & \textit{ordinateur}\\ 
    \hline
    xzééna & xzéé | na & \textit{armoire}\\ 
    \hline
    xobz & xobz & \textit{du pain}\\ 
    \hline
    xobza & xob | za & \textit{une baguette}\\ 
    \hline
   \end{tabular}
\end{center}

\section{Position de l'accent tonique}
XXX

\section{Assimilation des consonnes}\label{Assimilation}
Parler de \textbf{\c{a}h > \textcrh\textcrh} et les autres assimilations qu'ils peut y avoir.

\section{Métathèse et simplification vocalique}
XXX

\section{Harmonie consonantique}
Pour être totalement franc, vous pouvez ignorer ce passage, sauf si vous êtes curieux ou que vous voulez apprendre à avoir un accent parfait en tunisien. 

Le tunisien présente une caractéristique assez particulière, qu'il partage avec d'autres langues d'origine arabe, qui est \textbf{l'harmonie consonantique}, plus particulièrement une \textbf{harmonie d'articulation secondaire}. Concrètement, cela veut dire que les \textbf{consonnes emphatiques}, qui sont pharyngalisées, ont tendance à "pharyngaliser" les consonnes qui sont autour. 

Par exemple, un mot comme \textbf{[mba\textrevglotstop \textschwa d] (après)} sera plutôt prononcé \textbf{[mb\super\textrevglotstop a\textrevglotstop \textschwa d\super\textrevglotstop]}. L'harmonie ne s'étend généralement qu'aux deux consonnes les plus proches d'un même mot, et s'étendent en quelques occasions à deux consonnes.

Ce détail n'a pas d'incidence sur la sémantique, puisque, comme l'exemple ci-dessus le montre, les sons qui sont produits (\textbf{[b\super\textrevglotstop]} et \textbf{[d\super\textrevglotstop]} ici) ne sont pas des phonèmes qui existent isolés de tout autre contexte consonantique. Il permettra par contre à votre accent d'être plus naturel.
