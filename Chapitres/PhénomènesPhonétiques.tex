\chapter{Phénomènes phonétiques en tunisien}
\chapterletter{A}vant d'aborder votre premier point de grammaire, il me paraît essentiel de parler de certains phénomènes phonétiques que vous pourrez rencontrer, qui affecteront a minima la prononciation, et jusqu'à la justification d'une retranscription différente de certaines formes grammaticales.

\section{Assimilation des consonnes}\label{Assimilation}
Parler de \textbf{\c{a}h > \textcrh\textcrh} et les autres assimilations qu'ils peut y avoir.

\section{Métathèse et simplification vocalique}
XXX

\section{Harmonie consonantique}
Pour être totalement franc, vous pouvez ignorer ce passage, sauf si vous êtes curieux ou que vous voulez apprendre à avoir un accent parfait en tunisien. 

Le tunisien présente une caractéristique assez particulière, qu'il partage avec d'autres langues d'origine arabe, qui est \textbf{l'harmonie consonantique}, plus particulièrement une \textbf{harmonie d'articulation secondaire}. Concrètement, cela veut dire que les \textbf{consonnes emphatiques}, qui sont pharyngalisées, ont tendance à "pharyngaliser" les consonnes qui sont autour. 

Par exemple, un mot comme \textbf{[mba\textrevglotstop \textschwa d] (après)} sera plutôt prononcé \textbf{[mb\super\textrevglotstop a\textrevglotstop \textschwa d\super\textrevglotstop]}. L'harmonie ne s'étend généralement qu'aux deux consonnes les plus proches d'un même mot, et s'étendent en quelques occasions à deux consonnes.

Ce détail n'a pas d'incidence sur la sémantique, puisque, comme l'exemple ci-dessus le montre, les sons qui sont produits (\textbf{[b\super\textrevglotstop]} et \textbf{[d\super\textrevglotstop]} ici) ne sont pas des phonèmes qui existent isolés de tout autre contexte consonantique. Il permettra par contre à votre accent d'être plus naturel.
