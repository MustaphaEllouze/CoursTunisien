\chapter{Les verbes défectueux}\label{VerbesDefectueux}
\chapterletter{N}ous avons vu lors des chapitres \ref{ConjSS} et \ref{VerbesGéminés} que l'arabe et le tunisien identifient des verbes dits \textbf{défectueux}, dont la conjugaison est légèrement différentes des verbes \textbf{sains}. Dans ce chapitre, nous allons nous attarder sur ces derniers groupes de verbes, qui comprennent de nombreux verbes importants du quotidien, et plusieurs verbes importés.

\section{Un peu d'histoire}
Nous l'avons déjà vu dans un chapitre précédent (chapitre \ref{VerbesGéminés}) : en \textbf{arabe}, les verbes se décomposent en plusieurs groupes en fonction des consonnes qui les composent.

Pour rappel, en grammaire \textbf{arabe}, on désigne comme défectueux l'ensemble des verbes qui contiennet au moins une lettre \textit{défectueuse}, c'est-à-dire un \textbf{/w/}, un \textbf{/y/} et/ou une \textbf{voyelle longue} (absence d'une des trois consonnes de la base). Ces trois lettres défectueses viennent modifier la conjugaison des verbes dans lesquels elles apparaissent\footnote{Le fait qu'une consonne manque ou que \textbf{/w/} et \textbf{/y/} soient des semi-voyelles y est bien sûr pour quelque chose.}, si bien que cela a valu la catégorisation de ces verbes dans un groupe bien distinct.

Ainsi, en \textbf{arabe}, on vient distinguer ces verbes-là en d'autres sous-groupes (en fonction de la conjugaison associée): 
\begin{itemize}
    \item Les verbes \textbf{assimilés}\footnote{Comme souvent, les dénominations françaises que j'ai choisies ne sont pas nécessairement les dénominations officielles, pour peu qu'il en existe...} \RL{الفعل المثال} qui ont une consonne défectueuse sur l'emplacement de la première consonne (par exemple \textbf{/waqafa/} \RL{وقف} \textit{s'arrêter});
    \item Les verbes \textbf{concaves} \RL{الفعل الأجوف} qui ont une consonne défectueuse sur l'emplacement de la deuxième consonne (par exemple \textbf{/qaala/} \RL{قال} \textit{dire});
    \item Les verbes \textbf{incomplets} \RL{الفعل الناقص} qui ont une consonne défectueuse sur l'emplacement de la dernière consonne (par exemple \textbf{/nasiya/} \RL{نسي} \textit{oublier})\footnote{Techniquement, il y a des groupes qui ne comprennent que des verbes avec deux consonnes défectueuses. Ces verbes sont tellement peu nombreux qu'il n'y a pas vraiment d'intrêt à les catégoriser (selon moi).}.
\end{itemize}

Bien évidemment, toutes les consonnes défectueuses n'apparaissent pas à la même fréquence aux mêmes emplacements. Ainsi, on trouvera souvent des voyelles longues \textbf{en milieu ou fin de mot}, alors que \textbf{/w/} et \textbf{/y/} apparaissent souvent au début du verbe.

En \textbf{tunisien}, le principe reste le même pour ces trois consonnes, cependant il me semble intéressant de mentionner ce qui suit: la plupart des verbes \textbf{arabe} comportant un \textbf{/'/} sont devenus \textit{en partie} défectueux puisque le coup de glotte disparaît au début et à la fin de tous les mots. On notera par exemple\linebreak \textbf{/'axa\dh a/} \RL{أخذ} (\textit{prendre}) qui est devenu \textbf{/x\dh èè/}. Dans la suite, j'analyserai donc ces verbes comme tels, bien qu'ils ne soient pas défectueux en \textbf{arabe}.

\section{Les verbes assimilés}
Les verbes \textbf{assimilés} sont les verbes dont la première consonne de leur base est une consonne défectueuse.

On dénombre alors deux sous-groupes de verbes assimilés : 
\begin{itemize}
    \item Les verbes en \textbf{/w/}, comme \textbf{wqef} (\textit{s'arrêter}) ;
    \item Les verbes en \textbf{/y/}, comme \textbf{ybes} (\textit{se durcir}) ;
\end{itemize}

Ces verbes sont dits \textbf{assimilés} car la première consonne \textit{tombe} souvent à certaines formes au profit d'une voyelle proche : \textbf{/w/} se transforme en \textbf{/u/}, et \textbf{/y/} se transforme en \textbf{/i/}.

En règle générale, les conjugaisons et les dérivés de ces verbe verront leur consonne défectueuse remplacée dès qu'elle sera encadrée de deux consonnes\footnote{Cela ne concerne donc que le \textbf{présent} pour la conjugaison, et que la \textbf{voix passive} pour les dérivés que nous avons vu au chapitre \ref{DérivesVerbes}}.

Nous pouvons voir ça avec la conjugaison de ces verbes, qui demeure régulière au passé : 

\conjugaison{wqef}
    {wqef\textbf{t}, \textbf{nuu}qef}
    {wqef\textbf{t}, \textbf{tuu}qef}
    {wqef, \textbf{yuu}qef}
    {w\textbf{e}qf\textbf{et}, \textbf{tuu}qef}
    {wqef\textbf{naa}, \textbf{nuu}qf\textbf{uu}}
    {wqef\textbf{tuu}, \textbf{tuu}qf\textbf{uu}}
    {w\textbf{e}qf\textbf{uu}, \textbf{yuu}qf\textbf{uu}}

\conjugaison{ybes}
    {ybes\textbf{t}, \textbf{nii}bes}
    {ybes\textbf{t}, \textbf{tii}bes}
    {ybes, \textbf{yii}bes}
    {y\textbf{e}bs\textbf{et}, \textbf{tii}bes}
    {ybes\textbf{naa}, \textbf{nii}bs\textbf{uu}}
    {ybes\textbf{tuu}, \textbf{tii}bs\textbf{uu}}
    {y\textbf{e}bs\textbf{uu}, \textbf{yii}bs\textbf{uu}}

Pour les dérivés, il n'y a qu'une seule forme qui est impactée par la consonne défectueuse : le dérivé de la voix \textbf{passive}. En l'occurrence : 
\begin{itemize}
    \item \textbf{wqef} devient \textbf{tuqef} ;
    \item \textbf{ybes} devient \textbf{tibes}\footnote{Remarquez pour ces deux verbes que la longueur de la voyelle n'est pas la même que la longueur de la voyelle du verbe simple conjugué à la deuxième personne du singulier au présent (\textbf{tuuqef}/\textbf{tiibes} vs. \textbf{tuqef}/\textbf{tibes}). Pour rappel, l'accent tonique n'est pas situé sur la même syllabe.}.
\end{itemize}

\section{Les verbes concaves}

Les verbes \textbf{concaves} sont les verbes dont la consonne centrale de leur base est une consonne défectueuse. En règle générale, on retrouvera plutôt une \textbf{voyelle longue} qu'un \textbf{/w/} ou un \textbf{/y/}\footnote{C'était déjà le cas en arabe.}.

On dénombre deux sous-groupes de verbes concaves : 
\begin{itemize}
    \item Les verbes qui prennent un \textbf{/u/} au présent, comme \textbf{qaal} (\textit{dire}), \textbf{mèèt} (\textit{mourir}) et \textbf{bèès} (\textit{embrasser}).
    \item Les verbes qui prennent un \textbf{/i/} au présent, comme \textbf{qaas} (\textit{mesurer}), \textbf{\v{s}èè\textcrh} (\textit{sécher}) et \textbf{mèèl} (se pencher).
\end{itemize}

Il n'y a priori pas de moyen de savoir si un verbe fait partie d'un groupe ou de l'autre, vous n'aurez donc le choix que d'apprendre par c\oe ur. 

Ces deux groupes se conjuguent de cette façon : 

\conjugaison{qaal}
    {q\textbf{o}l\textbf{t}, \textbf{n}q\textbf{uu}l}
    {q\textbf{o}l\textbf{t}, \textbf{t}q\textbf{uu}l}
    {qaal, \textbf{y}q\textbf{uu}l}
    {qaal\textbf{et}, \textbf{t}q\textbf{uu}l}
    {q\textbf{o}l\textbf{naa}, \textbf{n}q\textbf{uu}l\textbf{uu}}
    {q\textbf{o}l\textbf{tuu}, \textbf{t}q\textbf{uu}l\textbf{uu}}
    {qaal\textbf{uu}, \textbf{y}q\textbf{uu}l\textbf{uu}}

\conjugaison{qaas}
    {q\textbf{e}s\textbf{t}, \textbf{n}q\textbf{ii}s}
    {q\textbf{e}s\textbf{t}, \textbf{t}q\textbf{ii}s}
    {qaas, \textbf{y}q\textbf{ii}s}
    {qaas\textbf{et}, \textbf{t}q\textbf{ii}s}
    {q\textbf{e}s\textbf{naa}, \textbf{n}q\textbf{ii}s\textbf{uu}}
    {q\textbf{e}s\textbf{tuu}, \textbf{t}q\textbf{ii}l\textbf{uu}}
    {qaas\textbf{uu}, \textbf{y}q\textbf{ii}s\textbf{uu}}

Au niveau des verbes dérivés, on utilisera respectivement le \textbf{/w/} et le \textbf{/y/} comme \textit{lettres supports} si besoin\footnote{Je mets une étoile devant les formes qui n'existent pas seules. En l'occurrence, \textbf{fèè\textcrh} se rapporte à tout ce qui se rapporte au parfum.}:

\begin{center}
    \begin{tabular}{||c | c | c | c || c | c | c ||}
     \hline
     \textbf{Base} & \textbf{qaal} & \textbf{*fèè\textcrh} & \textbf{bèès} & \textbf{qaas} & \textbf{\v{s}èè\textcrh} & \textbf{mèèl} \\
     \hline\hline
     \textbf{Causatif} & - & fawwa\textcrh & - & qayyes & \v{s}ayye\textcrh & mayyel \\
    \hline
    \textbf{Passif} & tqaal & - & - & tqaas & - & - \\
   \hline
   \textbf{Réflexif} & - & tfawwa\textcrh & - & - & t\v{s}ayye\textcrh & - \\
  \hline
  \textbf{Réciproque} & - & - & *tbèèwes & - & - & tmèèyel \\
 \hline
    \end{tabular}
\end{center}

\section{Les verbes incomplets}

Les verbes \textbf{incomplets} sont les verbes dont la consonne finale de leur base est une consonne défectueuse. En règle générale, on retrouvera souvent une \textbf{voyelle longue}\footnote{Les autres consonnes défectueuses ont cessé d'être prononcées, comme \textbf{/nasiya/} qui est devenu \textbf{/nsèè/}.}. Ces verbes ont des origines étymologiques diverses, notamment plusieurs des verbes incomplets en \textbf{tunisien} proviennent de verbes avec des \textbf{/'/} en arabe (en position terminale ou finale).

On dénombre \textbf{cinq} sous-groupes : 

\begin{itemize}
    \item Les verbes qui se conjuguent avec un \textbf{/i/} au présent, qui dérivent de l'\textbf{arabe} de verbes incomplets se terminant par une voyelle longue et qui se conjuguent de la même manière : \textbf{m\v{s}èè - yem\v{s}ii} (\textit{marcher}, de \RL{مشى يمشي})\footnote{La plupart des verbes importés finissant par une voyelle terminent dans cette gatégorie.};
    \item Les verbes qui se conjuguent avec un \textbf{/u/} au présent, qui dérivent de l'\textbf{arabe} de verbes incomplets se terminant par une voyelle longue ou un \textbf{/w/} : \textbf{\textcrh bèè - ya\textcrh buu} (\textit{ramper}, de \RL{حبى يحبو}) ;
    \item Les verbes qui se conjuguent avec un \textbf{/è/} au présent, qui dérivent de l'\textbf{arabe} de verbes incomplets se terminant par un \textbf{/y/} : \textbf{nsèè - yensèè} (\textit{oublier}, de \RL{نسي ينسى}) ;
    \item Les verbes qui se conjuguent avec un \textbf{/a/} au présent, qui dérivent de l'\textbf{arabe} de verbes se terminant par \textbf{/'/} : \textbf{qraa - yaqraa} (\textit{lire/étudier}, \linebreak de \RL{قرأ يقرأ}) ; 
    \item Les verbes qui se conjuguent comme un verbe \textbf{assimilé} au présent, qui dérivent de l'\textbf{arabe} de verbes commençant par \textbf{/'/} : \textbf{klèè - yéékel} (\textbf{manger}, de \RL{أكل يأكل}).
\end{itemize}

De la même façon que pour les verbes concaves, à moins d'avoir fait de l'étymologie en amont, il n'y a pas moyen de savoir à quel groupe appartient quel verbe autrement qu'en l'apprenant par coeur\footnote{Hormis les verbes qui se conjuguent comme \textbf{qraa} qui finissent tous par \textbf{/aa/}, comme \textbf{raa} (\textit{voir}) et \textbf{braa} (\textit{guérir}).}. 

Ces groupes se conjuguent de cette façon :

\conjugaison{m\v{s}èè}
    {m\v{s}\textbf{iit} , \textbf{ne}m\v{s}\textbf{ii}}
    {m\v{s}\textbf{iit} , \textbf{te}m\v{s}\textbf{ii}}
    {m\v{s}èè , \textbf{ye}m\v{s}\textbf{ii}}
    {m\v{s}èè\textbf{t} , \textbf{te}m\v{s}\textbf{ii}}
    {m\v{s}\textbf{iinaa} , \textbf{ne}m\v{s}\textbf{iiw}}
    {m\v{s}\textbf{iituu} , \textbf{te}m\v{s}\textbf{iiw}}
    {m\v{s}èè\textbf{w} , \textbf{ye}m\v{s}\textbf{iiw}}

\conjugaison{\textcrh bèè}
    {\textcrh b\textbf{iit} , \textbf{na}\textcrh b\textbf{uu}}
    {\textcrh b\textbf{iit} , \textbf{ta}\textcrh b\textbf{uu}}
    {\textcrh bèè, \textbf{ya}\textcrh b\textbf{uu}}
    {\textcrh bèè\textbf{t} , \textbf{ta}\textcrh b\textbf{uu}}
    {\textcrh b\textbf{iinaa} , \textbf{na}\textcrh b\textbf{èèw}}
    {\textcrh b\textbf{iituu} , \textbf{ta}\textcrh b\textbf{èèw}}
    {\textcrh bèè\textbf{w} , \textbf{ya}\textcrh b\textbf{èèw}}
    
\conjugaison{nsèè}
    {ns\textbf{iit} , \textbf{ne}nsèè}
    {ns\textbf{iit} , \textbf{te}nsèè}
    {nsèè , \textbf{ye}nsèè}
    {nsèè\textbf{t} , \textbf{te}nsèè}
    {ns\textbf{iinaa} , \textbf{ne}nsèè\textbf{w}}
    {ns\textbf{iituu} , \textbf{te}nsèè\textbf{w}}
    {nsèè\textbf{w} , \textbf{ye}nsèè\textbf{w}}

\conjugaison{qraa}
    {qr\textbf{iit} , \textbf{na}qraa}
    {qr\textbf{iit} , \textbf{ta}qraa}
    {qraa , \textbf{ya}qraa}
    {qraa\textbf{t} , \textbf{ta}qraa}
    {qr\textbf{iinaa} , \textbf{na}qraa\textbf{w}}
    {qr\textbf{iituu} , \textbf{ta}qraa\textbf{w}}
    {qraa\textbf{w} , \textbf{ya}qraa\textbf{w}}

\conjugaison{klèè}
    {kl\textbf{iit} , \textbf{néé}kel}
    {kl\textbf{iit} , \textbf{téé}kel}
    {klèè , \textbf{yéé}kel}
    {klèè\textbf{t} , \textbf{téé}kel}
    {kl\textbf{iinaa} , \textbf{néé}kl\textbf{uu}}
    {kl\textbf{iituu} , \textbf{téé}kl\textbf{uu}}
    {klèè\textbf{w} , \textbf{yéé}kl\textbf{uu}}

Vous pouvez remarquer dans les tableaux du dessus que la conjugaison au passé de tous ces verbes est \textbf{identique} : seule diffère la conjugaison au présent. Pour les quatre premiers groupes (\textbf{m\v{s}èè}, \textbf{\textcrh bèè}, \textbf{nsèè} et \textbf{qraa}) la conjugaison au présent est relativement similaire dans le principe (il y a uniquement une subtilité sur les voyelles employées.). Le groupe se conjuguant comme \textbf{klèè} ressemble au présent à un verbe \textbf{assimilé}\footnote{Il n'y pas beaucoup de verbes dans ce groupe de toute façon, mais deux verbes assez importants s'y trouvent : \textbf{klèè} (\textit{manger}) et \textbf{x\dh èè} (\textit{prendre})}.

En ce qui concerne les dérivés verbaux, on retrouvera par moments les formes historiques de ces verbes, mais les principes de dérivation restent les mêmes. Notez que la défectuosité de la dernière consonne de la base n'a pas de grand impact sur les formes dérivées.

Je marque d'une étoile l'ensemble des formes qui ne sont pas utilisées de nos jours, ou qui ne sont pas attestées.

\begin{center}
    \begin{tabular}{||c | c | c | c | c | c ||}
     \hline
     \textbf{Base} & \textbf{m\v{s}èè} & \textbf{\textcrh bèè} & \textbf{nsèè} & \textbf{qraa} & \textbf{klèè} \\
     \hline\hline
     \textbf{Causatif} & ma\v{s}\v{s}aa & *\textcrh abba & nassaa & qarraa & wakkel \\
    \hline
    \textbf{Passif} & *tms\v{s}èè & *t\textcrh bèè & tensèè & teqraa & teklèè \\
   \hline
   \textbf{Réflexif} & tma\v{s}\v{s}aa & - & *tnassaa & tqarra & twakkel  \\
  \hline
  \textbf{Réciproque} & tmèè\v{s}a & - & *tnèèsa & *tqaara & *twèèkel  \\
 \hline
    \end{tabular}
\end{center}

\section*{Dialogue}
\section*{Vocabulaire}