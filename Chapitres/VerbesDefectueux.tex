\chapter{Les verbes défectueux}\label{VerbesDefectueux}
\chapterletter{N}ous avons vu lors des chapitres \ref{ConjSS} et \ref{VerbesGéminés} que l'arabe et le tunisien identifient des verbes dits \textbf{défectueux}, dont la conjugaison est légèrement différentes des verbes \textbf{sains}. Dans ce chapitre, nous allons nous attarder sur ces derniers groupes de verbes, qui comprennent de nombreux verbes importants du quotidien, et plusieurs verbes importés.

\section{Un peu d'histoire}
Nous l'avons déjà vu dans un chapitre précédent (chapitre \ref{VerbesGéminés}) : en \textbf{arabe}, les verbes se décomposent en plusieurs groupes en fonction des consonnes qui les composent.

Pour rappel, en grammaire \textbf{arabe}, on désigne comme défectueux l'ensemble des verbes qui contiennet au moins une lettre \textit{défectueuse}, c'est-à-dire un \textbf{/w/}, un \textbf{/y/} et/ou une \textbf{voyelle longue} (absence d'une des trois consonnes de la base). Ces trois lettres défectueses viennent modifier la conjugaison des verbes dans lesquels elles apparaissent\footnote{Le fait qu'une consonne manque ou que \textbf{/w/} et \textbf{/y/} soient des semi-voyelles y est bien sûr pour quelque chose.}, si bien que cela a valu la catégorisation de ces verbes dans un groupe bien distinct.

Ainsi, en \textbf{arabe}, on vient distinguer ces verbes-là en d'autres sous-groupes (en fonction de la conjugaison associée): 
\begin{itemize}
    \item Les verbes \textbf{assimilés}\footnote{Comme souvent, les dénominations françaises que j'ai choisies ne sont pas nécessairement les dénominations officielles, pour peu qu'il en existe...} \RL{الفعل المثال} qui ont une consonne défectueuse sur l'emplacement de la première consonne (par exemple \textbf{/waqafa/} \RL{وقف} \textit{s'arrêter});
    \item Les verbes \textbf{concaves} \RL{الفعل الأجوف} qui ont une consonne défectueuse sur l'emplacement de la deuxième consonne (par exemple \textbf{/qaala/} \RL{قال} \textit{dire});
    \item Les verbes \textbf{incomplets} \RL{الفعل الناقص} qui ont une consonne défectueuse sur l'emplacement de la dernière consonne (par exemple \textbf{/nasiya/} \RL{نسي} \textit{oublier})\footnote{Techniquement, il y a des groupes qui ne comprennent que des verbes avec deux consonnes défectueuses. Ces verbes sont tellement peu nombreux qu'il n'y a pas vraiment d'intrêt à les catégoriser (selon moi).}.
\end{itemize}

Bien évidemment, toutes les consonnes défectueuses n'apparaissent pas à la même fréquence aux mêmes emplacements. Ainsi, on trouvera souvent des voyelles longues \textbf{en milieu ou fin de mot}, alors que \textbf{/w/} et \textbf{/y/} apparaissent souvent au début du verbe.

En \textbf{tunisien}, le principe reste le même pour ces trois consonnes, cependant il me semble intéressant de mentionner ce qui suit: la plupart des verbes \textbf{arabe} comportant un \textbf{/'/} sont devenus \textit{en partie} défectueux puisque le coup de glotte disparaît au début et à la fin de tous les mots. On notera par exemple\linebreak \textbf{/'axa\dh a/} \RL{أخذ} (\textit{prendre}) qui est devenu \textbf{/x\dh èè/}. Dans la suite, j'analyserai donc ces verbes comme tels, bien qu'ils ne soient pas défectueux en \textbf{arabe}.

\section{Les verbes assimilés}
Les verbes \textbf{assimilés} sont les verbes dont la première consonne de leur base est une consonne défectueuse.

On dénombre alors deux sous-groupes de verbes défectueux : 
\begin{itemize}
    \item Les verbes en \textbf{/w/}, comme \textbf{wqef} (\textit{s'arrêter}) ;
    \item Les verbes en \textbf{/y/}, comme \textbf{ybes} (\textit{se durcir}) ;
\end{itemize}

Ces verbes sont dits \textbf{assimilés} car la première consonne \textit{tombe} souvent à certaines formes au profit d'une voyelle proche : \textbf{/w/} se transforme en \textbf{/u/}, et \textbf{/y/} se transforme en \textbf{/i/}.

En règle générale, les conjugaisons et les dérivés de ces verbe verront leur consonne défectueuse remplacée dès qu'elle sera encadrée de deux consonnes.

Nous pouvons voir ça avec la conjugaison de ces verbes, régulière au passé, mais légèrement différente au présent : 

\conjugaison{wqef}{
    wqef\textbf{t}, \textbf{nuu}qef
}{
    wqef\textbf{t}, \textbf{tuu}qef
}{
    wqef, \textbf{yuu}qef
}{
    w\textbf{e}qfe\textbf{t}, \textbf{tuu}qef
}{
    wqef\textbf{naa}, \textbf{nuu}qf\textbf{uu}
}{
    wqef\textbf{tuu}, \textbf{tuu}qf\textbf{uu}
}{
    w\textbf{e}qf\textbf{uu}, \textbf{yuu}qf\textbf{uu}
}

\section*{Dialogue}
\section*{Vocabulaire}