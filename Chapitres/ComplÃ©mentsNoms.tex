\chapter{Les compléments de nom}
\chapterletter{D}ans ce chapitre, je vous propose de nous intéresser aux compléments de noms, ce qui vous permettra de désigner avec plus de précision les objets qui vous entourent. Avec les possessifs et les démonstratifs, vous aurez à la fin de ce chapitre beaucoup d'outils à votre disposition pour commencer à créer des groupes nominaux complexes.

\section{Un peu d'histoire}
Il faut savoir que l'arabe classique est une langue à cas, fort heureusement beaucoup plus restreints que le latin ou le russe, si bien qu'on en compte que \textbf{trois}:

\begin{itemize}
    \item Le \textbf{nominatif} : en arabe, c'est le cas grammatical du sujet, et le cas grammatical par défaut;
    \item L'\textbf{accusatif} : ce cas regroupe notamment les compléments d'objets directs (COD), mais aussi tout un ensemble d'autres compléments;
    \item Le \textbf{génitif} : ce troisième cas regroupe à la fois les \textbf{compléments de noms} et tous les groupes nominaux précédés d'une préposition.
\end{itemize}

Remarquez qu'il y a \textbf{autant} de cas de grammaticaux que de voyelles distinguées par l'arabe. Cela n'est pas une coïncidence, mais bien le fait que les cas grammaticaux sont distingués par la voyelle terminale du mot !

Comme nous l'avons vu tout au long de ce cours, le tunisien a depuis longtemps abandonné la prononciation des voyelles terminales des mots, et donc a cessé de faire la distinction entre ces trois cas. Une légère subtilité a cependant perduré : la prononciation du son \textbf{/t/} final pour les mots féminins singuliers, lorsqu'ils sont suivi d'un complément de nom. La structure du complément de nom est quant à elle très similaire à celle du français : \textbf{nom à qualifier + complément de nom}, comme dans \textbf{la niche du chien}.

Illustrons ceci par un exemple : 

\begin{center}
\begin{tabular}{||c | c | c | c||}
 \hline
  \textbf{Arabe} & \textbf{Prononc. théorique} & \textbf{Prononc. réelle} & \textbf{Traduction}\\
 \hline\hline
  \RL{مدرسة الرجل} & Madrasati elrajol  & Madrasat-elrajol & L'école de l'homme\\
  \hline
\end{tabular}    
\end{center}

Comme vous le voyez, la voyelle finale du mot féminin à tendance à disparaître au profit de l'article défini du mot suivant. Le qualifié et le complément du nom auront donc tendance à se prononcer d'une seule traite, et c'est ce phénomène qui fait que le son \textbf{/t/} est encore prononcé dans ce contexte particulier.

\section{En tunisien}
En \textbf{tunisien}, la structure reste identique à celle de l'arabe classique. Ainsi, tous les compléments de noms en tunisien se formeront de cette manière : 
\begin{center}
    \textbf{groupe nominal (forme indéfinie) + complément de nom (forme définie)}
\end{center}

A la manière d'un possessif, le complément du nom joue le rôle de \textit{définir} le nom à qualifier. De ce fait, le groupe nominal sera \textbf{nécessairement} sous sa forme indéfinie, tandis que le complément sera sous une forme définie. 

Deux petites subtilités cependant. La \textbf{première} évoquée au paragraphe précédent : pour les noms féminins singuliers, on ajoute à la fin du groupe nominal le \textbf{/t/} correspondant à la marque du féminin en arabe classique.

La \textbf{deuxième} concerne les mots masculins se terminant par une voyelle, qui n'incorporent donc pas ce \textbf{/t/} intermédiaire. Systématiquement en tunisien, une succession de deux voyelles est impossible, et un \textbf{/'/} doit s'intercaler entre les deux sons. Cependant, les locuteurs ayant tendance à vers des prononciations plus simples, vous entendrez souvent une des deux voyelles \textit{se faire manger}.

Regardons ensemble quelques exemples : 

\begin{center}
\begin{tabular}{||c | c | c||}
 \hline
  \textbf{Tunisien} & \textbf{Traduction} & \textbf{Littéralement} \\
 \hline\hline
  Sanduuq lemraa  & La boîte de la dame & \textit{Boîte la-dame}\\
  \hline
  Ordinater elmakteb  & L'ordinateur de l'école & \textit{Ordinateur l'école}\\
  \hline
  Qlam Fat\textcrh i  & Le crayon de Fathi & \textit{Crayon Fathi} \\
  \hline
  Karehbet elbulisiyya  & La voiture de police & \textit{Voiture les-policiers}\\
  \hline
  Kraaheb elbulisiyya  & Les voitures de police & \textit{Voitures les-policiers}\\
  \hline
  Keswet el\c{a}ers  & Le costume du mariage & \textit{Costume le-mariage}\\
  \hline
\end{tabular}    
\end{center}


Quelques remarques : 
\begin{itemize}
    \item Les \textbf{noms propres} sont sous une forme définie, le complément de nom peut donc n'être composé que du prénom de quelqu'un (on sait forcément de qui il s'agit).
    \item On retrouve aussi ce qui peut s'apparenter à des noms composés, comme \textbf{biit la\textcrh méém} (salle de bain), dans la mesure où le complément de nom \textbf{\textcrh méém} est même devenu totalement désuet en dehors de contexte.
\end{itemize}

\section*{Dialogue}
\section*{Vocabulaire}