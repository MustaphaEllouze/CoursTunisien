\chapter{Les verbes dérivés et leur conjugaison}
\chapterletter{N}ous avons vu dans un précédent chapitre la conjugaison des verbes \textbf{simples} (cf. chapitre \ref{ConjSS}). Ces verbes avaient tous la particularité d'être formés de trois consonnes (pour les verbes sains), et d'une voyelle unique. Il existe en opposition à ces verbes-là les verbes \textbf{dérivés} qui sont comme leur nom l'indique produits à partir des verbes simples. Je vous propose de nous attarder quelques instants dessus.

\section{Un peu d'histoire}
\subsection{Les bases triconsonantiques et la dérivation}
Je vous le disais déjà au paragraphe \ref{ConjSS1}, les langues sémitiques s'articulent toutes autour de bases triconsonantiques (des \textbf{triples de consonnes}). Ces bases sont en charge de contenir un \textit{sens principal} à partir duquel on pourra dériver tous les mots appartenant à un champ lexical particulier. 

Par exemple, prenons la racine sémitique \textbf{K-T-B} (telle quelle en \textbf{arabe} \RL{كتب}, et \textbf{K-T-V} en \textbf{hébreu} \<ktb>\footnote{Cela se prononce bien avec un \textbf{/v/}, mais cela s'écrit bien avec un \<b> dont l'équivalent en arabe (\RL{ب}) produit le son \textbf{/b/}.}). Cette racine porte en elle le champ lexical de l'\textbf{écriture} : \textbf{/kataba/} (\textsc{ar}\footnote{Code ISO pour l'arabe}) et /\textbf{ktav}/ (\textsc{iw}\footnote{Code ISO pour l'hébreu}) veulent tous les deux dire \textit{il a écrit} ; tandis que \textbf{/kitéébon/} (\textsc{ar}) veut dire \textit{livre}, \textbf{/maktabon/} (\textsc{ar}) veut dire \textit{école}, et \textbf{/kétuva/} (\textsc{iw}) veut dire \textit{contrat de mariage}.

Comme vous le voyez, il est possible d'agrémenter la base triconsonantique de \textbf{voyelles} et \textbf{consonnes} supplémentaires afin d'en changer le sens. Ces changements de sens obéissent à des règles, et se manifestent par des \textbf{schèmes}, qui sont en réalité des schémas à appliquer, qui sont les mêmes pour l'ensemble des bases. 

Ainsi, en arabe, le schème correspondant au \textbf{sujet} de l'action est \textbf{/féé\c{a}ilon/} \RL{فاعل}\footnote{La base \textbf{F-\c{A}-L}, qui veut dire \textit{faire}, sert d'exemple par défaut à l'application des schèmes en arabe.}, alors que le schème correspondant au \textbf{patient} de l'action (celui qui la subit) est \textbf{/maf\c{a}uulon/} \RL{مفعول}. 

En pratique ce la donne : 

\begin{itemize}
    \item \textbf{/kéétibon/} est un \textit{écrivain}, \textbf{/léé\c{a}ibon/}\footnote{Vous connaissez déjà cette base. Rappelez-vous du verbe \textbf{l\c{a}ab} en tunisien qui veut dire \textit{jouer}.} est un \textit{joueur}, \textbf{/qaari'on/}\footnote{\textbf{Q-R-'} se rapporte à tout ce qui a trait à la \textit{lecture}.} est un \textit{lecteur}.
    \item \textbf{/ma\v{s}ruubéét/}\footnote{\textbf{\v{S}-R-B} veut dire \textit{boire}.} sont des \textit{boissons},\textbf{/ma'kuuléét/}\footnote{\textbf{'-K-L} veut dire \textit{manger}.} veut dire \textit{nourriture}, \textbf{/maktuubon/} veut dire \textit{destin}.
\end{itemize}

Retenez donc qu'il y a en arabe deux notions (qu'on retrouvera en tunisien) : 

\begin{itemize}
    \item La \textbf{base triconsonantique} qui code pour le champ lexical : c'est l'équivalent de la \textbf{racine} en français ;
    \item Le \textbf{schème} qui vient préciser le sens de la \textbf{base} : c'est l'équivalent des \textbf{préfixes} et \textbf{suffixes} en français.
\end{itemize}

\subsection{La dérivation des verbes}
La puissance de l'utilisation du duo \textbf{base/schème} se fait notamment sentir dès lors qu'il s'agit d'extraire de nouveaux verbes à partir de verbes qu'on connaît déjà.

L'arabe redouble de créativité quand il s'agit de trouver des schèmes, et en produit plusieurs dizaines pour les verbes seulement. Chacun apporte sa nuance particulière, et il est donc possible de générer très simplement un sens très précis, pour peu qu'on connaisse suffisamment bien les schèmes à notre disposition\footnote{Je soulève quand même un point noir : certains schèmes appliqués à des bases bien précises n'ont pas de sens bien définis, ou sont des mots un peu désuets. Ainsi, on ne se permettra pas n'importe quel schème avec n'importe quelle base, de la même manière que les mots \textit{rétro-marche} ou \textit{bi-stranguel} ne veulent rien dire en français, contrairement à \textit{rétro-conception} et \textit{bi-hebdomadaire}.}. 

Dans cette sous-partie, je ne souhaite pas détailler l'ensemble des schèmes qui existent en arabe, ce serait trop long. Mais je souhaite quand même évoquer les plus importants et les plus usés, ce qui via des cas d'usage pratique vous aidera à comprendre les formes existantes qui perdurent encore aujourd'hui en tunisien.

Je vous présente dans le tableau suivant des exemples d'application de schèmes sur diverses bases.

\begin{center}
\begin{tabular}{||c | c | c | c | c||}
 \hline
  \textbf{Base} & \textbf{Signif.} & \textbf{Dérivé} & \textbf{Signif.} & \textbf{Signif. schème}\\
 \hline\hline
  \RL{كَتَبَ} & Ecrire & \RL{كُتِبَ} & Être écrit & Forme passive\\
  \textbf{/kataba/} & & \textbf{/kutiba/} & & \\
  \hline
  \RL{أَكَلَ} & Manger & \RL{أَكَّلَ} & Faire manger & Causatif\\
  \textbf{/'èkèlè/} & & \textbf{/'èkkèlè/} & & \\
  \hline
  \RL{زَوَجَ} & Unir & \RL{تَزَوَّجَ} & Se marier & Réflexif\\
  \textbf{/zawaja/} &  & \textbf{/tazawwaja/} & & \\
  \hline
  \RL{عَوَنَ} & Secourir & \RL{تَعَاوَنَ} & S'entraider & Interaction\\
  \textbf{/\c{a}awana/} &  & \textbf{/ta\c{a}aawana/} & & \\
  \hline
  \RL{كَسَبَ} & Posséder & \RL{اكْتَسَبَ} & Acquérir & Action pour soi\\
  \textbf{/kasaba/} & & \textbf{/'ektèsèbè/} & &\\
  \hline
  \RL{خَرَجَ} & Sortir & \RL{اسْتَخْرَجَ} & Extraire & Action minutieuse\\
  \textbf{/xaraja/} &  & \textbf{/'estaxraja/} &  & \\
  \hline
\end{tabular}    
\end{center}

Les exemples ci-dessus ne sont que cela : des exemples. Les schèmes sont en réalité plus compliqués que cela, dans la mesure où certains d'entre eux portent des sens qui sont plus souvent soumis à l'interprétation. Cela se comprend relativement bien : le sens des mots a tendance à évoluer avec le temps, et cette dérive ne prend pas nécessairement le sens du schème en compte. Le sens global d'un schème dérive alors graduellement.

Ne nous reste alors qu'un seul élément à aborder : la \textbf{conjugaison}. En \textbf{arabe}, il faut globalement retenir que chaque \textit{schème verbal} a sa conjugaison qui lui est propre. On pourrait presque parler de groupes de verbes\footnote{C'est ce que j'ai décidé de faire dans le chapitre \ref{ConjSS}.}. Cependant dans l'ensemble, il apparaît que la conjugaison en arabe reste relativement régulière et déductible\footnote{Je vous accorde que cela est subjectif.} : le moyen le plus sécuritaire reste d'apprendre toutes les conjugaisons par c\oe ur, mais il reste tout à fait possible de déduire de façon subconsciente des règles générales via la pratique de la langue. 

Fort heureusement, en \textbf{tunisien}, les choses sont plus simples. Les schèmes verbaux sont encore présents, mais leur nombre s'est considérable réduit. Plusieurs schèmes ressortent clairement du lot, alors que d'autres schèmes présents en arabe se sont fait supplantés par l'usage de \textbf{marqueurs préverbaux} et les \textbf{verbes modaux}\footnote{Il s'agit d'une nouveauté, déjà partiellement présente en arabe, mais qui a été largement développée par le tunisien. J'y consacre deux chapitres plus loin, aux chapitres \ref{VerbMod} et \ref{MarqPVer}.}.

D'une façon générale, je vous conseille de ne pas essayer de sur-analyser l'ensemble des verbes que vous pourrez rencontrer. Découper les verbes en \textit{base + schème} se révélera très utile par moment pour comprendre rapidement le sens d'un mot que vous ne connaissez pas, mais ce ne sera pas une technique infaillible : l'usage faisant la grammaire (et non l'inverse), il vous arrivera de tomber sur des \textbf{schèmes inusités} ou des sur des \textbf{bases verbales qui n'ont plus de sens seule}.

\section{Dérivation des verbes en tunisien}
Rentrons dans le coeur du sujet : la \textbf{dérivation verbale} en tunisien. 

Avant toute chose j'aimerais que vous reteniez une information : \textit{malgré tous vos efforts, il vous serra inutile et quasi-impossible d'identifier tous les schèmes encore en usage en tunisien, et de déterminer leur sens}. Il vous sera par contre beaucoup plus utile de savoir faire ces deux choses que je vais vous présenter. 

La première compétence à maîtriser est la détermination dans chaque verbe  de ce qui relève : 
\begin{itemize}
    \item De la \textbf{base} : c'est la structure consonantique qui porte l'essentiel du sens du verbe (son champ lexical) ; 
    \item Du \textbf{schème} : c'est le moule à partir duquel le verbe est formé, et c'est ce qui termine de définir le sens du verbe ; 
    \item De la \textbf{conjugaison} : c'est ce qui donne l'information sur le temps et la personne à laquelle le verbe se réfère, et qui concrétise l'action dans le temps et le contexte.
\end{itemize}

La deuxième compétence est celle d'apprendre par c\oe ur les schèmes les plus importants du tunisien. Connaître ces schèmes, c'est savoir \textbf{exprimer des nuances efficacement} à partir d'une base verbale que vous connaissez déjà. 

A travers ce chapitre, je vous propose donc de développer ces deux compétences en vous présentant les \textbf{schèmes verbaux} du tunisien. Faites l'effort à chaque exemple de déterminer par vous-même les consonnes composant la base. 

\subsection{Schème de la voix causative}
Le schème de la voix \textbf{causative} est l'un des deux schèmes les plus utilisés en tunisien.

Il permet de transformer un verbe afin d'employer la \textbf{voix causative} dans un phrase, c'est-à-dire que le fait que \textbf{le sujet fasse exécuter l'action au complément, ou change l'état du complément}. En \textbf{français}, la voix causative est formée en ajoutant le verbe \textit{faire} avant un infinitif, comme dans \textit{je l'ai fait sortir} ou \textit{je lui ferai signer}.

Il se forme comme suit à partir de la base : 
\begin{center}
    \Large{\textbf{1 2 3} $\rightarrow$ \textbf{1 a 2 2 e 3}}
\end{center}
où les chiffres désignent chacun une consonne de la base.

\textbf{\textsc{Note :}} En tunisien, le schème de la voix causative produit des verbes \textbf{transitifs}, c'est-à-dire que ces verbes imposent la présence d'un complément d'objet direct ou indirect.

Voici quelques exemples : 

\begin{center}
\begin{tabular}{||c | c | c | c ||}
 \hline
  \textbf{Base} & \textbf{Trad.} & \textbf{Causatif} & \textbf{Trad.} \\
 \hline\hline
  l\c{a}ab & \textit{jouer} & la\c{a}\c{a}eb & \textit{faire jouer}\\
  \hline
  xraj & \textit{sortir} & xarrej & \textit{faire sortir}\\
  \hline
  fhem & \textit{comprendre} & fahhem & \textit{expliquer}\\
  \hline
  wqef & \textit{s'arrêter} & waqqef & \textit{faire s'arrêter}\\
  \hline
  fsed & \textit{devenir corrompu} & fassed & \textit{corrompre}\\
  \hline
  mro\c{\dh} & \textit{devenir malade} & marre\c{\dh} & \textit{contaminer}\\
  \hline
\end{tabular}    
\end{center}

D'une façon plus générale, vous pourrez également retrouver des verbes sous leur forme \textbf{causative}, sans pour autant que la forme \textbf{simple} ne soit usitée.

\begin{center}
\begin{tabular}{||c | c ||}
 \hline
  \textbf{Causatif} & \textbf{Trad.} \\
 \hline\hline
  sakker & \textit{fermer} \\
  \hline
  sawwed & \textit{noircir} \\
  \hline
\end{tabular}    
\end{center}

Et finalement quelques exemples d'utilisation : 

\begin{center}
\begin{tabular}{||c | c ||}
 \hline
 \textbf{Tunisien} & \textbf{Français} \\
 \hline\hline
 La\c{a}\c{a}abt ess\v{r}aar. & \textit{J'ai fait jouer les enfants.} \\ 
 \hline
 Xarrej elkalb. & \textit{Il a fait sortir le chien.} \\ 
 \hline
 Fahhmet eddars lelbnayya. & \textit{Elle a expliqué le cours à la fille.} \\ 
 \hline
 Nwaqqef elmu\c{t}uur ? & \textit{Est-ce que j'arrête le moteur ?} \\ 
 \hline
 Hé\dh uukom errjèèl fassduu ejjaw. & \textit{Ces hommes ont pourri l'ambiance.} \\ 
 \hline
 Huwwa marre\c{\dh} oxtii. & \textit{Il a contaminé ma s\oe ur.} \\
 \hline
 Sakkert elbééb. & \textit{J'ai fermé la porte.} \\
 \hline
 Sawwed yidduu belf\textcrh am. & \textit{Il a noirci sa main avec le charbon.} \\
 \hline
\end{tabular}
\end{center}

\subsection{Schème de la voix passive}
Le schème de \textbf{la voix passive} est le deuxième schème le plus employé en tunisien. 

Il transforme le verbe afin d'employer la \textbf{voix passive} dans une phrase, c'est-à-dire le fait que \textbf{le sujet subit l'action}. En \textbf{français}, la voix passive est formée en juxtaposant le verbe être et le participe passé du verbe, comme dans \textit{la pomme a été mangée}.

En \textbf{tunisien}, la voix passive se construit \textbf{en ajoutant /t/ \underline{avant} le verbe.}. Si le verbe commence par \textbf{deux consonnes}, alors on ajoute un \textbf{/e/} supplémentaire pour aider la prononciation.  

\begin{center}
    \Large{\textbf{verbe} $\rightarrow$ \textbf{t (+ e) + verbe}}
\end{center}

\textbf{\textsc{Note :}} En tunisien, le schème de la voix passive produit des verbes \textbf{intransitifs}, c'est-à-dire que ces verbes n'acceptent pas de complément d'objet direct ou indirect.

Voici quelques exemples : 

\begin{center}
\begin{tabular}{||c | c | c | c ||}
 \hline
  \textbf{Base} & \textbf{Trad.} & \textbf{Voix passive} & \textbf{Trad.} \\
 \hline\hline
  l\c{a}ab & \textit{jouer} & tel\c{a}ab & \textit{se jouer}\\
  \hline
  fhem & \textit{comprendre} & tefhem & \textit{se faire comprendre}\\
  \hline
  bla\c{a} & \textit{avaler} & tebla\c{a} & \textit{être avalé}\\
  \hline
  kteb & \textit{écrire} & tekteb & \textit{être écrit}\\
  \hline
\end{tabular}    
\end{center}

Et également des exemples d'utilisation : 

\begin{center}
\begin{tabular}{||c | c ||}
 \hline
 \textbf{Tunisien} & \textbf{Français} \\
 \hline\hline
 E\v{ss}kobba tetel\c{a}ab belkwaaret. & \textit{La chkobba se joue avec des cartes.} \\ 
 \hline
 Elxo\c{t}\c{t}a tfehmet. & \textit{Le plan a été compris.} \\ 
 \hline
 El\textcrh arbuu\v{s}a tbel\c{a}et. & \textit{La pilule a été avalée.} \\ 
 \hline
 Elkontraatu tekteb. & \textit{Le contrat a été écrit.} \\ 
 \hline
\end{tabular}
\end{center}

\subsection{Schème de la voix réfléchie}
Le schème de la \textbf{voix réfléchie} est également un schème assez courant en tunisien. 

Il transforme le verbe afin d'employer la \textbf{voix réfléchie} dans une phrase, c'est-à-dire le fait que \textbf{le sujet soit l'objet de sa propre action}. En \textbf{français}, la voix réfléchie s'exprime par l'emploi de verbes pronominaux comme dans \textit{le garçon s'est lavé}. En \textbf{anglais}, on utilisera les constructions dans le style de \textit{he taught himself}.

En \textbf{tunisien}, la voix réfléchie se forme à partir de la base :

\begin{center}
    \Large{\textbf{1 2 3} $\rightarrow$ \textbf{t 1 a 2 2 e 3}}
\end{center}

où les chiffres désignent chacun une consonne de la base.

\textbf{\textsc{Note:}} Il existe d'autres manières d'exprimer la voix réfléchie. Cela sera abordé dans un chapitre ultérieur (cf. chapitre \ref{PronCompl}).

Voici quelques exemples : 

\begin{center}
\begin{tabular}{||c | c | c | c ||}
 \hline
  \textbf{Base} & \textbf{Trad.} & \textbf{Voix réfléchie} & \textbf{Trad.} \\
 \hline\hline
  \c{a}ro\c{\dh} & \textit{croiser (qqch/qqn)} & t\c{a}arre\c{\dh} & \textit{se confronter (à qqch) / prévoir}\\
  \hline
  wqa\c{a} & \textit{se dérouler} & twaqqe\c{a} & \textit{s'imaginer (qqch)}\\
  \hline
  \c{a}lem & \textit{informer} & t\c{a}allem & \textit{apprendre}\\
  \hline
  \dh kar & \textit{mentionner} & t\dh akker & \textit{se rappeler}\\
  \hline
\end{tabular}    
\end{center}

Puis d'autres exemples dont la base n'est plus employée : 

\begin{center}
\begin{tabular}{||c | c ||}
 \hline
 \textbf{Voix réfléchie} & \textbf{Trad.} \\
 \hline\hline
 tsakker & \textit{se fermer} \\
 \hline
 tmassex & \textit{se salir} \\
 \hline
 twajjeh & \textit{se diriger} \\
 \hline
 t'axxer & \textit{être en retard} \\
 \hline
 tsalleq & \textit{escalader} \\
 \hline
 tkallem & \textit{parler} \\
 \hline
 t\textcrh adde\th & \textit{discuter} \\
 \hline
\end{tabular}
\end{center}

Finalement des exemples d'utilisation : 

\begin{center}
\begin{tabular}{||c | c ||}
 \hline
 \textbf{Tunisien} & \textbf{Français} \\
 \hline\hline
 T\c{a}arre\dh\ lelmo\v{s}kla kbiira. & \textit{Il a été confronté à un gros problème.} \\ 
 \hline
 Twaqqa\c{a}naa muut erra'iis. & \textit{Nous avons prévu la mort du président.} \\ 
 \hline
 Yet\c{a}allmuu elgitaar. & \textit{Ils apprennent la guitare.} \\ 
 \hline
 Ta\dh kkret elli nséét portabelhaa. & \textit{Elle s'est rappelée qu'elle a oublié son portable.} \\ 
 \hline
 Elbééb tsakker. & \textit{La porte s'est fermée.} \\ 
 \hline
 Eddba\v{s} tmassex. & \textit{Les vêtements se sont salis.} \\ 
 \hline
 Twajjeht lelxruuj. & \textit{Tu t'es dirigé vers la sortie.} \\ 
 \hline
 \c{A}lèè\v{s} t'axxertuu ? & \textit{Pourquoi étiez vous en retard ?} \\ 
 \hline
 Yetsalleq el\textcrh ii\c{t}. & \textit{Il a escaladé le mur.} \\ 
 \hline
 Netkallem m\c{a}aak. & \textit{Je parle avec toi.} \\ 
 \hline
 Tet\textcrh adde\th\ m\c{a}aaya. & \textit{Tu discutes avec toi.} \\ 
 \hline
\end{tabular}
\end{center}

\subsection{Schème de la voix causative-passive}
Le schème de la \textbf{voix causative-passive} est le dernier des grands schèmes à retenir pour la maîtrise du tunisien. 

Il transforme le verbe afin d'employer la \textbf{voix causative-passive} dans une phrase, c'est-à-dire le fait que le \textbf{le sujet soit forcé par quelqu'un d'autre à faire une action}. On on peut retrouver cette voix en \textbf{français} dans des constructions comme \textit{On m'a fait mangé quelque chose que je n'aimais pas}. Le \textbf{japonais} possède d'ailleurs une forme verbale causative-passive : \textbf{/taberu/}veut dire \textit{manger}, \textbf{/tabesaserareru/} veut dire \textit{être forcé à manger}.

En \textbf{tunisien}, la voix causative-passive se forme de la même façon que la voix réflexive\footnote{Ce qui est en soit très logique pour deux raisons. Morphologiquement, la voix réflexive se construit en préfixant un \textbf{/t/} à la forme causative. Sémantiquement, la voix réflexive peut s'interpréter comme le fait de se forcer soi-même à faire quelque chose.} : 

\begin{center}
    \Large{\textbf{1 2 3} $\rightarrow$ \textbf{t 1 a 2 2 e 3}}
\end{center}

où les chiffres désignent chacun une consonne de la base.

\textbf{\textsc{Note :}} Contrairement au \textbf{français}, le schème de la voix causative-passive en tunisien produit des verbes intransitifs, c'est-à-dire qu'ils n'admettent pas de complément. Ainsi, on se saura pas \textit{qui} a forcé l'accomplissement de l'action.


Voici quelques exemples : 

\begin{center}
\begin{tabular}{||c | c | c | c ||}
 \hline
  \textbf{Causatif} & \textbf{Trad.} & \textbf{Voix caus-pass} & \textbf{Trad.} \\
 \hline\hline
  xarraj & \textit{faire sortir} & txarraj & \textit{se faire sortir}\\
  \hline
  sajjal & \textit{enregistrer} & tsajjal & \textit{se faire enregistrer}\\
  \hline
  la\c{s}\c{s}aq & \textit{coller} & tla\c{s}\c{s}aq & \textit{se faire coller}\\
  \hline
  wazza\c{a} & \textit{distribuer} & twazza\c{a} & \textit{se faire distribuer}\\
  \hline
\end{tabular}    
\end{center}

Et des exemples d'utilisation : 

\begin{center}
\begin{tabular}{||c | c ||}
 \hline
 \textbf{Tunisien} & \textbf{Français} \\
 \hline\hline
 E\dh\dh ebbééna txarrjet melkujiina. & \textit{\makecell{La mouche a été sortie de \\ la cuisine.}} \\ 
 \hline
 El\v{r}néyéét tsajjluu filkasèèt. & \textit{\makecell{Les chansons ont été enregistrées \\ dans la cassette.}} \\ 
 \hline
 Elpostèèr tlassaq \c{a}al \textcrh iit. & \textit{Le poster a été collé sur le mur.} \\ 
 \hline
 Lekwaaret twazz\c{a}uu. & \textit{On a distribué les cartes.} \\ 
 \hline
\end{tabular}
\end{center}

\subsection{Schème de la voix réciproque}
Le schème de la \textbf{voix réciproque} est un schème relativement moins utilisé que ceux qu'on a vu jusqu'à présent. 

Il transforme le verbe afin d'employer la \textbf{voix réciproque} dans une phrase, c'est-à-dire le fait que \textbf{le sujet entreprenne une action avec le complément}\footnote{En tunisien, on sous-entend généralement avec la forme réciproque que l'action dure dans le temps.}. En \textbf{français}, cette voix est souvent signifiée par la présence de \textit{avec}, comme dans \textbf{j'ai parlé avec lui}. En \textbf{anglais}, on utilisera plutôt les constructions de la forme \textit{they spoke with each other}.

En \textbf{tunisien}, la voix réfléchie se forme à partir de la base :

\begin{center}
    \Large{\textbf{1 2 3} $\rightarrow$ \textbf{t 1 é é 2 e 3}}
    
    \Large{\textbf{1 2 3} $\rightarrow$ \textbf{t 1 a a 2 e 3}}
\end{center}

où les chiffres désignent chacun une consonne de la base.

\textbf{\textsc{Note:}} Les verbes exprimant la voix réciproque nécessite nécessairement l'emploi de la préposition \textbf{m\c{a}aa} (\textit{avec}) pour signifier avec qui l'action est menée.

Voici quelques exemples : 

\begin{center}
\begin{tabular}{||c | c | c | c ||}
 \hline
  \textbf{Base} & \textbf{Trad.} & \textbf{Voix réfléchie} & \textbf{Trad.} \\
 \hline\hline
  l\c{a}ab & \textit{jouer} & tléé\c{a}eb & \textit{jouer (avec qqn)}\\
  \hline
  fhem & \textit{comprendre} & tfééhem & \textit{s'entendre (avec qqn)}\\
  \hline
  \c{a}mal & \textit{faire} & t\c{a}aamel & \textit{interagir (avec qqn)}\\
  \hline
  - & \textit{-} & t\c{a}aarek & \textit{se battre (avec qqn)}\\
  \hline
  - & \textit{-} & tkéélem & \textit{parler (avec qqn)}\\
  \hline
\end{tabular}    
\end{center}

Et des exemples d'utilisation : 

\begin{center}
\begin{tabular}{||c | c ||}
 \hline
 \textbf{Tunisien} & \textbf{Français} \\
 \hline\hline
 Lulééd yetléé\c{a}buu m\c{a}aa b\c{a}\c{/dh}hom. & \textit{Les enfants jouent ensemble.} \\ 
 \hline
 Monya tfééhmet m\c{a}aah. & \textit{Monia s'est arrangée avec lui.} \\ 
 \hline
 Taw net\c{a}aamel m\c{a}aahaa. & \textit{Je vais voir avec elle.} \\ 
 \hline
 \c{A}lèè\v{s} t\c{a}aarektuu ? & \textit{Pourquoi vous êtes vous battus ?} \\ 
 \hline
 Yetkéélem m\c{a}aaya. & \textit{Il parle avec moi.} \\ 
 \hline
\end{tabular}
\end{center}

\subsection{Schème de l'aspect inchoatif}
Le schème de l'\textbf{aspect inchoatif} est un schème assez rare, qui ne se retrouve bien souvent que dans un cas assez particulier, que nous allons aborder.

Il transforme le verbe afin d'employer l'aspect \textbf{inchoatif} dans une phrase, c'est-à-dire le fait que \textbf{le sujet entre dans un état particulier}. 

En \textbf{tunisien}, l'aspect inchoatif ne se forme qu'avec les \underline{couleurs}. Il se forme à partir de la base : 

\begin{center}
    \Large{\textbf{1 2 3} $\rightarrow$ \textbf{1 2 a a 3}}
    
    \Large{\textbf{1 2 3} $\rightarrow$ \textbf{1 2 è è 3}}
\end{center}

où les chiffres désignent chacun un consonne de la base. 

\textbf{\textsc{Note :}} Les verbes à l'aspect inchoatif sont tous \textbf{intransitifs}, c'est-à-dire qu'ils n'admettent pas de complément d'objet.

Voici quelques exemples : 

\begin{center}
\begin{tabular}{||c | c | c | c ||}
 \hline
  \textbf{Nom} & \textbf{Trad.} & \textbf{Aspect inchoatif} & \textbf{Trad.} \\
 \hline\hline
  'a\textcrh mar & \textit{rouge} & \textcrh maar & \textit{rougir}\\
  \hline
  'a\c{s}far & \textit{jaune} & \c{s}faar & \textit{jaunir}\\
  \hline
  'ax\c{\dh}ar & \textit{vert} & x\c{\dh}aar & \textit{verdir}\\
  \hline
  'ax\c{\dh}ar & \textit{vert} & x\c{\dh}aar & \textit{verdir}\\
  \hline
  'azraq & \textit{bleu} & zrèèq & \textit{bleuir}\\
  \hline
  'ak\textcrh al & \textit{noir} & k\textcrh aal & \textit{noircir}\\
  \hline
  'abya\c{\dh} & \textit{blanc} & byaa\c{\dh} & \textit{blanchir}\\
  \hline
  fèèta\textcrh & \textit{clair} & ftèè\textcrh & \textit{s'éclaircir}\\
  \hline
  \v{r}aameq & \textit{sombre} & \v{r}mèèq & \textit{s'assombrir}\\
  \hline
\end{tabular}    
\end{center}

Ainsi que des exemples d'utilisation :

\begin{center}
\begin{tabular}{||c | c ||}
 \hline
 \textbf{Tunisien} & \textbf{Français} \\
 \hline\hline
 \textcrh maarnaa. & \textit{Nous avons rougi.} \\ 
 \hline
 Elfee x\dh aar. & \textit{Le feu est passé au vert.} \\ 
 \hline
 Elmèè ftèè\textcrh. & \textit{L'eau est devenue claire.} \\ 
 \hline
 Essmèè \v{r}emqet. & \textit{Le ciel s'est assombri.} \\ 
 \hline
\end{tabular}
\end{center}

\section{Conjugaison des verbes dérivés}
Maintenant que nous avons vu les schèmes principaux de la langues tunisienne, je vous propose de prendre un exemple par schème et de parcourir sa conjugaison\footnote{Gardez à l'esprit qu'il existe potentiellement des variations vocaliques mineures, relativement ponctuelles. Les conjugaisons qui sont données après représentent la conjugaison "principale", celle qu'il faut apprendre, mais des variations/exceptions peuvent exister.}.

Dans la suite de cette section, gardez à l'esprit deux choses : 
\begin{itemize}
    \item Les tables de conjugaison seront toutes relativement similaires à celles des verbes sains simples (cf. \ref{ConjSS43}), notamment au niveau des préfixes et suffixes\footnote{La différence majeure réside dans les clusters de consonnes, que le tunisien cherche toujours à éviter, et de la décomposition en syllabes des verbes conjugués.}. 
    \item Faites attention aux personnes et au temps pour lesquels il y a une inversion de lettres (systématiquement une consonne avec une voyelle).
\end{itemize}

\subsection{Schème de la voix causative}
Le verbe qui servira d'exemple est \textbf{fahhem} (\textit{faire comprendre}).

\begin{center}
\begin{tabular}{||c | c | c||}
 \hline
 \textbf{Pronom} & \textbf{Passé} & \textbf{Présent} \\
 \hline\hline
 'Éna & fahhem\textbf{t} & \textbf{n}fahhem \\ 
 \hline
 'Enti & fahhem\textbf{t} & \textbf{t}fahhem\\ 
 \hline
 Huwwa & fahhem & \textbf{y}fahhem\\ 
 \hline
 Hiyya & fahhme\textbf{t} & \textbf{t}fahhem\\ 
 \hline
 'A\textcrh na  & fahhem\textbf{naa} & \textbf{n}fahhm\textbf{uu}\\ 
 \hline
 'Entuuma  & fahhem\textbf{tuu} & \textbf{t}fahhm\textbf{uu}\\ 
 \hline
 Huuma  & fahhm\textbf{uu} & \textbf{y}fahhm\textbf{uu}\\ 
 \hline
\end{tabular}
\end{center}



\subsection{Schème de la voix passive}
Le verbe qui servira d'exemple est \textbf{tekteb} (\textit{être écrit}).

\begin{center}
\begin{tabular}{||c | c | c||}
 \hline
 \textbf{Pronom} & \textbf{Passé} & \textbf{Présent} \\
 \hline\hline
 'Éna & tekteb\textbf{t} & \textbf{ne}tekteb \\ 
 \hline
 'Enti & tekteb\textbf{t} & \textbf{te}tekteb\\ 
 \hline
 Huwwa & tekteb & \textbf{ye}tekteb\\ 
 \hline
 Hiyya & tketbe\textbf{t} & \textbf{te}tekteb\\ 
 \hline
 'A\textcrh na  & tekteb\textbf{naa} & \textbf{ne}tketb\textbf{uu}\\ 
 \hline
 'Entuuma  & tekteb\textbf{tuu} & \textbf{te}tkteb\textbf{uu}\\ 
 \hline
 Huuma  & tketb\textbf{uu} & \textbf{ye}tketb\textbf{uu}\\ 
 \hline
\end{tabular}
\end{center}

\subsection{Schèmes des voix réfléchie et causative-passive}
Les deux schèmes se ressemblant, le verbe qui servira d'exemple pour les deux est \textbf{t\c{a}allem} (\textit{apprendre}).

\begin{center}
\begin{tabular}{||c | c | c||}
 \hline
 \textbf{Pronom} & \textbf{Passé} & \textbf{Présent} \\
 \hline\hline
 'Éna & t\c{a}allem\textbf{t} & \textbf{ne}t\c{a}allem \\ 
 \hline
 'Enti & t\c{a}allem\textbf{t} & \textbf{te}t\c{a}allem\\ 
 \hline
 Huwwa & t\c{a}allem & \textbf{ye}t\c{a}allem\\ 
 \hline
 Hiyya & t\c{a}allme\textbf{t} & \textbf{te}t\c{a}allem\\ 
 \hline
 'A\textcrh na  & t\c{a}allem\textbf{naa} & \textbf{ne}t\c{a}allm\textbf{uu}\\ 
 \hline
 'Entuuma  & t\c{a}allem\textbf{tuu} & \textbf{te}t\c{a}allm\textbf{uu}\\ 
 \hline
 Huuma  & t\c{a}allm\textbf{uu} & \textbf{ye}t\c{a}allm\textbf{uu}\\ 
 \hline
\end{tabular}
\end{center}

\subsection{Schème de la voix réciproque}
Le verbe qui servira d'exemple est \textbf{tlèè\c{a}eb} (\textit{jouer avec quelqu'un}).

\begin{center}
\begin{tabular}{||c | c | c||}
 \hline
 \textbf{Pronom} & \textbf{Passé} & \textbf{Présent} \\
 \hline\hline
 'Éna & tlèè\c{a}eb\textbf{t} & \textbf{ne}tlèè\c{a}eb \\ 
 \hline
 'Enti & tlèè\c{a}eb\textbf{t} & \textbf{te}tlèè\c{a}eb\\ 
 \hline
 Huwwa & tlèè\c{a}eb & \textbf{ye}tlèè\c{a}eb\\ 
 \hline
 Hiyya & tlèè\c{a}be\textbf{t} & \textbf{te}tlèè\c{a}eb\\ 
 \hline
 'A\textcrh na  & tlèè\c{a}eb\textbf{naa} & \textbf{ne}tlèè\c{a}b\textbf{uu}\\ 
 \hline
 'Entuuma  & tlèè\c{a}eb\textbf{tuu} & \textbf{te}tlèè\c{a}b\textbf{uu}\\ 
 \hline
 Huuma  & tlèè\c{a}b\textbf{uu} & \textbf{ye}tlèè\c{a}b\textbf{uu}\\ 
 \hline
\end{tabular}
\end{center}

\subsection{Schème de l'aspect inchoatif}
Les deux verbes qui serviront d'exemples sont \textbf{\textcrh maar} (\textit{rougir}) et \textbf{zrèèq} (\textit{bleuir}).

\begin{center}
\begin{tabular}{||c | c | c||}
 \hline
 \textbf{Pronom} & \textbf{Passé} & \textbf{Présent} \\
 \hline\hline
 'Éna & \textcrh mar\textbf{t} & \textbf{ne}\textcrh maar \\ 
 \hline
 'Enti & \textcrh mar\textbf{t} & \textbf{te}\textcrh maar\\ 
 \hline
 Huwwa & \textcrh maar & \textbf{ye}\textcrh maar\\ 
 \hline
 Hiyya & \textcrh maare\textbf{t} & \textbf{te}\textcrh maar\\ 
 \hline
 'A\textcrh na  & \textcrh mar\textbf{naa} & \textbf{ne}\textcrh maar\textbf{uu}\\ 
 \hline
 'Entuuma  & \textcrh mar\textbf{tuu} & \textbf{te}\textcrh maar\textbf{uu}\\ 
 \hline
 Huuma  & \textcrh maar\textbf{uu} & \textbf{ye}\textcrh maar\textbf{uu}\\ 
 \hline
\end{tabular}
\end{center}

\begin{center}
    \begin{tabular}{||c | c | c||}
     \hline
     \textbf{Pronom} & \textbf{Passé} & \textbf{Présent} \\
     \hline\hline
     'Éna & zreq\textbf{t} & \textbf{ne}zrèèq \\ 
     \hline
     'Enti & zreq\textbf{t} & \textbf{te}zrèèq\\ 
     \hline
     Huwwa & zrèèq & \textbf{ye}zrèèq\\ 
     \hline
     Hiyya & zrèèqe\textbf{t} & \textbf{te}zrèèq\\ 
     \hline
     'A\textcrh na  & zreq\textbf{naa} & \textbf{ne}zrèèq\textbf{uu}\\ 
     \hline
     'Entuuma  & zreq\textbf{tuu} & \textbf{te}zrèèq\textbf{uu}\\ 
     \hline
     Huuma  & zrèèq\textbf{uu} & \textbf{ye}zrèèq\textbf{uu}\\ 
     \hline
\end{tabular}
\end{center}

\section{Quelques mots}
Ce chapitre est déjà relativement long, et je vous ai déjà plus que submergé d'informations. Je ne souhaite pas l'allourdir davantage, mais il y a plusieurs points que je tenais quand même à évoquer rapidement, d'une part pour compléter votre compréhension du tunisien, et d'autre part pour votre culture générale (si ça vous intéresse).

\subsection{Sur les verbes importés et leurs dérivés}
Quelque chose de particulièrement vrai à propos du tunisien, et dans une mesure qui m'est inconnue à propos de l'arabe, est la facilité d'importer des mots de d'autres langues, et de les forcer dans des moules existants pour les faire obéir à la logique interne de la langue. 

Dans le cas de \textbf{l'arabe}, on peut par exemple parler de \RL{خَرِيطَة} \textbf{/xarii\c{t}a/} (\textit{une carte}) et \RL{قرْطَاس} \textbf{/qar\c{t}aas/} (\textit{du papier}) qui dérivent tous les deux d'un seul même mot \textbf{grec} $\chi\alpha\rho\tau\eta\zeta$ \textbf{/khártēs/}\footnote{Vous aurez sans reconnu l'ancêtre du mot \textbf{carte} en français.}. Le mot \textbf{/xarii\c{t}a/} tout particulièrement a été introduit dans le moule des bases triconsonantiques, et en est sorti \textit{de force} la base \textbf{X-R-\c{T}}.

De cette base, beaucoup d'autres mots on été dérivés, aux sens potentiellement exotiques pour certains : \textbf{/xara\c{t}a/} (\textit{dépouiller}), \textbf{/maxruu\c{t}/} (\textit{un cône}), \textbf{/'enxara\c{t}a/} (\textit{dégainer}) ou encore \textbf{/'exrawwa\c{t}a/} (\textit{être emmêlé}).

On pourrait trouver encore beaucoup d'autres exemples de mots de langues voisines qui se sont intégrées à l'arabe au fil du temps, certains qui je sui sûr étonneront plus d'un arabophone. Il est important de se dire que ce système d'emprunt existe encore de nos jours en \textbf{tunisien}. Ne vous étonnez donc pas de trouver plusieurs mots qui vous sont familiers qui ont été plus ou moins \textit{charcutés}, soit pour en extraire un équivalent de base triconsonantique, soit pour leur appliquer directement des schèmes existants.

Quelques exemples notables : 
\begin{itemize}
    \item \textbf{riigel} (\textit{réparer}), qui vient du français \textit{réguler}, qui se décline par exemple en \textbf{triigel} (\textit{se faire réparer}) ;
    \item \textbf{trééna} (\textit{s'entraîner}), qui vient du français \textit{s'entraîner}, qui s'est vu extraire une racine : \textbf{rééna}. Cette extraction a notamment été permise par l'assmiliation du \textbf{/t/} de \textit{s'en\underline{t}rainer} à un schème de la \textbf{voix passive}.
    \item \textbf{barres} (\textit{débarasser la table}), qui vient du français \textit{débarasser}, qui provient sans doute de l'assmilation de \textit{tu débarasses} à une forme \textbf{'enti tebarres / tbarres}, d'où la forme \textbf{huwwa barres}.
\end{itemize}

Ces quelques notes pour vous dire que l'ensemble des mots importés en tunisien sont suceptibles d'être adaptés à souhait, et dérivés à leur tour. Le système de dérivation des verbes est donc sans doute plus \textit{fluide} que ce que le début du chapitre peut laisser penser.

\subsection{Sur les bases quadriconsonantiques}
XXX

\subsection*{Dialogue}
\subsection*{Vocabulaire}